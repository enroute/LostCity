\chapter{大雨枪王庙}
\label{chap:da-yu-qiang-wang-miao}


路程虽短,奈何洪小黑人小力气小,待到进了枪王庙后,二人早已成了落汤鸡。洪小黑有点虚脱地倒在地上大口喘气,二人身下慢慢渗满了衣物上溢出来的水。庙外乌云遮天,天色已然如入夜般阴暗。伴随大雨而来的东风从早已破败的窗户带进来阵阵凉意。

“不行,这厮已然昏倒,若任由这般下去,早晚会被风寒夺了性命,想来阎王还是会将罪过怪到我头上来。”念及此处,洪小黑不知从哪来借来了神力,马上爬了起来,在枪王庙里寻找起来。

此枪王庙本是当地民众为祭奠大宋神将杨将军而自发修建的小庙,随着时光流逝,朝代更迭,物是人非,枪王庙早已破落得成了残垣断壁,不再有香火。

洪小黑幸运地在残破的枪王像下找到了些许干草,想来是哪个过路的同行乞丐用来休息之物。“前辈,借你干草一用。枪王爷爷,休怪莫怪,借你金身一用。所谓救人一命,胜造七级浮屠,想来你也会支持我的。”洪小黑从枪王像中取了块稍硬的残片,击石取火,将口中那前辈留下的干草点着了。二人便围在这微小若明若灭的火光中取暖,只不过一个是蹲坐在地,一个是大字躺地。过不多久,前辈的床铺都被洪小黑全借用完了。

就在火堆在草料烧完即将熄灭时,只见岳飞羽开始抽搐起来,有一下没一下的。洪小黑正盯着岳飞羽呢,毕竟是小孩心性,被他举动吓得跳将起来:“这厮又要诈尸了吗?”洪小黑哆哆嗦嗦地慌忙连走带爬地远离岳飞羽,在一墙角处抱膝蹲着,慌张地看着岳飞羽。

原来,岳飞羽本就有顽疾,受了风寒之后又被火气诱发,便发作起来了。

眼见岳飞羽抽搐得越来越厉害,口中发出野兽般的低吼,洪小黑心中的惊恐马上积聚到了极点,在即将爆发之际,突然屋顶硕大一颗水珠滴在洪小黑额头上。洪小黑便如醍醐灌顶般清醒了过来。“七叔叔说过,世间本无鬼,不过是凡人的业障罢了,但万法相通,鬼怪论的妙方用到凡人的业障有时倒也有奇效。如今看这厮一惊一咋,又沉睡不醒,似乎处在二魂与三魂游走不归之间。看在馒头包子的份上,且看洪老爷大发慈悲助你岳小八渡过此劫罢,”洪小黑回想起七叔叔说过的招魂术,有模有样地舞起来。洪小黑滑稽地上蹿下跳,左摇右摆,跳了一会,便自觉好笑,心中笑骂:“便宜你岳小八了,本大爷还从来未曾跳得如此可笑过。阎王老爷,小子对岳小八已是仁至义尽,这本是他自身劫难,若渡不过去,你老人家要明辨是非,莫要怪到我头上来。”跳到最后,洪小黑高高跃起,落下时一指头点在岳飞羽额头上,大声喝到:“岳小八,你回来罢!”

洪小黑定格在最后一击的姿势,任由不时漏进来的风吹得衣衫猎猎作响。

“失败了。”洪小黑看着免疫招魂的岳飞羽,暗想,俄而跌坐在地,旋即不甘心地一跃而起,再次使出一套独创招魂术,在岳飞羽身上四处敲打起来。

“小子,你再这样敲打下去,乱了他经脉,就真的魂飞魄散回天无力一命呜呼了。”

听着这突如其来的苍老却有铿锵有力的声音,洪小黑再次被吓了一大跳,立马停下动作,扭头看去。只见枪王庙入门处不知何时站了个仙风道骨的道人。经过洪小黑此番折腾,火堆早已尽灭,只剩冒着点点火星的灰烬。隐约间只见那道人披着一身玄色道袍,梳得整整齐齐的长发在头顶盘成了个标准的道髻,背着个若大的葫芦最为惹人耳目,空无一物的双手自然地垂放在着,几可够到膝盖处。显然声音出自老道人之口。看着那道人,洪小黑无由头的心中大定,但看他口中略略含笑,似乎在嘲笑自己方才的滑稽行为,便心中恼怒,坐在一旁扭头不理他。

那道人见洪小黑不理自己,也不觉生气,微微一笑,径自走到岳飞羽身前,蹲下给岳飞羽把脉。在即将抓到岳飞羽的左手腕时,道人似乎心有所感,道袍猛然一鼓,在空中激起阵阵涟漪,神奇的是,地上杂草及火堆的灰烬却无动于衷,仍然安安静静地躺在地上逍遥自在。“小友放心,贫道清风子,断无要害你之理。”那道人声音不大,但远远地送将出去,只惊得枪王庙附近密林中躲雨的大鸟四处飞散。

洪小黑本在恼怒或许被那道人看到了自己的丑态,感受到微微激荡的气流,再听他如此一说,不由转过头来,诧异地看着那自称清风子的道人。只见清风子握住岳飞羽的左腕,闭目沉思。过了良久,清风子睁眼,叹了一口气,喃喃自语道:“枉我清风子自命熟读黄帝内经阅尽天下奇难杂症,不曾想今日之症却是闻所未闻,远非老朽之学所能及。今日方知山外有山。”清风子自语半天后,摸出一颗猩红的药丸,拇指般大小。“小友,老朽虽未能将你根治,但缓解你此刻的痛苦却还是力所能及。”清风子将药丸递给洪小黑:“小子,你若想他醒过来,便将这药丸给他服用了吧。”

洪小黑接过药丸,闻着其散发出来的沁人心脾的阵阵清香,强行压住想要生吞活咽下去的冲动,将信将疑地说:“听说最艳丽的蘑菇通常都是有毒的,最斑斓的蛇通常也是巨毒的,据说海外还有一种色彩绚丽的青蛙,更是毒王中的毒王。老先生你这个药丸这么好看这么香,该不会......”洪小黑心中还有一句话没说出来:这老头该不会是想借我的手毒死岳小八吧?

清风子一听,气得脸都绿了,可自持身份又不好对小辈发作,刚举起的想指着洪小黑的手顺势就把药丸夺了过来,一把便灌进了岳飞羽的嘴里,颤抖着声音大声喝到:“我便毒死了他给你看!”。岳大公子却也了得,昏迷中刚要叫喊一声诺大个药丸就顺势滑落进去,然后卡在喉咙里呜呜地出不了声,本来苍白的脸也慢慢涨红了。清风子便在他胸口处推拿了一番,只见岳大公子喉咙滚动了一下,洪小黑见状心中一喜,进去了,至少不会被噎死。

过了一柱香功夫,岳飞羽渐渐平静下来不再抽搐。

“好你个岳小八,睡得倒舒坦。”事情告一段落,洪小黑才想起自己早上到现在都还滴水未沾,肚子开始敲锣打鼓起来。清风子也是个妙人,本来还在跟洪小黑这小孩儿斗气,听他肚子一叫,便乐了,动作夸张地从怀里摸出一个大煎饼,放在鼻子下用力的闻了一下,然后得意洋洋地咬了一小口,在嘴巴里嘎嘣嘎嘣地大声咀嚼起来,脸上神色要多惬意有多惬意。可洪小黑不乐意了,心中暗暗诅咒他:“个糟老头子,那么大个煎饼那么香,小口小口吃也不怕撑死你。”洪小黑恨不得亲自上阵帮他把煎饼一口就吞到肚子里。洪小黑信奉的是到了肚子里的才是自己,有得吃就赶快吃抓紧吃吃撑了还要吃。

清风子乜斜着眼睛看着洪小黑口水都快要滴到地上了,心里别提有多痛快了:小娃子,想跟老夫斗,你还嫩得很,老夫有的是招式对付你。

“老先生,你的煎饼上有只蜘蛛。”

怎么可能,世上还有能逃过我法眼静悄悄来我面前的存在吗?清风子冷哼一声,小娃子想嫌我来着。突然清风子想起一事,暗叫不好,难不成是南苗神凰来了?如果是此人的毒虫,我还真无法察觉。清风子天不怕地不怕,平生只惧三人,南苗神凰便是其中之一。念及此处,清风子急忙睁眼一看,煎饼上果真隐约有个小黑影,他心里想着南苗神凰便以为这黑影是南苗神凰的毒物,便怪叫一声马上把手上的煎饼远远地丢了出去,飞到庙外不见了踪影。洪小黑暗道可惜了好香一张煎饼。

“老先生,那蜘蛛往你怀里爬进去了!”

这还得了,清风子赶紧伸手往怀里掏,掏到什么扔什么。洪小黑有了上次经验,稳稳当当地接住了清风子从怀里扔出来的几个煎饼,就大口地吃了起来。

清风子把怀里的东西扔完了才发现洪小黑在大口大口地吃着煎饼,马上明白了事情的缘由,暗道着了这小娃子的道了,不由老脸一红,自言自语地说:”唉,现在的小娃娃,连老头子都要骗,真是人心不古哇。”虽说是清风子的自言自语,然而每个字却清晰无比的传入到洪小黑的耳朵里。洪小黑自是免疫此等言语,否则凭他一个小乞丐如何能活到现在。清风子见洪小黑毫无惭愧羞耻之心,乃心生一计欲以其人之道还治其人之身,于是装做突然想起的模样,大叫道:“哎呀,老夫的醉魂散怎么撒了,糟糕,小娃子,快停下,那煎饼上沾了老夫的醉魂散!”

洪小黑饿极正吃得带劲,哪里管它什么醉魂散,只管吃饱肚子,管它洪水滔天。清风子虽童心未泯但本不是多心之人,见此无效也就黔驴技穷了,暗道好个土包子,连老夫的醉魂散也不晓得厉害。