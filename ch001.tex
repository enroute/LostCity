\chapter{寒雨连江夜入吴}
\label{chap:han-yu-lian-jiang-ye-ru-wu}

\psentence{
寒雨连江夜入吴,平明送客楚山孤。\footnote{\ptitle{唐}{王昌龄}{芙蓉楼送辛渐}寒雨连江夜入吴,平明送客楚山孤。洛阳亲友如相问,一片冰心在玉壶。}
}

清明时节的靖安镇,细雨在斜风中淅淅沥沥地飘着。黄昏时分,日薄西山,华灯将上未上时,靖安镇街道上显得格外的冷清。除了几个行色匆匆的赶路人以及他们引起的犬吠声,这个位于江宁府的小镇似乎早早地就进入了冬眠。

一条官道的尽头,是靖安镇最繁华的同福客栈。虽是最繁华,此刻也只有稀稀拉拉的几桌客人在低语吃饭。

“四弟,此去一别,不知何日才能重逢了。”靠墙角的一桌,一个国字脸的雄壮中年人不胜唏嘘地感叹着。

坐在他对面的虬髯大汉看着他的同伴,感受着他不经意间散发出来的威严,胸中的热血再次沸腾起来:“大哥,天大地大,总有属于我们的地方。等我这趟把三哥找回来,咱哥几个再痛痛快快地喝一场!”

风萧萧兮易水寒\footnote{\ptitle{战国}{荆轲}{易水歌}风萧萧兮易水寒,壮士一去兮不复还。探虎穴兮入蛟宫,仰天呼气兮成白虹。}。雄壮中年人喃喃自语,然而他并没有说出来让他的兄弟扫兴。“四弟,此去一为别\footnote{\ptitle{唐}{李白}{送友人}青山横北郭,白水绕东城。此地一为别,孤蓬万里征。浮云游子意,落日故人情。挥手自兹去,萧萧班马鸣。},不管前路若何,谨记靖安还有几个老哥哥。我只嘱咐你一句话,不可以身犯险!”

“大哥你放心,俺好歹跟在二哥身边多年,总也见过猪跑不是?”

二人复长饮,直到客栈打烊方休。

\splitline

靖安镇东隅一所大宅子,是本地殷实人家岳府。二更时分,中年人步伐稳健地走进了进去。

“大哥,四哥走了?”岳府简朴的书房里,在两盏摇曳的烛光下显得格外的宁静。一中年妇人满脸疼惜地看着中年人,温柔地问道。

二人正是岳府的主人岳东来及其发妻沐夫人沐素衣。

岳东来轻轻地将沐素衣拥入怀中,轻声地责备她:“素素,跟你说过多少回,不要熬夜等我。你说你有哪回是听我的。”沐夫人安静地享受着丈夫宽阔温暖的胸膛,浅笑不语。

良久无语,夫妻二人仿佛融入了天地,都在依恋着这片刻的安宁。

“啊------”

突然间不远处传来一声低呼。两人随即惊醒,对视一眼,心有灵犀地说到:“羽儿!”随即同时奔向书房右手侧不远处的厢房。只见厢房一床上躺着个七八岁的小男孩,双眉紧闭,口中发出啊啊的叫声。他的叫喊声是如此小,似乎是在努力强迫自己不喊出来。

“这些年来真是苦了这孩子。”沐夫人轻轻地擦拭掉小男孩额头上的细细汗水,又轻轻地拍着他的背。小男孩似乎感受到了母亲的关怀,不一会又沉沉地睡去了。

岳东来沉默地站在床边,紧握双拳,紧咬钢牙。

“这些年来亏得有二哥的细心照料,否则这苦命的孩子......”说到这,沐夫人用长袖抹了一下眼角。然后好像抓住了救命稻草般,眼中也露出了一丝希望的光芒,急声问到:“大哥,你说四哥这次能成功么?”

“素素,我和二弟四弟都已经计划妥当,你就安心等着四弟的好消息便是。”岳东来安慰着妻子。

夜色在夫妻二人的相互低声慰藉中慢慢褪去。随着宅中雄鸡响亮的一声,岳府忙碌的一天就渐渐开始了。

\splitline

“来福,你一大早的慌慌张张到处乱跑什么呢?”岳东来几乎一夜未合眼,但仍看不出丝毫的疲态。他逮着年轻的来福问到。

岳家虽说家境殷实,但下人数量也可一手数得过来。这个来福是个高壮憨厚青年。听到岳东来的问话,他急忙停下来回到:“老爷,飞羽少爷又不见人影了,我在宅子里都找遍了还是没找到他。”

岳东来略一沉吟,便道:“我知道了,你可以暂时不用管羽儿了,先去忙其他的事吧。”

“不行啊,老爷!万一飞羽少爷跑出去又被人打破了头那可怎生了得。对了,我要出去找一找。”说完,来福也不管岳东来这个主人,竟径自一溜小跑出了岳府,转眼便无影无踪了。

岳东来摇摇头,转身对旁边的管家金伯说:“金伯,双儿回来一事可都已经准备妥当了?”

“老爷,前日听说消息后,老奴昨日便领着大伙收拾妥当了。”

“老爷,老爷,门口来了好多官差------”一个下人突然慌慌张张地跑进来。

“来德,不用着急,你且说说来了多少人,领头的是谁,骑马没骑马,都说了些什么话......”

来德一个也不知道,他说:“老爷,我一看到那么多官差就赶紧跑回来告诉您了,其它的我都没有留意啊。”

“无妨。”说完,岳东来便领着金伯走向岳府大门。

\splitline

“原来是袁统领,失迎失迎。袁统领来到岳某这里,真是令某家寒舍蓬荜增辉呀。不知统领是否有空进寒舍喝口茶水?”岳东来看着府门前一字排开的十八骑高头大马上武装整齐的士兵,不卑不亢地对领头的袁克敌打着招呼。靖安是个小镇,最大的武官也不过是百夫长,岳东来称袁克敌为统领是有意抬高他了。

“你我素无交情,茶水便不必了。袁某是公干来的,奉县官老爷的令,请岳先生到县衙门走一趟。”袁克敌坐在他那西北黄毛战马上冷冰冰地说,无视岳东来的奉承言语,屁股都不曾挪一挪。

“既是县衙大人有请,小民自当前往,何必惊动袁统领亲自走一趟。不知县衙大人有何吩咐?”

“袁某是军人,只是奉命行事,不敢过问大人们的事。岳先生,请吧。”

岳东来对金伯使了个眼色,暗示他不要冲动,对他说到:“金伯,我去去就回,你回去告诉夫人今天的晚饭可要准备得热热闹闹的。”说完,转头对袁克敌打了个手势,“袁统领,请。”

袁克敌脸上一丝诧异转瞬即逝,紧握着战刀的右手也暗暗松开,随即带着十八骑及岳东来往县衙门方向缓缓而行。

\splitline

岳府正堂里,金伯把岳东来临走时所说的话报于沐夫人。

“既然老爷这么说,想必他已有对策。再说,谅那几个燕骑也奈何不了老爷。金伯,此事不用声张,只管安排既定事项即可。今日双儿回来,万事不可出纰漏。否则那混世魔王......”说到此处,沐夫人又倍觉头疼。转念想到醒来时盖着的被子,心中不免又多了一丝温暖。

金伯脸部猛然抽搐了下,干笑一声,说到:“夫人放心,老奴再去检查一番,势必不给少爷留下丝毫借口。”说完,金伯一脸冷汗匆匆而去。

\splitline

却说岳家公子岳飞羽清晨醒来后,俨然忘了昨夜的噩梦。

“娘怎么又趴在我床边睡过去了?”\marginpar{岳东来呢??怎么能任由他夫人趴着睡着了不管!}岳飞羽喃喃自语。他人小力气小,抱不动娘亲,就扯过床上的被褥给母亲盖上,小心地穿上衣服,蹑手蹑脚地走出去,再轻轻地带上房门,然后一下跑没影了。

飞羽轻车熟路地躲开所有的下人,来到后院一个墙角,东张西望一会确认附近无人时,鬼鬼祟祟地从杂草中掏出一根草绳,抛到墙头突出的石柱上,顺着绳子就爬了上去。

“小哥我自由了!”爬上墙头的飞羽撇嘴微笑,纵身一跃就跳到墙外了。

“咚!”

摔了个四脚朝天的岳飞羽摸摸头上鼓起的大包,警惕地四周望了望,松了口气:“幸亏没人。”

“小王八,爬呀爬,爬上墙头要跳崖。墙又高,风又大,摔到地上也不怕。幸亏有个大龟甲,摔疼不用找妈妈。”

听着这准时出现的嘹亮的童音,岳大公子脸一黑,爬起来就跑。没想却与唱童谣的小孩儿撞了个满怀,呯的一声再次四脚朝天。岳大公子皮糙肉厚,马上又爬了起来,定睛一看,脸再次黑了起来。“真是晦气。洪小黑你是故意站那里等我撞的吗?”

那叫洪小黑的小孩儿,一身乞丐打扮,又黑又矮又瘦小,衣衫虽褴褛但倒也算干净。洪小黑哎哟哎哟地缓缓爬将起来,白了岳大公子一眼,说:“你还恶人先告状了啊?我告诉你,别看你皮糙肉厚带个王八甲经打,再惹我,我,老......老子一招分花拂柳手保证让你三天下不了床!”洪小黑说完站好,似乎意犹未尽,于是又补了一个重重的“哼”字。

岳大公子回想起上次地狱般的惨状,立马冷汗涔涔。然而他嘴上却不肯服输:“我怎么没听说过分花拂柳手是你们丐帮的武功呀?”

洪小黑一时语塞,愣在原地,黝黑的脸上现出了些许红色。

岳大公子大仇得报,得意扬扬。“哼,洪小骗子,收起你那张嘴就来的谎话,哥可不是被你吓大的。”岳大公子雄赳赳气昴昴地走向洪小黑。洪小黑气极,正欲赏他一记分花拂柳手,却见岳大公子脸色大变,拉着他就跑进拐角的胡同里。

“岳------唔---唔---”洪小黑还没来得及说出第二个字,嘴巴就被岳大公子的手捂住了。“嘘------别出声。来福出来找我了。“岳大公子轻声对洪小黑说。


\splitline

二人偷偷摸摸跑得气喘吁吁地躲过来福,来到靖安镇西南空地杂草丛中,岳大公子松开紧开一直紧抓着的洪小黑的手,双手撑着膝盖弯腰大口喘气。洪小黑看着被岳大公子抓得黑里透红的手腕,怒目圆睁地盯着岳飞羽,只等还不自知的岳大公子点起一丁点火星就要爆发出来。可惜过了良久,岳大公子只顾喘气,没做任何动作发出丝毫言语来点爆洪小黑,洪小黑的怒意火山也就慢慢冷却下来了。

“洪小黑,你个天生的扫把星,我每次遇到你准没好事。”

完蛋了,洪小黑冷却的怒意找到了出口立马爆发了。

“分花拂柳手!”

洪小黑哪里还管这个招式是否丐帮所传,只管用生平最得意威力最大的招式招呼过去。瞬时只见杂草与尘土飞扬,呯呯与哎哟同现,吓得杂草丛中觅食的雀儿与不知名的小鸟齐飞。过了约摸盏把茶的功夫,洪小黑自己先累了,便停下手来,冷的如刀般的目光却依然在围剿着岳大公子。可惜洪小黑的杀人目光没有效果,可怜的岳大公子正捂着肿得像猪头般的脸嗷嗷惨叫,根本腾不出空暇来感受洪小黑的这一最后杀招。

好在岳大公子皮实,叫了一会便觉得不痛了,便想着如何找回场子,然而寻思过后又自知打不过不知从哪里学了两手歪招的洪小黑,便索性继续躺在地上慢慢想法子。洪小黑见他突然一动不动,紧闭的眼珠似乎在不停的转动,心中吓了一跳,寻思不会是错手把这厮给打死了吧。洪小黑用脚拔了下岳大公子的腰间,大声喊到:“快起来,别给洪大爷装死。”岳大公子正想着法子呢,一下被洪小黑的惊天吼打断了思路,便气呼呼地要与他作对,你叫我起来我偏不起来。洪小黑轻踢了几下后,见岳大公子还是没有动静,甚至连眼珠子也不再转动,一下便失了举措,慌忙俯下身去查探岳大公子的鼻息,只觉岳大公子的鼻子只有细微往外出的气,便觉天崩地裂,豆大的泪水瞬间涌出眼眶流下脸颊滴到岳大公子的脸上。洪小黑抓着岳大公子的衣领,泪眼婆娑断断续续地说着:“平日里打你三五招都没事,为何今日却如此孱弱?岳小八,我------”

岳大公子本就觉得脸上洪小黑的眼泪奇痒无比,再听得洪小黑叫他这个名字,便再也忍不住,一下跳将起来,怒喝到:“你说谁是岳小八?!”

洪小黑被他吓了一大跳一屁股跌坐在地上,黑漆漆的脸都变绿了:“你,诈......诈尸了?”

岳大公子跳起来本就为吓他一吓,跳起来后便觉此仇已报此恨已销,心情舒畅无比,本想顺着洪小黑的言语装做个诈尸的怪物继续吓他一吓,以便扳回一城扭转下与洪小黑对垒以来一直处于劣势的状态,但此刻见洪小黑似乎真被吓着了,便觉不忍,于是恢复常态,轻声对洪小黑说:“洪小黑,我是吓你的呢,我这么结实防御无敌的人,怎么可能是你的花拳绣腿挠痒手所能伤得了的呢?”

洪小黑将信将疑,但一看他那嬉皮笑脸样,再想到传说中鬼怪都是阴森可怕的,便知遭他骗了一回。

“分花拂柳手!隔山打老牛!好你个岳小八,嫌皮痒还没打够是吗?”

于是杂草丛中再次上演了鸡飞狗跳。等到洪小黑再次停下来的时候,岳大公子已经奄奄一息地躺在草地上了。

“啪嗒”一声,突然又有一滴水落到岳大公子的脸上。岳飞羽闭着眼睛想,这小黑子不会又被我骗了吧?他仔细感觉了一下气氛,没有听到洪小黑的任何异动,便知道不是。“啪嗒”又一滴。岳大公子再顾不得许多,睁眼一看,原来天不知道何时已然暗了下来,隐隐有暴雨来临之势。

“洪小黑快跑啊,再不走就要成洪汤鸡了!”岳大公子边说边跑,全然忘了自己还在装死当中。洪小黑早已看到天色变暗,只是被岳飞羽气到了,没去考虑淋雨一事。如今听他一噪子,便跳起来跟着岳大公子一起跑了。跑到半路,突然回味过来,怒声喝到:“岳小八,你刚才说谁是洪汤鸡?!”

% \splitline

“真是天有不测风云。双儿的师傅都算过今日是黄道吉日,没想成今日还是要下起雨来了。金伯,令来齐早点出发多走点路去接双儿吧。”沐夫人看着满楼风雨,露出一丝担忧的神色。

“是,夫人。”

“羽儿还没回来么?”沐夫人对尚未走远的金伯问到。

“是的,夫人。来福已经去出去找少爷了,老爷说无妨,我料想应该也是无妨的。”

“也是,在靖安这个小镇,想来应该也是无妨的,就由他胡闹去吧。但金伯,他若是闹出些伤天害理人神共愤之事,你可不要帮着他来瞒着大哥和我,你也瞒不住。”

“夫人,少爷本性淳厚,被人伤被人害倒有可能,小打小闹也有可能,真要是伤天害理之事则是断无可能的。”

% \splitline

“阿嚏------”正在飞奔中的岳大公子打了个大喷嚏停了下来,心里正盘算着是谁在暗地里说诅咒他,突然被后面追赶而至的洪小黑撞了一下,一个踉跄下盘没稳住,趴在地上摆了个大大的大字。原来洪小黑一心要追上他问个明白,没想到他会突然停下来,惯势使然再加上心里就想着把他撞倒出口恶气,便全力撞了岳大公子一下。有心算无心,再加上岳大公子力气本就没洪小黑大,不倒地上才怪了。

看着岳飞羽在已见泥泞的路上摔得像个泥人般,洪小黑不由欢声大笑起来,终于是出了一口恶气,不枉他追了岳飞羽一路。

“喂,起来了,别再装死了。”

过了半晌,雨渐渐大了起来,岳飞羽与洪小黑的衣服都已经被淋透了。洪小黑见岳飞羽在他的连吼之下仍不见动静,忙蹲下去。只觉岳大公子的四肢已渐冰凉,“还好仍有呼吸。”洪小黑似乎见过鬼之后就不再害怕了,这次倒镇定了许多。“该不会真被我一下给撞傻了吧。唉,念在你好歹请我吃了那么多馒头的份上,洪大爷今日便也发发慈悲,拖你回家吧。”

于是路上多了一个瘦弱的身板在雨中拖着一个重物在雨中一步一挪地行走的风景。

“岳小八,你是属猪的吧,怎么这么重。”

突然天空一白,紧接着轰隆一声巨响,吓得洪小黑把拖在手中的岳飞羽的手都丢了。原来天空中一道极长的闪电直接划落到远处的地平线上,把阴暗的天空一劈为二,很有盘古开天辟地的气势,也很吓人,比如吓到洪小黑。

“妈呀,报应不会来得这么快吧,他都没死呢就来劈我了?”洪小黑心里发憷,便寻思着要找个地方躲避一下,好不让雷劈到,先渡过这一劫再说。洪小黑四处看去,见到不远处似乎有个房子的影子,心中一喜,便想起那该是靖安西郊早已破落的枪王庙的方位。洪小黑再次看向仍然躺在地上的岳飞羽,心中暗到:“岳小八,虽说有冤报冤,但也要有恩报恩,今日我带你去枪王庙好让你躲过这一场雷劫,日后不论你是人是鬼,都不要来找我报仇。”此念一起,便拖着岳大公子往枪王庙而去。


