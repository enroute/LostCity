
\chapter{吾有仲子,忘忧忘情}
\label{chap:wang-you-wang-qing}

密道里黑得伸手不见五指,只闻得到一股久了不通风的霉味。雁南飞在密道中摸索到了岳飞羽及沐夫人的包裹,深一步浅一步地往前挪动。密道狭窄,呯的一声雁南飞背上的岳飞羽撞到了密道顶上凸出来的石块,带得雁南飞一个踉跄摔倒在地,身上的包裹也掉地上了。

“羽儿,你没事吧?”雁南飞急忙问道。见无人回应,才想起岳飞羽仍在昏迷当中。

雁南飞伸手要去摸那包裹,胡乱摸索间手臂竟用力将那包裹拔了一下。只见那摔下来后本已有些松散的包裹此刻发出些微弱的光芒,显然是包裹松开之后里面物品暴露了出来。就着这些光芒,雁南飞打开包裹查看,一张厚厚的黑布里赫然包裹着一颗拳头般大小的夜明珠!在夜明珠的照耀下,密道登时亮如黄昏,一切地形与布置看得清清楚楚。雁南飞在珠光下检查了包裹中的物品,除了那夜明珠外,还有一个散发着淡淡蓝光的玉佩,玉佩入手亦温亦凉,一面雕刻着喷火的五爪金龙,一面雕刻着一昂首飞翔的凤凰,一龙一凤栩栩如生。此外包裹里还有一把暗淡无光的短匕首,以及金叶、碎银与银票若干。

“大嫂倒想得周到。”雁南飞收拾了包裹,再次背起岳飞羽,依着夜明珠的亮光顺着密道往前走。

走出百米距离,便听得轰的一声闷响,刚走过的密道坍塌了!滚滚尘土向雁南飞二人直扑而来,良久才平息下来。雁南飞二人除了满身灰尘之外,倒是毫发无损。

雁南飞一阵默然,只觉喉间哽塞。以他之智,自是知道沐夫人此举之意,是下定了决心要为他二人争取时间。雁南飞虽非武林中人,但也豪迈非凡,很快便拾掇了心绪,再次快步而去。

雁南飞以步计时,算着在密道中走了大概有三个时辰,终于依稀见着了前方有微弱的亮光。走近了看,原来天色已晚,那亮光是从密道尽头的洞口处倾洒下来的月光与星光。雁南飞心中计算了一番,估算现在位置大约是在东山山脉深处,便谨慎起来。走到洞口一看,果然洞口开在了悬崖峭壁之上,向下看黑漆漆一片不知深浅。

雁南飞心想唯有等天亮再做打算,便寻了个靠里的位置将岳飞羽安置好,他自己坐在外侧以防岳飞羽翻身坠落山崖。他一边闭目养神,一边将日间之事在心中反复推演,也没理出个靠谱的头绪,便也缓缓地睡去了。

第二天清晨,雨后的晨光柔和地洒入洞内。雁南飞慢慢睁开眼,坐起来朝岳飞羽的方向瞥了一眼,便蓦地跳了起来。岳飞羽不见了!雁南飞虽然人称神相,素来谋而后动,然而此时情急之下,也只剩余大喊的本能。他羽儿羽儿的大喊着,可是只有洞内的回音在回应他。雁南飞便静了下来,借着晨光查看四周,思索对策。

过不多久,洞外传来隐隐约约的窸窸窣窣的声音,在寂静的洞穴中听得格外清楚。雁南飞心中一惊,难道是蛇?雁南飞正欲寻个有利的地形,找个称手的武器,却只见一个小小的黑影从洞外一跃而入,直接跳到了他的肩头!雁南飞大惊,哪里还管他是蛇不是蛇,有毒无毒,慌忙用手去拔。那小黑影却也灵活得紧,见雁南飞双手拔来,便灵巧地一跳,立于两米开外,躲过了雁南飞的双手。

雁南飞定了定神,朝那黑影看去,原来是一只漂亮的花栗鼠。只见那花栗鼠毛茸茸的,背上一道与众不同的显眼的紫色纵纹,尾巴宽大到能把它自己包裹起来。那花栗鼠也不怕人,正用那滴溜溜的眼珠子盯着雁南飞。雁南飞被它那宛如人般的眼神打量得上下不自在。雁南飞见多识广,知这花栗鼠必有灵性,想来非是寻常野兽,便欲想法子要将它捉了来。

“呯”雁南飞正想着呢,突然听得一声响,洞中又落下了一个人影,赫然便是岳飞羽。雁南飞见状,又惊又喜,早将捉拿花栗鼠一事忘了,赶紧过去扶起岳飞羽,仔细检查他身上是否完整。只见岳飞羽口青鼻肿,两根香肠般的嘴唇红的发亮,脸上其余地方倒是完好无损。雁南飞连忙问他缘由,可是岳飞羽只能哇哇比划,说不出话来。

比划了半天,连猜带蒙,雁南飞总算是大概知道了事情的经过。原来岳飞羽昏迷多时,天刚亮时便先于雁南飞醒来,见状不知发生什么变故,想问个明白却又不忍将雁南飞叫起。正无聊间,见着那个花栗鼠在洞口盯着他看,好奇心起,便跟了过去。那花栗鼠见他过来,便跳出洞外,灵活地顺着峭壁往上爬走了。见洞口连着一粗壮树枝,自幼胆大的岳飞羽便抓住树枝爬了上去,见那花栗鼠正在上方另一洞口处吃着不知名的果子呢。算来岳飞羽也是有几顿没吃了,见着花栗鼠吃得有滋有味,便觉饥肠辘辘,借着树枝落脚便也爬到了那洞中,抢了花栗鼠的果子便吃,直气得那花栗鼠哇哇乱叫。奈何岳飞羽体型比它大太多,那花栗鼠见抗议无效,便跳上到他头上,高高跃起,使劲在他头顶踩了一下泄愤。岳飞羽吃得急,被它一踢,被一果子呛着了,咳了半天。稍稍舒服后再去寻那花栗鼠报仇时,却早已不见了它踪影。而后听得雁南飞在叫自己名字,正欲回话时却突然发现只能发出呜哇之声。岳飞羽这一下被惊得非同小可,连忙爬回去找二叔帮忙。然而上山容易下山难,好不容易岳飞羽才回到了雁南飞所在的洞里。

雁南飞见他口青鼻肿,嘴不能言,猜想这多半是吃了那果子的结果。便与岳飞羽再次来到花栗鼠吃果子那洞里。那洞里还有若干剩余的果子,红彤彤的,鲜艳得要滴出水来。雁南飞虽不识此果,但他多了个心眼,撕破衣袖,包了几颗收好。

相生相克,雁南飞猜测附近可能会有能解此果毒的生物。便在这洞内外细细打量起来。只见那洞方丈见宽,在晨光中只见到丈许深,再住里便是黢黑一片,不知深浅。地面与墙壁均是光秃秃的岩石,不知那花栗鼠是从何处寻来的果子。正想着,忽觉一阵凉意袭来,原来是一股凉风自洞内吹了过来。雁南飞一喜,洞内必有乾坤!

雁南飞找了块石头用力往里一扔,却半天不见声响。正犹豫着,却见那花栗鼠不知从何处又冒了出来,嗖的一声跳到洞内漆黑处不见了。雁南飞不再犹豫,拿出夜明珠,与岳飞羽一同往洞内走去。岳飞羽虽是岳家少爷却也未见过夜明珠,见此甚是惊奇,跟着雁南飞深一步浅一步往里走。走不多时便到底了,变成了一个垂直向下的坑洞。

雁南飞交待了岳飞羽几句,便拿着夜明珠往下一跃,跳了下去。滑滑梯般滑了一会,雁南飞便落地了,眼前豁然开朗。可是还未等他感叹完毕,咚的一声,他便被从后面撞倒在地。原来岳飞羽见雁南飞跳了下去,没有片刻犹豫便跟着也跳了。二人爬起来,相视而笑。

原来此处竟是东山一未为人知的小山岰,其间花草丛生,果树密布,穿花蛱蝶深深见,点水蜻蜓款款飞。更为稀奇的是不远处竟有一人家,炊烟袅袅!

二人连忙走过去敲门。开门的是一老叟,鹤发童颜,精神抖擞,穿着一褐色长袍,整齐的道士髻甚是惹人注目。

“老丈,我叔侄二人落难至此,还请方便一二。”雁南飞连忙行礼。

“此处人烟罕至,你我能在此相见,亦是缘份。”老人眯眼微微一笑还礼。

“贫道河西散人久居于此未曾外出,不知居士何处人士?”

雁南飞只觉此名有些耳熟,只是一时回想不起。除却隐瞒了密道一事外,雁南飞将二人遭遇一五一十地告诉了河西散人,并取出那红色果子请河西散人鉴别。河西散人接过果子仔细端详,再看了岳飞羽脸上的症状,说:“居士无需过于担心。这是此处峭壁上的特产,名曰忘情果。寡情之人吃了自是无反应;然而多情之人吃了虽各有不同,轻则忘忧,重则忘情,却均无性命之忧。吾观小友面相端庄,天庭饱满,虽多有劫难却会逢凶化吉。行到水穷处,坐看云起时,雁居士只需静观其变即可。我这有颗清灵丸,倒是对他脸上的红肿略有效果。”

雁南飞接过药丸,却没立刻给岳飞羽服用,心里飞速回忆着河西散人这名字。他突然灵光一闪,难道是他?“道长可曾去过洛水白云观?”

“想不到居士还知道我白云观。没错,我就是那个师父要将白云观交到我手里时便逃了出来的那人。现在想来应该是师弟玉真人在管理罢。”

大汉王朝四大圣地,一观一寺二院,乃东方洛水城白云观,西方飞来峰灵隐寺,南方圣灵城麓山书院,以及北方武陵山尚武院。此四地,一道一僧,一文一武,乃大汉稳定之根基。

雁南飞不再有疑:“不想在此遇到白云观的河西散人,雁某当真三生有幸。”雁南飞本想问问他为何隐居于此,但此问未免涉及人家私隐,但忍而不问,转头便将药丸给岳飞羽喂了。岳飞羽不似洪小黑那般多疑,张嘴便咽了下去,只觉此药丸入口清凉,酸中带甜,苦中藏辛,似集百般滋味于一体。不一会,岳飞羽便觉喉咙搔痒,便哇的一声喊了出来:“哇------二叔,我能说话啦!”

河西散人笑而不语,雁南飞则是高兴地拍了拍岳飞羽肩膀。高兴之余,雁南飞又开始担忧起来,轻则忘忧,重则忘情。河西散人看穿了雁南飞所想,便对他说:“贫道对医药一道只是略懂皮毛,无能治疗小友此误吃忘情果之后症。但吾少年时曾遇一人,医术天下无双。你若寻得此人,必能解得了令侄之疾。但此人来去无常,无人知其踪迹。吾与之略有渊源,也只听闻其出身在雪莲坞。”

又是雪莲坞!

雁南飞双眼一亮,一事不烦二主,只要能找到雪坞神医,问题就都能迎刃而解了。

此时清风袭来,带来了淡淡的饭香。岳飞羽虽然能抗饿,但毕竟是半大小伙,正是能吃能喝的年纪,闻到这香味,肚子便不由自主地咕咕起来。河西散人看着满脸通红的岳飞羽,笑着说:“两位居士早上想来还未曾吃过吧,正好贫道今早的稀饭多煮了点,两位将就着吃点罢。”

“如此便叨扰了。”雁南飞一来亦是饿了,二来因曾听闻过河西散人的传言,也没推辞。

饭后雁南飞向河西散人问明了方向,便带着岳飞羽告辞而去。

两人走远后,只见那只花栗鼠不知从何处跳了出来,直接跳到了河西散人的肩头上。一人一鼠静立了一会,随后河西散人说:“你闻到了?”

“吱吱。”神奇了,那花栗鼠就像是听懂了似的竟然回应了河西散人。一人一鼠似乎是老伙伴!

“想去便去吧,莫要学我,一言不合就逃避。逃来逃去,追悔莫及!”河西散人说完便将那花栗鼠远远地抛了出去。

那花栗鼠轻盈地落地,转身盯着河西散人,半晌后转过身,快速飞奔而去。

却说雁南飞与岳飞羽二人沿着河西散人所指的方向,翻山越岭。岳飞羽走着走着,突然问题:“二叔,你是不是有什么事情要告诉我?”

雁南飞怔住了,羽儿猜到了?他看着岳飞羽,只见他满脸迷惑。雁南飞正想说出早已编好的故事,却见岳飞羽突然蹲地,双手猛地紧拽着头发。雁南飞心里一紧,要发作了,是华一针所说的痛心还是河西散人所说的忘忧忘情?雁南飞虽有智谋,医术却只是略懂些皮毛,只能紧盯着岳飞羽,防止他做出些出格之事伤害到他自己。

岳飞羽除了痛苦的模样,倒没做其他出格之事,他慢慢平静了下来,双手渐渐地松开了紧拽着的头发,抬起头四处看了看,然后眼光盯在了雁南飞脸上。

雁南飞见他双眼似要滴血般的通红,担心之极,连忙问道:“羽儿,你现在感觉如何,还有哪里不舒服不?”

岳飞羽怔怔地看着雁南飞那关切的眼睛,缓缓地说:“大叔,请问你是谁?我怎么会在这里?”

忘情果!

“呦呦鹿鸣,食野之苹;吾有仲子,忘忧忘情。”雁南飞喃喃地自言自语。


