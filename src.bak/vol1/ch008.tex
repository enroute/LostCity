\chapter{杨家有女初长成}
\label{chap:chen-jia-you-nv-chu-zhang-cheng}

岳飞羽对着大尾招了招手,大尾闭目养神假装没看到他。果然是女大不中留,想来这大尾是个母的吧。岳飞羽歪头看了洪小黑一会,只觉对此人有莫名其妙的亲切感,再加上实是无法还他那二十天馒头,便说:“只要它愿意,大尾便归你了。”

听得此言,洪小黑大喜过望,大跳起来,吓得那大尾在他肩上也跟着跳了一下,差点没掉下地。他紧紧地握住大尾的一只后爪,生怕它跑了。

“我们要赶路了。”洪小黑向他的七叔叔跑去,生怕岳飞羽反悔将大尾要回去。他拉着那青年乞丐沿着官道快步离开了。

岳飞羽再坐了一会,觉得无趣,便回了猎户家。雁南飞已经回来了,他铁青着脸,站在屋门前一言不发。岳府之事闹得沸沸扬扬,他没花什么功夫便打听清楚,料来那救了沐夫人的仙女便是双儿。只是奇怪双儿既入仙门,便不能再管凡间事,她师尊又怎会袖手旁观让她出手?原来此间世界有仙凡之分,二者素不相干,仙门自有仙门的约束,凡间亦有凡间的法则。因此凡间人能遇到仙门中人,那是少之又少。即便是岳家,知道双儿身份的人也是一手能数得过来。

雁南飞只是空有点打铁力气的一介文人,无处用力,在靖安四处暗访过后未找到岳府其余下人的下落后便只得先行回到猎户家再作打算。

岳飞羽见他一脸不善,心想糟了,二叔叫我不要乱跑,该不会是生我气了吧。他打小点子多敢做敢为,为此没少被岳东来收拾,但也是攒下了不少的挨训抗揍经验。

嬉皮笑脸是不行的,只会火上浇油。苦情戏也是不行的,一眼便知是在作弊。岳飞羽的窍门是,坦白从宽,外加一点点半真半假的借口。于是他上前坦白:“二叔,我未听从你的嘱咐,走了出门,请二叔责罚。”

雁南飞见他平安无事,本无意要责罚他。见他如此说来,所谓执法必严,不罚也不行了,便罚他在附近砍柴。可怜岳飞羽聪明反被聪明误,原本设计好的等二叔问起原由时再说出来的半真半假的借口也没机会说了。

岳飞羽拿着猎户那被磨得锃亮的斧头,对着一颗枯死的小树用力地砍着,不一会便累得浑身是汗,口干舌燥想要找水喝。他举起手以袖擦汗。用力之下,只觉那被洪小黑赏过一爆栗的额头被袖中一硬物硌得隐隐作痛,拿出一看,原来是前几日拜托雁南飞打造的发簪。他拿在手上,只觉贴着手心的地方隐隐有一层浅浅的蓝色光华,手感时温时凉,好不舒服。他呆呆地看了一会,全然想不起来这是他自己设计来要送给双儿姐姐的,只是觉得此物十分别致。把玩了一会,便放入怀中贴身收好了。岳飞羽忙活了半日,总算是完成了两担柴,放于猎户家门前放好。

翌日清晨,二人吃过早点,拜谢了猎户便借着晨光上路了。沐夫人既被双儿救走,想是再无大碍;至于岳东来,若是以他的本事也被困住,即便再加自己一人亦无济于事。故雁南飞的计划是先带岳飞羽去雪莲坞寻那飘渺的神医,先解了岳飞羽的隐疾。往日岳飞羽是三五日便要在睡梦中发作受罪一次,而二人从岳府出来至此也有四日了仍未见其发作,雁南飞更是担心这次是否会更甚。

二人翻过东山,来到东山北边的莫家堡。此处本名是黑水城,因其盛产黑金即石油而得名,其中有陈、莫、完颜与欧阳四大望族把持着墨水城所有的黑金矿。后大汉崛起时莫家出了一武将从龙有功,从此一人得道,鸡犬升天,莫家渐渐地成了黑水城的霸主,其余三大望族渐渐衰败,便成莫家的依附。终有一天,莫家竟上表朝廷将黑水城更名为莫家堡,从此莫家在此地是飞扬跋扈,甚嚣尘上,莫家堡宛然成了他莫家的私产。

雁南飞的计划是在莫家堡逗留两日,一来打听靖安岳府事件后续,二来做好出远门去雪莲坞的准备。守门军士见雁南飞二人的穷酸猎户打扮,想来亦是无钱之人,便大发善心的只勒索了十几铜板后便让二人进了城。走在莫家堡的街道上,只觉此处人来人往,热闹非凡,比起靖安是繁华的太多。雁南飞自是知道莫家堡的传闻,心中暗想,果然资源垄断好做大事。

二人在路边人少的地方的寻了个路边小摊,点了两碗牛肉面。此时日已高升晌午大错,早已过了饭点。正昏昏入睡的面摊老板见来了生意立马手脚麻利地忙活起来了。二人坐着边休息等待,边观察此处风景。岳飞羽虽自幼未出过靖安,但此刻见莫家堡商铺风景均与靖安无异,只是路更宽人更多商铺更华丽,瞧了一会便觉无趣,转而去看边上两只流浪狗为抢食而呜呜打架。两狗厮打了一会,突然便同时夹起尾巴一溜烟跑了,也不管有人没人,在人群中乱窜,只惊得小孩与少女阵阵呼唤。

俄而传来阵阵马蹄声,不远处几匹高头骏马疾驰而来,为首一匹毛发白胜雪,其上一青年男子白衣飘飘,英俊潇洒,仪表神态无不透露着高贵之气,好不风流倜傥。岳飞羽虽小且不甚注意仪态,但爱美之心人皆有之,此刻亦是心中暗赞一声,好一个俊俏的小哥。

青年男子一行人疾驰而过,不想前方突然跑出来了个少女!原来那少女被那两狗所吓便往大路中移了两步,没想一下没站稳便多走了几步,径直到了大路中央。眼见就要被撞上了,直引得路边众人惊呼连连,甚至有人捂住了眼睛不敢观看。在那千钧一发之际,只见那青年男子左手一勒缰绳,一队人马便齐刷刷地停了下来。好精湛的骑术!

马上一仆人打扮的人说:“哪里来的不长眼的丫头!快点离开!如果不是公子骑术了得,你现在已经是马蹄下的亡魂了!”

那少女死里逃生似乎被吓晕了,一下还没反应过来,对那人的辱骂是听而不见。好在旁边一老叟赶紧跑了过去,将那少女扶到了街道边上。那奴仆嫌二人走得慢,长长的马鞭在空中甩了好大一声响,怒喝道:“走快点,别挡了公子的路!”

“慢着。莫三,我时常教导而等在外不要张狂,为何总是屡教不改?”

“公子恕罪!小人下次不敢了。”明眼人都看得出来那莫三在假装忏悔。

“真是抱歉让这位姑娘受了惊吓,为了表达歉意,姑娘何不随莫某往百草堂走一趟,好替姑娘压压惊。”那为首的莫公子翩翩下马,潇洒地走近那少女,俯身凑近其脸庞,微笑着说。

那少女何曾如此近距离地接触过这般英俊潇洒的少年郞,一下子便脸红耳赤,心如鹿撞,连呼吸也急促起来。她害羞地低下了头一言不发。

“莫公子,小女打小粗养,这些小事根本不碍事的,反而是小女吓到公子的马,还望公子大人大量不要计较。”原来那老叟与少女是父女,那老叟见女儿的模样便将话接了过来。

莫公子闻言脸色一顿,用冰冷的眼神刮了那老叟一眼,随即又恢复了和绚的笑容。“既如此,莫某便告辞了。”莫公子对着那少女点头微微一笑,随后一个后空翻,利索地上了马。随着他响亮的“驾”的一声,一行人又疾驰而去了,只留下路人议论纷纷。

“杨老头,送上门的福气你都不要?”

“就是就是,那可是莫家二公子。他要是看上我女儿,那我在莫家堡就要横着走了!”

“切,就你那水缸般女儿,别说人莫公子,就是,就是......”那人犹豫了下,没往下接着说。

“好你个王二麻子!我女儿怎么了?好歹是个姑娘家,早晚可以寻一人家嫁了。你倒好,都半截入土了还是孤身一人!”

众人说着说着便歪了主题了,谁也没留意到当事人杨老头搀扶着女儿走了。

杨老头二人走到面摊前,只见一猎户打扮的男子对他拱手行礼,口中说道:“老丈请了,可否借一步说话?”

杨老头见雁南飞相貌端正,仪表端庄,不似坏人,便停下问道:“大兄弟,看你不像本地人,是初来莫家堡么?”

“老丈好眼光,我叔侄二人刚到莫家堡,对此地人生地不熟,想跟老丈请教一二。老丈请坐,喝碗茶水再走,如何?”

此时岳飞羽正在大口吃面,费劲地嚼着面中点点牛肉。那时的牛都是重要的劳力,一般都不会宰了吃,普通人能吃到的要不是老死的就是病死的牛,难嚼得很。反而羊肉是相对高级的食物,所以大口气说来两斤牛肉的大多是普通人,便宜量足。杨老头扭头看了眼岳飞羽,心生怜悯,便与女儿一同坐了。岳飞羽正吃得起劲,见二人坐下,便抬头瞧了一眼,稍微愣了一下便转头问雁南飞:“二叔,这位姐姐怎么看着如此熟悉?”

“想是你认错了吧。”雁南飞暗自叹了口气,他何尝不知,此少女虽说衣着朴素姿色平平,但却有一股与双儿甚像的气质,而且那一双纯洁的眼眸,更是与双儿的眼眸无二。正是这些特点让他二人觉得熟悉。

问起刚才那莫公子,杨老头朝四处望了望,见无人关注,便低头小声地说了起来。原来那莫公子是莫家堡莫家的嫡子,排行第二,大名莫无天,人称莫二公子。此人风流成性,时常游走于花丛中,在外的相好不说一百也有八十。然而其自恃风度,明面上从不强来。雁南飞微微点头,难怪那莫公子如此对待杨家女儿。若是花中老手,自是一眼便能看出杨家女儿,若是打扮一番,那便是倍胜于风尘女子了。

杨老头只此一女,名曰兮如,素来疼爱得紧,如何肯让她跟了莫无天。雁南飞本想着该是春花翠花之流,普通人家哪里能起得了如此之名,此刻听得杨家女儿的名字,暗自一怔,便仔细端详起杨老头。杨老头见他看来,便知其意,说:“我们一介村夫,怎么起得了这样的名字。实不相瞒,小女初生时,正好有一个道人路过到我家讨水喝。我们本来就不识字,正好便请那道人帮忙给小女起个名字,那道人见着了小女,说什么清扬婉兮,如璞如玉,便起了个兮如这个拗口的名字。”

杨老头见雁南飞见识宽广,谈吐间不经意地跳出些自己听不懂的言论,便知此人来历非凡,定不是寻常猎户,便要告辞。雁南飞知他心意,便对他说:“老丈请留步,听我一言。今晚万万不可回家,否则恐有血光之灾。”
