
\chapter{山雨欲来风满楼}
\label{chap:shan-yu-yu-lai-feng-man-lou}

日薄西山但仍未落时,岳府已华灯初上,上上下下忙碌一团,准备着迎接小姐回来。唯有来福在边干活边发愣,他出去一上午都没寻着小少爷,心里担心得紧。

岳飞羽虽一天未回,但沐夫人似胸有成竹丝毫不担心,只是安排府里上下准备迎接双儿,准备晚宴。沐夫人站在府中最高的主楼上,迎着稍有凉意的晚风,突然心里无来由的一阵悸动。她抚着心口,微蹙着眉,情不自禁的想到岳东来,心中暗暗祈祷丈夫平安归来。

正想着,一阵急促的脚步声将她惊醒,只见金伯急匆匆地跑了上来。“金伯,可是老爷回来了?”

“夫人,不好了,少爷晕倒了,被人送了回来。”金伯跑得有点上气不接下气。

闻言沐夫人脑袋嗡的一声只觉天旋地转,世界都在变得模糊。她强忍着不倒下去,只是在大口大口喘气。“人呢?”

“已送回少爷房里,我已命人去请靖安城西华一针。还有二爷那里,我也遣了人去送信。只是老爷那里如何打算,还需夫人定夺。”金伯是见过风浪的人,临危不乱,早已安排的妥妥当当。

沐夫人不语,脚步不稳地要走去岳飞羽房间。旁边侍女赶紧上来搀扶着她。毕竟常居高位的人,很快沐夫人便恢复了常态,健步如飞步进了岳飞羽的房间。沐夫人虽不习武,但耳濡目染之下多少懂点武学医术常识,她探过岳飞羽的手后见他脉相平稳,便知他大概无事,醒来也是早晚之事,心便稍安下来。

此时,城西华一针也被接到。华一针乃靖安医术最高之人,亦是最古怪之人。传闻他治病从来只要一针,一针就痊愈。若一针不好,便不要再找他了,找了他也不会再施第二针。华一针替岳飞羽早已看过多次,对岳府及府中人相当熟悉,他对着沐夫人施了一礼,二话不说便坐在岳飞羽床边,给他把起脉来。华一针自不是沐夫人所能比的,岳飞羽平稳的脉相甚至能迷惑到一流医师,但在华一针眼里却显得有些奇怪。华一针脸色阴晴不定。过了很久,他轻吐一口浊气,对沐夫人说:“夫人但请放心,公子性命无碍,无需动作他自会醒来。”

沐夫人也是会察颜观色之人,观他脸色便知他有后话,便对转头说道:“金伯,公子既然无事,你忙去吧,接到双儿立马禀报于我。若兰、若兮,你二人去厢房给华大夫准备好茶。”金伯与两侍女领命而去。

沐夫人对华一针说:“华大夫,有话请讲。”

“老朽外号华一针,夫人应该知道我的规矩。但我多次进府替公子把脉,从未施过一针。夫人可知缘由?”

“大夫说过小儿脉相稳中带奇,如石上垒卵,平衡至极致。若受任意外力冲击,平衡即破。”

“没错。我施一针,是外力。我开药调和,亦是外力。故老朽从未替公子施针开方。然则今日,吾观公子脉相,平衡已破,恐心脉首当其冲,今后公子怕是要与噬心之痛作伴了。”

沐夫人如五雷轰顶。回过神后把华一针当救命稻草,急忙问到:“还请华大夫指点一二。”

华一针摇摇头,沉默不语。

“靖安西北八百里处,有一隐秘地方,名曰雪莲坞,鲜有人知。传闻雪莲坞中有一神医,能活死人,肉白骨。若是求得此人出手,公子之疾或可根除。”华一针是真没办法了,见沐夫人这般模样,于心不忍,便连传说都搬出来了。

果然,沐夫人如黑暗中见到光明,大海中遇到了陆地,一下充满了希望。华一针行医多年,虽生死早已看淡,却不忍戳破她的幻想,便告辞而去。

沐夫人轻抚岳飞羽双颊,双目垂泪,暗怨上苍不公,让岳飞羽命运多舛,自幼便承受常人所不能之痛。怨归怨,沐夫人还是双手合十,默诵心经祈祷神灵降福。一通经文下来,见岳飞羽仍是沉睡,沐夫人便思索起雪莲坞神医一事。待大哥与双儿回来,明日便派人去寻一遭吧。

沐夫人起身走出房外,掩上房门,便见金伯快步而来。

“华大夫送走了?”

“是,夫人。迎接小姐的人也回来了。”

“双儿回来了?哪儿呢,如何不先来见我?”沐夫人心中一喜。

“夫人,小姐没回来。来齐与来兴二人在东郊乌沙河对岸的迎客亭处守了一天也没见着影子,便着来兴先回来报信了。”

“这丫头与她师傅一同回来,想来也不会出什么意外。许是又遇难民,治病救人去了,导致耽搁了时辰。老爷那边有消息吗?”

“来勤到县衙门前打听过了,众人都瞧着老爷与吴克敌进了衙门。来勤守候多时,没听着衙门内传出异样,也不见老爷出来。来勤亦到县狱买通了狱卒,打听到今日未曾有新犯入监。”

沐夫人自是对枕边人充满信心,无论文武,料来这小小的靖安衙门还没有人是岳东来的对手。最让她放心不下的还是岳飞羽。

少时,雁南飞亦到了。雁南飞对着沐夫人行了一礼,见她除却脸上略有忧色之外,神色甚是淡定,便料想岳飞羽并无大碍。“嫂嫂,羽儿可好?”

“无碍,华大夫说他不日自会醒来,无需特别护理。”此处人多耳杂,沐夫人只说了一半。

“二哥,今早大哥被官府邀去,至今未回。你可有收到风声,能猜测官府此举何为?”沐夫人一五一十地将早间岳东来被袁克敌请走之事说与雁南飞听了。雁南飞听后,便细细思索起来。

“大哥素来不愿与官家牵扯,但也从未交恶。岳家在靖安以来也是安分守己,未见飞扬跋扈之事。至于钱银一事,以大哥的脾性,必定没有瞒税不报之举。如今为官当权上位者,无非钱权色。岳府生钱有道,与权无沾。再者岳府这等家底,根本入不了大人物眼里。至于色,府上女眷$\cdots\cdots$”

“双儿?”沐夫人与雁南飞对视一眼,不约而同的蹦出同一个名字。

“双儿这丫头自幼标致,人见人怜。多年不见,想来必已长成国色天香沉鱼落雁之姿。窈窕淑女,君子好逑。难不成双儿被上位者相中了?”雁南飞轻声自语,不断推测。

“若是相中,必要交好于我家。不来我岳府下拜贴,却将大哥带入衙门不放,这是哪门子的交好之理?”沐夫人不理解。

“大哥与嫂嫂均是善良有礼之人,自是不会做出此等行径。然则上位者久居高位,凡事有人在周边对其唯唯诺诺,多的是不可一世之徒。如若双儿遇到一家势权倾大汉天下的纨绔子弟,他如何会将小小靖安岳家放在眼里。没有强行将双儿抢去,只是将大哥扣压在衙门逼迫就范,对其而言,已是给足了双儿面子。”

沐夫人将信将疑,双儿打小乖巧,如何会招惹到那等人物?

“无需双儿去招惹。酒香不怕巷子深,富在深山有远亲。想来是双儿被人招惹了。”

沐夫人闻言,越想越有可能,再念及双儿至今未归,不觉又为小女担忧起来。便将来齐二人一天都未等到双儿之事也告诉了雁南飞。

“双儿的师尊是世外高人,凡世间还无能谋算她敢谋算她之人。嫂嫂但请放心,双儿必能平安归来。至于大哥,以大哥的本事,想来也能遇事化吉。”

他们结拜兄妹里,雁南飞是智谋核心,历来算无遗策。虽知他是安慰的话语,但听后亦自心安下来。晚风习习,吹得二人长袖翩翩欲舞。沐夫人无心感受,只觉阵阵寒意,遂对雁南飞说:“二哥,你去瞧瞧羽儿这孩子吧。”雁南飞点头称诺。二人不一会便到了岳飞羽房间,推门而入,便见到了安安静静躺在床上的岳飞羽。走近了只见岳飞羽脸色安详,气息平稳,乍一眼看去并无大碍。

“适才嫂嫂似乎有难言之隐,可否告知南飞参谋参谋?”雁南飞是聪颖之人,先前早已猜到沐夫人有未尽之言,心中疑惑,直到此刻才提起。沐夫人遂对雁南飞说了华一针所言之雪莲坞一事。雁南飞向来视岳飞羽如己出,闻得此言神色一变,但瞬间又恢复如常。

“华大夫乃靖安医术泰斗,比起吾等的野路子,实是高明太多。既然华大夫如此诊断,我们不可不妨。至于雪莲坞,我少年游历时却也听闻一二。至于雪莲坞的神医,”雁南飞本想说从未听闻雪莲坞神医,瞥了一眼沐夫人后便接着说到:“向来神龙见首不见尾,要寻此人,府中下人恐怕都难当此任。等大哥回来,我便带羽儿走一遭吧。”

“此行若有二哥同往,必能成功。”沐夫人大喜,她本有此意,欲待岳东来回来后便让其请求雁南飞往雪莲坞走一趟,不想雁南飞自行先提了出来。二人再商议了若干出行细节后,夜色终于吞噬完了最后一缕阳光,天渐渐暗了下来,府中灯火显得格外通红明亮。

“嫂嫂,那我先行回去了。若双儿或大哥回来,你再遣人报信于我,我便过来。”商议完毕,雁南飞便要告辞回去早做准备。沐夫人颌首回礼:“有劳二哥了。”雁南飞正欲转身回他的草庐,忽闻外面吵杂大作,金伯神色匆匆赶了进来,便停下脚步。

“夫人,二爷。门外袁克敌带了百多军士,明火执仗,将岳府团团围了起来。”

沐夫人闻言,脑袋又是嗡的一声,摇摇欲坠。今日之事,恐怕早已击溃了寻常女子。然则沐夫人不寻常,她很快便稳住了身形,眉头紧蹙。

“恐怕大哥已然从岳府脱身,否则县衙不会如此声势浩大地派出军士来此。夫人不宜抛头露面,待我先去会一会那袁克敌。”雁南飞在一旁开口说罢,拱手一礼,便出去了。

岳府位于靖安镇僻静的东南一隅,附近只有寻常百姓三五家,少见喧闹。雁南飞出得府门,只见门外整整齐齐每隔丈许距离便站了一个手中拿着火炬的盔甲武士。那火炬整齐地围了岳府一圈,照得靖安东南角通红一片,早已惊动了在家躲寒的靖安百姓,纷纷出门观望交头接耳。那袁克敌还是骑着那头高大的西北黄毛战马,如鹰般冰冷的目光直视雁南飞,手中握着的战刀拔出了半截。

“来者何人,速速报上名来。上峰有令,任何人不得出入岳府。擅闯者,格杀无论!”

“袁统领,在下东山脚下雁老二。敢问袁统领来此是奉了何人之令?”

“擅问者,杀无赦。”袁克敌手中的刀又拔出了少许,刀峰闪烁着寒冷的火光。

雁南飞知多说无益,便退了回去,嘱咐来德把门关好便去找沐夫人。

正堂里,沐夫人坐在主位上以手抬额,雁南飞坐在下手沉思不语,金伯站在一旁纹丝不动。一波未平一波又起。过了良久,雁南飞叹了口气,说:“大哥消失与双儿未归,此二者必有联系。如今岳府被围,想来是出现了对方不可控的因素。如今之势,岳府已成鱼肉。依我之计,唯有大嫂你带着羽儿从密道走了,否则久围必生变,恐危及大嫂与羽儿安危。”

沐夫人闻言默然不语。

“大嫂,此乃生死存亡之危,当断则断,切勿犹豫。留得清山在,不怕没柴烧。我与金伯留在此处照应,一来迷惑袁克敌,二来若大哥双儿归来,亦好联络。”

沐夫人亦是女中豪杰,稍做思考便下了决心,于是当机立断回屋收拾了细软。雁南飞抱了岳飞羽,四人来到岳府祭祀先人的祠堂。沐夫人对金伯点头示意,金伯便趴在灵位台下扣动了机关,一顿吱吱的机关声后,墙边便出现了一人宽的一条密道,黑膝膝的。金伯早有准备,点了个火把。

“二哥,你把羽儿给我,你下去接应。”雁南飞将岳飞羽递给沐夫人,跳下了密道,转头正欲让沐夫人将岳飞羽递给他,却见沐夫人紧紧抱着岳飞羽,火光中隐约可见两道泪痕在脸上流淌。雁南飞见状,便闭上了刚张开的嘴,没有出声。

过了良久,沐夫人似乎是下了狠心,将岳飞羽递与了密道中的雁南飞。雁南飞接过并放下岳飞羽,转身正欲返回时,只见沐夫人将她的包裹丢了下来,密道门亦开始徐徐合拢,雁南飞一惊,立马往外跳,刚露得半个头在外时,却见金伯双手推来。雁南飞在空中无处借力,被金伯一推便跌回了密道中。待得他爬起来时,密道门已然合拢了,再不见一丝亮光,伸手见不着五指。

“二哥,我一女流之辈,如何能带着羽儿逃脱得出去。就算出去了也寻不着那雪莲坞神医。此行还是二哥去把握大些,就有劳二哥多担待些。我与金伯留在此处照应。二者大哥双儿未回,我怎可一人离去。”

雁南飞在密道中只听得沐夫人的声音越来越轻,想是渐行渐远了。雁南飞苦笑一声,自认已是算无遗策,想不到还是屡次中了沐夫人的算计。\marginpar{岳府的劫难先告一段落,后续通过他人之口交待一下。}