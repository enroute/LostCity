\documentclass[b5paper,12pt,twoside,openright]{memoir}

% relax footnote
\renewcommand{\footnote}[1]{}
\renewcommand{\footnotetext}[1]{}
% \renewcommand{\footnotemark}[1]{}

\renewcommand{\chaptermark}[1]{\markboth{#1}{}}
\makeoddfoot{plain}{}{}{-\space\thepage\space-} % page number on the bottom right
\makepagestyle{mystyle}
% \makeoddhead{mystyle}{\thechapter. \itshape\leftmark}{}{}
\makeoddhead{mystyle}{\itshape\thetitle}{\rule[-.3\baselineskip]{\linewidth}{.4pt}}{}
\makeevenhead{mystyle}{}{\rule[-.3\baselineskip]{\linewidth}{.4pt}}{\thechapter. \itshape\leftmark}
\makeoddfoot{mystyle}{}{}{-\space\thepage\space-}
\makeevenfoot{mystyle}{-\space\thepage\space-}{}{}


\pagestyle{mystyle}
\chapterstyle{dash}

% For Chinese fonts
% \usepackage{fontspec}
% \setmainfont{SimSun}
\usepackage[AutoFakeBold,SlantFont]{xeCJK}
% \usepackage[SlantFont]{xeCJK}   %AutoFakeBold=true makes CJK characters in the generated PDF can't be copied

\setCJKmainfont{SimSun}

\usepackage[english]{babel}     % For enquote
%% If babel package is used, then the following won't work
% \renewcommand{\contentsname}{目~~~~录}
%% One has to use the following to make it work:
\addto\captionsenglish{% Replace "english" with the language you use
  \renewcommand{\contentsname}%
    {目~~~~录}%
} 

% dots in chapter of ToC
\renewcommand*{\cftchapterdotsep}{\cftdotsep}

\usepackage{csquotes} 

\usepackage{bookmark}

\usepackage{metalogo}           %for \XeTeX, \XeLaTeX, \LuaTeX and \LuaLaTeX

\usepackage{colortbl}
\usepackage{tikz}
\usetikzlibrary{intersections,calc}
\usetikzlibrary{shapes}
\usetikzlibrary{plotmarks}

\usepackage{mathtools,amsthm,amsfonts,amssymb,bm}
\setCJKfamilyfont{kai}{KaiTi}
\newcommand{\kai}{\CJKfamily{kai}}

\makeatletter
% % Original \marginpar 
% \long\def\@ympar#1{%
%   \@savemarbox\@marbox{#1}%
%   \global\setbox\@currbox\copy\@marbox
%   \@xympar}

% Change the font to small
\long\def\@ympar#1{%
  \@savemarbox\@marbox{\small #1}%
  \global\setbox\@currbox\copy\@marbox
  \@xympar}

\makeatother

% Add border to marginpar
\let\oldmarginpar\marginpar
\renewcommand{\marginpar}[1]{%
  \oldmarginpar{%
    \tikz \node [anchor=north west,text width=.8\marginparwidth,fill=gray!30,rectangle,draw,rounded corners]{\tiny #1};%
  }%
}



\renewcommand{\baselinestretch}{1.1}
\renewcommand*{\arraystretch}{1.5}

\usepackage{graphicx}%\resizebox

\usepackage[makeroom]{cancel} % for xcancel/cancel/bcancel etc.

\usepackage{fancyvrb}           % for BVerbatim environment

\usepackage{verbatim}
\usepackage{varwidth}
% Center verbatim environment, with the help from verbatim package
\makeatletter
\def\verbatim@font{\normalfont\ttfamily\kai%\Large
  \hyphenchar\font\m@ne
  \@noligs}
\makeatother
\newcommand{\vfchar}[1]{%
  % the usual trick for using a "variable" active character
  \begingroup\lccode`~=`#1 \lowercase{\endgroup\def~##1~}{%
    % separate the footnote mark from the footnote text
    % so the footnote mark will occupy the same space as
    % any other character
    %\makebox[0.5em][l]{\footnotemark}% uncomment this if footnote required
    %\footnotetext{##1}%% This goes to the bottom of the minipage
  }%
  \catcode`#1=\active
}
\newenvironment{pcontent}[1]{%
  \par
  \centering
  \varwidth{\linewidth}%
  \verbatim\vfchar{#1}
}{%
  \endverbatim
  \endvarwidth
  \par\vskip10pt
}
\newcommand{\ptitle}[3]{【{#1}·{#2}·{#3}】}
\newcommand*{\psentence}[1]{\begingroup\normalfont\ttfamily\kai{#1}\endgroup}

\newcommand*{\splitline}{\begingroup\vskip.5cm\centering{\S~~\S~~\S~~\S~~\S~~\S}\vskip.5cm\endgroup}
\newcommand*{\cispace}{\hspace{1cm}}
% relax splitline
\let\splitline\relax

% indent the first paragraph after section header
\usepackage{indentfirst}
% parindent to 2 characters
\setlength{\parindent}{2em}
\setlength{\parskip}{3pt}


\title{失落之城}
\author{白羽}
\date{}

% \includeonly{preface}
% \includeonly{明月出天山}

\begin{document}
\frontmatter
% 4 parts of book/memoir/scrbook class:
% \frontmatter
% \appendix
% \backmatter
% \mainmatter

{
  \thispagestyle{empty}
  \definecolor{titlepagecolor}{cmyk}{0.436,.865,0,.478}
  \begin{tikzpicture}[remember picture,overlay,shorten >= -10pt]
    \coordinate (left) at (current page.west);
    \coordinate (right)at (current page.east);
    \coordinate (top)  at (current page.north);
    \coordinate (bottom) at (current page.south);
    % \coordinate (top-left) at (top -| left);
    \coordinate (top-left) at (current page.north west);
    \coordinate (top-right) at (right |- top);
    \coordinate (O) at ($.5*(left) + .5*(right)$);
    \coordinate (bottom-left) at (bottom -| left);
    \coordinate (bottom-right) at (bottom -| right);
    % \draw(left)--(right)--(top)--(bottom);
    \filldraw[draw=titlepagecolor,fill=titlepagecolor] (left)--(O)--(top-right)--(top-left)--cycle;
    \filldraw[draw=titlepagecolor!30!white,fill=titlepagecolor!30!white,opacity=0.5](top-left)--(top-right)--([xshift=-5cm]top-right)--([xshift=-5cm]O)--(left)--cycle;
    \filldraw[draw=titlepagecolor!30!white,fill=titlepagecolor!30!white,opacity=0.3]($.6*(left)+.4*(top-left)$)--(left)--(O)--($.6*(O)+.4*(top-right)$)--cycle;
    \filldraw[draw=titlepagecolor,fill=titlepagecolor](O)--([xshift=-5cm]O)--([xshift=-5cm]bottom-right)--(bottom-right)--cycle;
    \filldraw[draw=titlepagecolor!30!white,fill=titlepagecolor!30!white,opacity=0.3] (O)--(right)--(bottom-right)--cycle;

    \node[below right] (T) at ([xshift=3cm,yshift=-4cm]current page.north west) {\parbox{\textwidth}{\color{white}\fontsize{60}{72} \selectfont\kai \ctitle}};
    \filldraw[very thick, draw=white,opacity=50]([xshift=1.5cm,yshift=-.5cm]T.south west)--([xshift=1cm,yshift=-.5cm]T.south east);
    \node[below right,color=white,opacity=.5] at ([xshift=-1.5cm,, yshift=-1.2cm]T.south west) {\scalebox{1}[-1]{\textsc{\fontsize{30}{36}\selectfont\kai \etitle}}};
    \node[below left] at ([xshift=-2cm,yshift=-3cm]current page.east){\parbox{.3\textwidth}{\raggedleft
      {\kai\large 西南往南}\par
      {\small\texttt mario.li@foxmail.com}\par
      {\tiny powered by}\par
      {\small \XeLaTeX}
      }};
    % \node[above] at ([yshift=4cm]current page.south) {\large \today\par};
    \node[above] at ([yshift=4cm]current page.south) {\large \the\year--\the\month--\the\day\par};
  \end{tikzpicture}
}

% Reset page counter
\cleardoublepage
\setcounter{page}{1}
% \maketitle
\thispagestyle{empty}           % no page numbering
\cleardoublepage

\include{preface
\cleardoublepage}

\setcounter{page}{1}
\tableofcontents*

\mainmatter

\chapter{寒雨连江夜入吴}
\label{chap:han-yu-lian-jiang-ye-ru-wu}

\psentence{
寒雨连江夜入吴,平明送客楚山孤。\footnote{\ptitle{唐}{王昌龄}{芙蓉楼送辛渐}寒雨连江夜入吴,平明送客楚山孤。洛阳亲友如相问,一片冰心在玉壶。}
}

清明时节的靖安镇,细雨在斜风中淅淅沥沥地飘着。黄昏时分,日薄西山,华灯将上未上时,靖安镇街道上显得格外的冷清。除了几个行色匆匆的赶路人以及他们引起的犬吠声,这个位于江宁府的小镇似乎早早地就进入了冬眠。

一条官道的尽头,是靖安镇最繁华的同福客栈。虽是最繁华,此刻也只有稀稀拉拉的几桌客人在低语吃饭。

“四弟,此去一别,不知何日才能重逢了。”靠墙角的一桌,一个国字脸的雄壮中年人不胜唏嘘地感叹着。

坐在他对面的虬髯大汉看着他的同伴,感受着他不经意间散发出来的威严,胸中的热血再次沸腾起来:“大哥,天大地大,总有属于我们的地方。等我这趟把二哥找回来,咱哥几个再痛痛快快地喝一场!”

风萧萧兮易水寒\footnote{\ptitle{战国}{荆轲}{易水歌}风萧萧兮易水寒,壮士一去兮不复还。探虎穴兮入蛟宫,仰天呼气兮成白虹。}。雄壮中年人喃喃自语,然而他并没有说出来让他的兄弟扫兴。“四弟,此去一为别\footnote{\ptitle{唐}{李白}{送友人}青山横北郭,白水绕东城。此地一为别,孤蓬万里征。浮云游子意,落日故人情。挥手自兹去,萧萧班马鸣。},不管前路若何,谨记靖安还有几个老哥哥。我只嘱咐你一句话,不可以身犯险!”

“大哥你放心,俺好歹跟在二哥身边多年,总也见过猪跑不是?”

二人复长饮,直到客栈打烊方休。

\splitline

靖安镇东隅一所大宅子,是本地殷实人家岳府。二更时分,中年人步伐稳健地走进了进去。

“大哥,四哥走了?”岳府简朴的书房里,在两盏摇曳的烛光下显得格外的宁静。一中年妇人满脸疼惜地看着中年人,温柔地问道。

二人正是岳府的主人岳东来及其发妻沐夫人沐素衣。

岳东来轻轻地将沐素衣拥入怀中,轻声地责备她:“素素,跟你说过多少回,不要熬夜等我。你说你有哪回是听我的。”沐夫人安静地享受着丈夫宽阔温暖的胸膛,浅笑不语。

良久无语,夫妻二人仿佛融入了天地,都在依恋着这片刻的安宁。

“啊------”

突然间不远处传来一声低呼。两人随即惊醒,对视一眼,心有灵犀地说到:“羽儿!”随即同时奔向书房右手侧不远处的厢房。只见厢房一床上躺着个七八岁的小男孩,双眉紧闭,口中发出啊啊的叫声。他的叫喊声是如此小,似乎是在努力强迫自己不喊出来。

“这些年来真是苦了这孩子。”沐夫人轻轻地擦拭掉小男孩额头上的细细汗水,又轻轻地拍着他的背。小男孩似乎感受到了母亲的关怀,不一会又沉沉地睡去了。

岳东来沉默地站在床边,紧握双拳,紧咬钢牙。

“这些年来亏得有二哥的细心照料,否则这苦命的孩子......”说到这,沐夫人用长袖抹了一下眼角。然后好像抓住了救命稻草般,眼中也露出了一丝希望的光芒,急声问到:“大哥,你说四哥这次能成功么?”

“素素,我和二弟四弟都已经计划妥当,你就安心等着四弟的好消息便是。”岳东来安慰着妻子。

夜色在夫妻二人的相互低声慰藉中慢慢褪去。随着宅中雄鸡响亮的一声,岳府忙碌的一天就渐渐开始了。

\splitline

“来福,你一大早的慌慌张张到处乱跑什么呢?”岳东来几乎一夜未合眼,但仍看不出丝毫的疲态。他逮着年轻的来福问到。

岳家虽说家境殷实,但下人数量也可一手数得过来。这个来福是个高壮憨厚青年。听到岳东来的问话,他急忙停下来回到:“老爷,飞羽少爷又不见人影了,我在宅子里都找遍了还是没找到他。”

岳东来略一沉吟,便道:“我知道了,你可以暂时不用管羽儿了,先去忙其他的事吧。”

“不行啊,老爷!万一飞羽少爷跑出去又被人打破了头那可怎生了得。对了,我要出去找一找。”说完,来福也不管岳东来这个主人,竟径自一溜小跑出了岳府,转眼便无影无踪了。

岳东来摇摇头,转身对旁边的管家金伯说:“金伯,双儿回来一事可都已经准备妥当了?”

“老爷,前日听说消息后,老奴昨日便领着大伙收拾妥当了。”

“老爷,老爷,门口来了好多官差------”一个下人突然慌慌张张地跑进来。

“来德,不用着急,你且说说来了多少人,领头的是谁,骑马没骑马,都说了些什么话......”

来德一个也不知道,他说:“老爷,我一看到那么多官差就赶紧跑回来告诉您了,其它的我都没有留意啊。”

“无妨。”说完,岳东来便领着金伯走向岳府大门。

\splitline

“原来是袁统领,失迎失迎。袁统领来到岳某这里,真是令岳某的寒舍蓬荜增辉呀。不知统领是否有空进寒舍喝口茶水?”岳东亚看着府门前一字排开的十八骑高头大马上武装整齐的士兵,不卑不亢地对领头的袁克敌打着招呼。靖安是个小镇,最大的武官也不过是百长,岳东来称袁克敌为统领是有意抬高他了。

“你我素无交情,茶水便不必了。袁某是公干来的,奉县官老爷的令,请岳先生到县衙门走一趟。”袁克敌冷冰冰地说。

“既是县衙大人有请,小民自当前往,何必惊动袁统领亲自走一趟。不知县衙大人有何吩咐?”

“袁某是军人,只是奉命行事,不敢过问大人们的事。岳先生,请吧。”

岳东来对金伯使了个眼色,暗示他不要冲动,对他说到:“金伯,我去去就回,你回去告诉夫人今天的晚饭可要准备得热热闹闹的。”说完,转头对袁克敌打了个手势,“袁统领,请。”

袁克敌脸上一丝诧异转瞬即逝,紧握着战刀的右手也暗暗松开,随即带着十八骑及岳东来往县衙门方向缓缓而行。

\splitline

岳府正堂里,金伯把岳东来临走时所说的话报于沐夫人。

“既然老爷这么说,想必他已有对策。再说,谅那几个燕骑也奈何不了老爷。金伯,此事不用声张,只管安排既定事项即可。今日双儿回来,万事不可出纰漏。否则......”说到此处,沐夫人又倍觉头疼。转念想到醒来时盖着的被子,心中不免又多了一丝温暖。

金伯脸部猛然抽搐了下,干笑一声,说到:“夫人放心,老奴再去检查一番,势必不给那混世魔王\marginpar{他一个下人怎么敢叫少爷混世魔王?}留下丝毫借口。”说完,金伯一脸冷汗匆匆而去。

\splitline

却说岳家公子岳飞羽清晨醒来后,俨然忘了昨夜的噩梦。

“娘怎么又趴在我床边睡过去了?”\marginpar{岳东来呢??怎么能任由他夫人趴着睡着了不管!}岳飞羽喃喃自语。他人小力气小,抱不动娘亲,就扯过床上的被褥给母亲盖上,小心地穿上衣服,蹑手蹑脚地走出去,再轻轻地带上房门,然后一下跑没影了。

飞羽轻车熟路地躲开所有的下人,来到后院一个墙角,东张西望一会确认附近无人时,鬼鬼祟祟地从杂草中掏出一根草绳,抛到墙头突出的石柱上,顺着绳子就爬了上去。

“小哥我自由了!”爬上墙头的飞羽撇嘴微笑,纵身一跃就跳到墙外了。

“咚!”

摔了个四脚朝天的岳飞羽摸摸头上鼓起的大包,警惕地四周望了望,松了口气:“幸亏没人。”

“小王八,爬呀爬,爬上墙头要跳崖。墙又高,风又大,摔到地上也不怕。幸亏有个大龟甲,否则摔疼找妈妈。”

听着这准时出现的嘹亮的童音,岳大公子脸一黑,爬起来就跑。没想却与唱童谣的小孩儿撞了个满怀,呯的一声再次四脚朝天。岳大公子皮糙肉厚,马上又爬了起来,定睛一看,脸再次黑了起来。“真是晦气。洪小黑你是故意站那里等我撞的吗?”

那叫洪小黑的小孩儿,一身乞丐打扮,又黑又矮又瘦小,衣衫虽褴褛但倒也算干净。洪小黑哎哟哎哟地缓缓爬将起来,白了岳大公子一眼,说:“你还恶人先告状了啊?我告诉你,别看你皮糙肉厚带个王八甲经打,再惹我,我,老......老子一招分花拂柳手保证让你三天下不了床!”洪小黑说完站好,似乎意犹未尽,于是又补了一个重重的“哼”字。

岳大公子回想起上次地狱般的惨状,立马冷汗涔涔。然而他嘴上却不肯服输:“我怎么没听说过分花拂柳手是你们丐帮的武功呀?”

洪小黑一时语塞,愣在原地,黝黑的脸上现出了些许红色。

岳大公子大仇得报,得意扬扬。“哼,洪小骗子,收起你那张嘴就来的谎话,哥可不是被你吓大的。”岳大公子雄赳赳气昴昴地走向洪小黑。洪小黑气极,正欲赏他一记分花拂柳手,却见岳大公子脸色大变,拉着他就跑进拐角的胡同里。

“岳------唔---唔---”洪小黑还没来得及说出第二个字,嘴巴就被岳大公子的手捂住了。“嘘------别出声。来福出来找我了。“岳大公子轻声对洪小黑说。


\splitline

二人偷偷摸摸跑得气喘吁吁地躲过来福,来到靖安镇西南空地杂草丛中,岳大公子松开紧开一直紧抓着的洪小黑的手,双手撑着膝盖弯腰大口喘气。洪小黑看着被岳大公子抓得黑里透红的手腕,怒目圆睁地盯着岳飞羽,只等还不自知的岳大公子点起一丁点火星就要爆发出来。可惜过了良久,岳大公子只顾喘气,没做任何动作发出丝毫言语来点爆洪小黑,洪小黑的怒意火山也就慢慢冷却下来了。

“洪小黑,你个天生的扫把星,我每次遇到你准没好事。”

完蛋了,洪小黑冷却的怒意找到了出口立马爆发了。

“分花拂柳手!”

洪小黑哪里还管这个招工是否丐帮所传,只管用平生最得意威力最大的招式招呼过去。瞬时只见杂草与尘土飞扬,呯呯与哎哟同现,吓得杂草丛中觅食的雀儿与不知名的小鸟齐飞。过了约摸盏把茶的功夫,洪小黑自己先累了,便停下手来,冷的如刀般的目光却依然在围剿着岳大公子。可惜洪小黑的杀人目光没有效果,可怜的岳大公子正捂着肿得像猪头般的脸嗷嗷惨叫,根本腾不出空暇来感受洪小黑的这一最后杀招。

好在岳大公子皮实,叫了一会便觉得不痛了,便想着如何找回场子,然而寻思过后又自知打不过不知从哪里学了两手歪招的洪小黑,便索性继续躺在地上慢慢想法子。洪小黑见他突然一动不动,紧闭的眼珠似乎在不停的转动,心中吓了一跳,寻思不会是错手把这厮给打死了吧。洪小黑用脚拔了下岳大公子的腰间,大声喊到:“快起来,别给洪大爷装死。”岳大公子正想着法子呢,一下被洪小黑的惊天吼打断了思路,便气呼呼地要与他作对,你叫我起来我偏不起来。洪小黑轻踢了几下后,见岳大公子还是没有动静,甚至连眼珠子也不再转动,一下便失了举措,慌忙俯下身去查探岳大公子的鼻息,只觉岳大公子的鼻子只有细微往外出的气,便觉天崩地裂,豆大的泪水瞬间涌出眼眶流下脸颊滴到岳大公子的脸上。洪小黑抓着岳大公子的衣领,泪眼婆娑断断续续地说着:“平日里打你三五招都没事,今日却为何如此虚弱?岳小八,我------”

岳大公子本就觉得脸上洪小黑的眼泪奇痒无比,再听得洪小黑叫他这个名字,便再也忍不住,一下跳将起来,怒喝到:“你说谁是岳小八?!”

洪小黑被他吓了一大跳一屁股跌坐在地上,黑漆漆的脸都变绿了:“你,诈......诈尸了?”

岳大公子跳将起来本就为吓他一吓,跳将起来后便觉此仇已报此恨已销,心情舒畅无比,本想顺着洪小黑的言语装做个诈尸的怪物断续吓他一吓,以便扳回一城扭转下与洪小黑对垒之间一直处于劣势的状态,但此刻见洪小黑似乎真被吓着了,便觉不忍,于是恢复常态,轻声对洪小黑说:“洪小黑,我是吓你的呢,我这么结实防御无敌的人,怎么可能是你的花拳绣腿挠痒手所能伤得了的呢?”

洪小黑将信将疑,但一看他那嬉皮笑脸样,再想到传说中鬼怪都是阴森可怕的,便知遭他骗了一回。

“分花拂柳手!隔山打老牛!好你个岳小八,嫌皮痒还没打够是吗?”

于是杂草丛中再次上演了鸡飞狗跳。等到洪小黑再次停下来的时候,岳大公子已经奄奄一息地躺在草地上了。

“啪嗒”一声,突然又有一滴水落到岳大公子的脸上。岳飞羽闭着眼睛想,这小黑子不会又被我骗了吗?他仔细感觉了一下气氛,没有听到洪小黑的任何异动,便知道不是。“啪嗒”又一滴。岳大公子再顾不得许多,睁眼一看,原来天不知道何时已然暗了下来,隐隐有暴雨来临之势。

“洪小黑快跑啊,再不走就要成洪汤鸡了!”岳大公子边说边跑,全然忘了自己还在装死当中。洪小黑早已看到天色变暗,只是被岳飞羽气到了,没去考虑淋雨一事。如今听他一噪子,便跳起来跟着岳大公子一起跑了。跑到半路,突然回味起岳飞羽最后一句话,怒声喝到:“岳小八,你刚才说谁是洪汤鸡?!”

\splitline

“真是天有不测风云。双儿的师傅都算过今日是黄道吉日,没想成今日还是要下起雨来了。金伯,令来齐早点出发多走点路去接双儿吧。”沐夫人看着满楼风雨,露出一丝担忧的神色。

“是,夫人。”

“羽儿还没回来么?”沐夫人对还未走远的金伯问到。

“是的,夫人。来福已经去出去找少爷了,老爷说无妨,我料想应该也是无妨的。”

“也是,在靖安这个小镇,想来应该也是无妨的,就由他胡闹去吧。但金伯,他若是闹出些伤天害理人神共愤之事,你可不要帮着他来瞒着大哥和我,你也瞒不住。”

“夫人,少爷本性淳厚,被人伤被人害倒有可能,小打小闹也有可能,真要是伤天害理之事则是断无可能的。”

\splitline

“阿嚏------”正在飞奔中的岳大公子打了个大喷嚏停了下来,心里正盘算着是谁在暗地里说诅咒他,突然被后面追赶而至的洪小黑撞了一下,一个踉跄下盘没稳住,趴在地上摆了个大大的大字。原来洪小黑一心要追上他问个明白,没想到他会突然停下来,惯势使然再加上心里就想着把他撞倒出口恶气,便全力撞了岳大公子一下。有心算无心,再加上岳大公子力气本就没洪小黑大,不倒地上才怪了。

看着岳飞羽在已见泥泞的路上摔得像个泥人般,洪小黑不由欢声大笑起来,终于是出了一口恶气,不枉他追了岳飞羽一路。

“喂,起来了,别再装死了。”

过了半晌,雨渐渐大了起来,岳飞羽与洪小黑的衣服都已经被淋透了。洪小黑见岳飞羽在他的连吼之下仍不见动静,忙蹲下去。只觉岳大公子的四肢已渐冰凉,“还好仍有呼吸。”洪小黑似乎见过鬼之后就不再害怕了,这次倒镇定了许多。“该不会真被我一下给撞傻了吧。唉,念在你好歹请我吃了那么多馒头的份上,洪大爷今日便也发发慈悲,拖你回家吧。”

于是路上多了一个瘦弱的身板在雨中拖着一个重物在雨中一步一挪地行走的风景。

“岳小八,你是属猪的吧,怎么这么重。”

突然天空一白,紧接着轰隆一声巨响,吓得洪小黑把拖在手中的岳飞羽的手都丢了。原来天空中一道极长的闪电直接划落到远处的地平线上,把阴暗的天空一劈为二,很有盘古开天辟地的气概,也很吓人,比如吓到洪小黑。

“妈呀,报应不会来得这么快吧,他都没死呢就来劈我了?”洪小黑心里发憷,便寻思着要找个地方躲避一下,好不让雷劈到,先渡过这一劫再说。洪小黑四处看去,见到不远处似乎有个房子的影子,心中一喜,便想起那该是靖安西郊早已破落的枪王庙的方位。洪小黑再次看向仍然躺在地上的岳飞羽,心中暗到:“岳小八,虽说有冤报冤,但也要有恩报恩,今日我带你去枪王庙好让你躲过这一场雷劫,日后不论你是人是鬼,都不要来找我报仇。”此念一起,便拖着岳大公子往枪王庙而去。



\chapter{大雨枪王庙}
\label{chap:da-yu-qiang-wang-miao}


路程虽短,奈何洪小黑人小力气小,待到进了枪王庙后,二人早已成了落汤鸡。洪小黑有点虚脱地倒在地上大口喘气,二人身下慢慢渗满了衣物上溢出来的水。庙外乌云遮天,天色已然如入夜般阴暗。伴随大雨而来的东风从早已破败的窗户带进来阵阵凉意。

“不行,这厮已然昏倒,若任由这般下去,早晚会被风寒夺了性命,想来阎王还是会将罪过怪到我头上来。”念及此处,洪小黑不知从哪来借来了神力,马上爬了起来,在枪王庙里寻找起来。

此枪王庙本是当地民众为祭奠大宋神将杨将军而自发修建的小庙,随着时光流逝,朝代更迭,物是人非,枪王庙早已破落得成了残垣断壁,不再有香火。

洪小黑幸运地在残破的枪王像下找到了些许干草,想来是哪个过路的同行乞丐用来休息之物。“前辈,借你干草一用。枪王爷爷,休怪莫怪,借你金身一用。所谓救人一命,胜造七级浮屠,想来你也会支持我的。”洪小黑从枪王像中取了块稍硬的残片,击石取火,将口中那前辈留下的干草点着了。二人便围在这微小若明若灭的火光中取暖,只不过一个是蹲坐在地,一个是大字躺地。过不多久,前辈的床铺都被洪小黑全借用完了。

就在火堆在草料烧完即将熄灭时,只见岳飞羽开始抽搐起来,有一下没一下的。洪小黑正盯着岳飞羽呢,毕竟是小孩心性,被他这举动吓得跳将起来:“这厮又要诈尸了吗?”洪小黑哆哆嗦嗦地慌忙连走带爬地远离岳飞羽,在一墙角处抱膝蹲着,慌张地看着岳飞羽。

原来,岳飞羽本就有顽疾,受了风寒之后又被火气诱发,便发作起来了。

眼见岳飞羽抽搐得越来越厉害,口中发出野兽般的低吼,洪小黑心中的惊恐马上积聚到了极点,在即将爆发之际,突然屋顶硕大一颗水珠滴在洪小黑额头上。洪小黑便如醍醐灌顶般清醒了过来。“七叔叔说过,世间本无鬼,不过是凡人的业障罢了,但万法相通,鬼怪论的妙方用到凡人的业障有时倒也有奇效。如今看这厮一惊一咋,又沉睡不醒,似乎处在二魂与三魂游走不归之间。看在馒头包子的份上,且看洪老爷大发慈悲助你岳小八渡过此劫罢,”洪小黑回想起七叔叔说过的招魂术,有模有样地舞起来。洪小黑滑稽地上蹿下跳,左摇右摆,跳了一会,便自觉好笑,心中笑骂:“便宜你岳小八了,本大爷还从来未曾跳得如此可笑过。阎王老爷,小子对岳小八已是仁至义尽,这本是他自身劫难,若渡不过去,你老人家要明辨是非,莫要怪到我头上来。”跳到最后,洪小黑高高跃起,落下时一指头点在岳飞羽额头上,大声喝到:“岳小八,你回来罢!”

洪小黑定格在最后一击的姿势,任由不时漏进来的风吹得衣衫猎猎作响。

“失败了。”洪小黑看着免疫招魂的岳飞羽,暗想,俄而跌坐在地,旋即不甘心地一跃而起,再次使出一套独创招魂术,在岳飞羽身上四处敲打起来。

“小子,你再这样敲打下去,乱了他经脉,就真的魂飞魄散回天无力一命呜呼了。”

听着这突如其来的苍老却又铿锵有力的声音,洪小黑再次被吓了一大跳,立马停下动作,扭头看去。只见枪王庙门处不知何时站了个仙风道骨的道人。经过洪小黑此番折腾,火堆早已尽灭,只剩冒着点点火星的灰烬。隐约间只见那道人披着一身玄色道袍,梳得整整齐齐的长发在头顶盘成了个标准的道髻,背着个偌大的葫芦最为惹人耳目,空无一物的双手自然地垂放着,几可够到膝盖。显然声音出自老道之口。看着那道人,洪小黑无由头的心中大定,但看他口中略略含笑,似乎在嘲笑自己方才的滑稽行为,便心生恼怒,坐在一旁扭头不理他。

那道人见洪小黑不理自己,也不觉生气,微微一笑,径自走到岳飞羽身前,蹲下给岳飞羽把脉。在即将抓到岳飞羽的左手腕时,道人似乎心有所感,道袍猛然一鼓,在空中激起阵阵涟漪,神奇的是,地上杂草及火堆的灰烬却无动于衷,仍然安安静静地躺在地上逍遥自在。“小友放心,贫道清风子,断无要害你之理。”那道人对着岳飞羽说。他声音不大,但远远地送将出去,只惊得枪王庙附近密林中躲雨的大鸟四处飞散。

洪小黑本在恼怒或许被那道人看到了自己的丑态,感受到微微激荡的气流,再听他如此一说,不由转过头来,诧异地看着那自称清风子的道人。只见清风子握住岳飞羽的左腕,闭目沉思。过了良久,清风子睁眼,叹了一口气,喃喃自语道:“枉我清风子自命熟读黄帝内经阅尽天下奇难杂症,不曾想今日之症却是闻所未闻,远非老朽之学所能及。今日方知山外有山。”清风子自语半天后,摸出一颗猩红的药丸,拇指般大小。“小友,老朽虽未能将你根治,但缓解你此刻的痛苦却还是力所能及。”清风子将药丸递给洪小黑:“小子,你若想他醒过来,便将这药丸给他服用了吧。”

洪小黑接过药丸,闻着其散发出来的沁人心脾的阵阵清香,强行压住想要生吞活咽下去的冲动,将信将疑地说:“听说最艳丽的蘑菇通常都是有毒的,最斑斓的蛇通常也是巨毒的,据说海外还有一种色彩绚丽的青蛙,更是毒王中的毒王。老先生你这个药丸这么好看这么香,该不会......”洪小黑心中还有一句话没说出来:这老头该不会是想借我的手毒死岳小八吧?

清风子一听,气得脸都绿了,可自持身份又不好对小辈发作,刚举起的指着洪小黑的手顺势就把药丸夺了过来,一把便灌进了岳飞羽的嘴里,颤抖着声音大声喝到:“我便毒死了他给你看!”。岳大公子却也了得,昏迷中刚要叫喊一声偌大个药丸就顺势滑落进去,然后卡在喉咙里呜呜地出不了声,本来苍白的脸也慢慢涨红了。清风子便在他胸口处推拿了一番,只见岳大公子喉咙滚动了一下,洪小黑见状心中一喜,进去了,至少不会被噎死。

过了一柱香功夫,岳飞羽渐渐平静下来不再抽搐。

“好你个岳小八,睡得倒舒坦。”事情告一段落,洪小黑才想起自己早上到现在都还滴水未沾,肚子开始敲锣打鼓起来。清风子也是个妙人,本来还在跟洪小黑这小孩儿斗气,听他肚子一叫,便乐了,动作夸张地从怀里摸出一个大煎饼,放在鼻子下用力的闻了一下,然后得意洋洋地咬了一小口,在嘴巴里嘎嘣嘎嘣地大声咀嚼起来,脸上神色要多惬意有多惬意。可洪小黑不乐意了,心中暗暗诅咒他:“个糟老头子,那么大个煎饼那么香,小口小口吃也不怕撑死你。”洪小黑恨不得亲自上阵帮他把煎饼一口就吞到肚子里。洪小黑信奉的是到了肚子里的才是自己,有得吃就赶快吃抓紧吃吃撑了还要吃。

清风子乜斜着眼睛看着洪小黑口水都快要滴到地上了,心里别提有多痛快了:小娃子,想跟老夫斗,你还嫩得很,老夫有的是招式对付你。

“老先生,你的煎饼上有只蜘蛛。”

怎么可能,世上还有能逃过我法眼静悄悄来我面前的存在吗?清风子冷哼一声,小娃子想嫌我来着。突然清风子想起一事,暗叫不好,难不成是南苗神凰来了?如果是此人的毒虫,我还真无法察觉。清风子天不怕地不怕,平生只惧三人,南苗神凰便是其中之一。念及此处,清风子急忙睁眼一看,煎饼上果真隐约有个小黑影,他心里想着南苗神凰便以为这黑影是南苗神凰的毒物,便怪叫一声马上把手上的煎饼远远地丢了出去,飞到庙外不见了踪影。洪小黑暗道可惜了好香一张煎饼。

“老先生,那蜘蛛往你怀里爬进去了!”

这还得了,清风子赶紧伸手往怀里掏,掏到什么扔什么。洪小黑有了上次经验,稳稳当当地接住了清风子从怀里扔出来的几个煎饼,就大口地吃了起来。

清风子把怀里的东西扔完了才发现洪小黑在大口大口地吃着煎饼,马上明白了事情的缘由,暗道着了这小娃子的道了,不由老脸一红,自言自语地说:”唉,现在的小娃娃,连老头子都要骗,真是人心不古哇。”虽说是清风子的自言自语,然而每个字却清晰无比的传入到洪小黑的耳朵里。洪小黑自是免疫此等言语,否则凭他一个小乞丐如何能活到现在。清风子见洪小黑毫无惭愧羞耻之心,乃心生一计欲以其人之道还治其人之身,于是装做突然想起的模样,大叫道:“哎呀,老夫的醉魂散怎么撒了,糟糕,小娃子,快停下,那煎饼上沾了老夫的醉魂散!”

洪小黑饿极正吃得带劲,哪里管它什么醉魂散,只管吃饱肚子,管它洪水滔天。清风子虽童心未泯但本不是多心之人,见此无效也就黔驴技穷了,暗道好个土包子,连老夫的醉魂散也不晓得厉害。

呆呆地看着洪小黑狼吞虎咽地吃着自己的烧饼,清风子沉默良久之后,轻叹了一口气,默默地说到:“你还是追来了。”

洪小黑正吃得过瘾,哪里注意到他这句本就小声的话,同样也没注意到庙门口突然多了个中年男子。

“是的,少爷,心月来了。”自称心月的中年秀气男子笔直地站着,恭敬地对清风子说道。

“心月,你跟了我这么多年,到底是知我呢,还是不知我?”

“少爷,心月既知亦不知。”

“是啊,你要是不知,又如何寻得到我。你要是知,又如何会来寻我!”

“夫人有命,心月不敢不从。”

清风子默然。突然身影一动,凭空消失于枪王庙,只剩下远远传来的声音:“吾去矣,勿寻也。”

心月似已早知如此,静立片刻后朝清风子消失的方向拜了拜,随即亦缓缓离去。

洪小黑听到他二人后面的言语,早已从两耳不闻窗外事的吃饼状态中恢复过来。听着二人前不着村后不着店的言语,一头雾水。待到清风子突然消失,更是惊为天人,但鉴于今日受惊配额早已被岳飞羽用完,才没被吓得叫吓得跳。虽是如此,洪小黑仍是被清风子凭空消失的一幂惊得口瞪目呆。

清风心月二人离去良久,岳飞羽轻轻咳了一声,悠悠醒来。洪小黑这才回过神来,回头一看,只见岳飞羽已缓缓睁开双眼,原本苍白的脸色亦已恢复红润。洪小黑不禁心头一松,心想这厮的鬼魂好歹是不会来寻我复仇了。

岳飞羽虽已睁开眼睛,但仍觉脑袋昏昏沉沉,正在慢慢回想之前发生之事。当他发现洪小黑犹自在嚼着嘴巴时,肚子便条件反射般地咕噜咕噜起来。此时雨已渐歇,枪王庙里异常地安静。洪小黑听得真切,便窃窃暗笑,想不到这纨绔子弟岳小八也会有饿肚子的一天,于是嘴中吧嗒得更带劲了,心想这清风老头子的烧饼还真是香。

岳飞羽渐渐回复力气,思路也慢慢清晰起来,闻着空气中烧饼的余香,想起早起至今仍未进食,瞬间便觉得饥饿得紧。但岳飞羽虽是岳家少爷,却不是纨绔子弟,自幼便能吃苦。他忍着饥饿,正想跟洪小黑问清楚情况,忽然灵光一闪,瞧着庙门外依旧有些昏暗的天色,大喊糟糕,“洪小黑,现在是什么时辰?天都要黑了,是已经到晚上了么?!”

洪小黑本想再捉弄他一回,但见岳飞羽火急火燎的样子,但觉不忍,遂对他说午时未到,只是大雨刚歇,天未放晴。岳飞羽这才放下提起的心,长吁一口气,爬起来往庙门外便走。结果刚走两步便一个踉跄扑倒在洪小黑身上,两人扑通一声齐齐倒地。洪小黑便哇哇大叫起来,好你个岳小八!虚成这个样子还不忘整你洪小爷,看我今日如何收拾你!

岳飞羽回想起洪小黑的各种虐招便觉头大,于是使出洪荒之力爬起来往外便跑。于是二人再次上演老鹰抓小鸡的追逐游戏。追着追着,洪小黑便觉奇怪,这厮明明刚晕醒,外加饿得慌,为何我还追不上他呢?


\chapter{结庐东山下,悠然采南花}
\label{chap:cai-ju-dong-nan-xia}

不知不觉间二人已回到了靖安镇,跑到了一偏僻巷子一小草庐前。此时清明的微风早已将发过威风的雨云吹散,天空恢复了午时的明亮。路边铺的石头水渍也在渐渐消隐。

岳飞羽推门而入,嘴里大喊二叔。一中年男子闻声而出,从里屋走了出来,见了双手撑在膝盖上大口喘气的岳飞羽,便眼睛一亮,招呼岳飞羽:“小羽来了,快到屋里来。”

岳飞羽正欲跨步往里走,洪小黑也到了。原来洪小黑见他进了草庐,心想正好来个瓮中捉鳖,便跟了进来。洪小黑正欲往岳飞羽扑去,见着边上的中年男子,便硬生生停了下来,定睛打量起中年男子。只见此人瘦削的身体上松垮垮地套着一套破旧却不失整洁的普通衣衫,头上梳着整整齐齐的发髻,单薄的身体立在雨后的寒风中却给人坚韧之感。洪小黑的目光遇到中年男子炯炯有神的目光后,便立马收回了视线,对着中年男子说:“大叔,千万别让这小贼进了你屋里。我来帮你把他压送官府。”说着便要去抓岳飞羽。

岳飞羽见他双手伸来,急忙往里躲,同时口中大叫:“二叔二叔,快帮拦住他!他力气太大了,我打不过他。”岳大公子本是天不怕地不怕的人,奈何洪小黑似乎就是他的克星,让他从心底里打不起武力反抗的念头,只好求助于他人。

二叔?洪小黑愣了一下。糟糕,这不是虎口夺食吗?洪小黑虽小,却也知道老牛舐犊之情,在他人长辈面前打人,焉有不被打之理。于他开始慌了,颤抖着收回伸出老远的双手,结结巴巴地说:“二------二------二叔好,我------我跟岳------岳小------不------不------岳------公子闹------闹着------玩------玩呢。既------既然您------您老人家------在,那------我------我改天------再------再来了,告------告辞------了。”话音未落,洪小黑转身一溜烟地跑了,瞬间消失了。

似乎洞察其中一切般,中年男子捋须一笑,对着岳飞羽说:“小羽,你这朋友很有意思。”

他不打我不挤兑我的时候是挺有意思的。岳飞羽心中老大不同意二叔的结论,但他本非喜欢说人是非之人,只在心中暗暗反驳二叔。

“二叔,我今天找你有很重要的事。”岳飞羽说出心中来意。原来他前几日无意中得知\marginpar{后面需要交待一下,岳东来夫妇不想他过早知道的原因,是因为怕他胡闹?}岳无双今日难得回府,自幼与姐姐亲密无间的岳飞羽自然想要给姐姐准备一个超级无敌大惊喜。然而自从上次出门被揍了之后,岳东来夫妇便不准他随意私自出门。今日好不容易逃跑重获自由,本想急急忙忙过来就找足智多谋的二叔来帮忙,无奈半路被洪小黑胡搅蛮缠浪费了大半天时间。

原来此中年男子姓雁名南飞,以面相与谋略闻名于草庐小巷一隅,附近百姓都管他叫神相雁南飞。然而鲜有人知道,雁南飞还有一本事,就是打铁,也就是铁匠。

雁南飞听得他来意,便知岳飞羽需要自己帮他打造某件物品。这岳飞羽年龄虽小,却心思聪敏,常常能想出些异想天开而大人却无法合理解释反驳的事情。果然,岳飞羽从怀里摸出一张桑皮纸摊开递给了雁南飞。纸上画着一个与发簪似又不似的图形,旁边写满了工工整整的蝇头小楷注解。

雁南飞看着这些工整的小字,心中安慰他没有落下习作书法的工夫。再瞧那图形时,雁南飞的脸慢慢地严肃起来。发簪的设计自然充满稚气,然而其中竟然暗含了五行灵气相生之意。雁南飞再去看岳飞羽,只见他一脸期待的看着自己,于是放下心思,暗想只是巧合罢。

“小羽你这个发簪,不能以普通铁匠的方式打造,需要辅以天铸,否则就只得其形而丢其魂。”

“二叔,什么天铸?”

原来普通铁匠打造的是凡人所使用的器具,若要使打造出来的器具拥有灵性则需要天铸。世上谁人都能当铁匠,唯有天铸师是天生的。大汉王朝八百万人口都没有发现过任何一个有当天铸师的资质。听着雁南飞如此解释,岳飞羽目光暗淡下来。他不去想为何他画出来的东西需要天铸师才能打造,只是一心在想如何才能打造出来给他亲爱的姐姐。他仍不死心,小心翼翼地问道:“二叔,既然你知道这发簪需要天铸师才能打造,那你应该有办法吧?”

“没有”。答案很干脆。岳飞羽很沮丧,好不容易逃出来,难道要功亏一篑吗?

“不过,有个方法倒是可以一试,反正死马当活马医了。”雁南飞在岳飞羽即将崩溃的时候给了他一根救命稻草,岳飞羽也立马抓住了:“什么方法?”

原来大汉王朝虽然没有出现过一位天铸师,然而在它开疆拓土的这些年来,也缴获了不少外域物品,其中就有一本关于天铸师的简介书籍流传到了民间,只是留意的人不多罢了。恰好雁南飞就收藏了一本。岳飞羽拿着这本《天铸开物引》,如获至宝,快速翻看起来。翻完后,岳飞羽仰天长叹一声:“二叔,你这本是天书啊。你能给我讲解讲解吗?”

果然只是巧合罢了。雁南飞摇头表示不能,含笑不语,伸手要接回书籍继续收藏。岳飞羽可不是那么容易死心的人,他把书往袖袋里一塞,往外跑了。“二叔,这书借我看几天,回头我就给您送过来。”雁南飞也不留他,任他去了。

怎么办,姐姐马上就要回来了,这玩意估计也做不出来了。启动备用方案吧,去蝶香斋买些胭脂水粉算了。想到这里,岳飞羽就往蝶香斋跑去,全然忘了自己仍未吃过东西。

咚!刚跑出巷子岳飞羽就觉得眼前一黑,与人撞了个满怀,又熟练地在地上摆了个大字。岳飞羽还未爬起来,就听得一个熟悉的声音在大叫:“岳~小~八!你洪爷爷今天真是倒了八辈子的大霉了!”

原来洪小黑反正闲来无事,索性就在巷口守着岳飞羽,没想成又被他撞倒一次。

怕是我才是倒了八辈子的大霉。岳飞羽可不敢再与洪小黑纠缠,正事要紧,再说他从来也纠缠不赢。于是趁着洪小黑还在揉脑门的绝好机会,岳飞羽迅速爬起来悄悄溜走了,一路顺利跑到蝶香斋。

蝶香斋的侍女倒也见多识广,知道岳飞羽这小娃儿是岳家公子,殷勤地招呼。好不容易在侍女的热情推荐下选了一堆胭脂水粉,正要掏袖袋里的私房钱结账,却蓦然发现二叔的那本书不翼而飞了!这一惊非同小可,岳飞羽快速结完账,原路返回,低头寻找。

洪小黑坐在与岳飞羽相撞的路边,正摇着一根茅草赶小咬,看见岳飞羽手提一个蝶香斋的大盒子,低头踩蚂蚁般地踱步走来。

“岳小八,地上有金子吗?能让我一起捡不?”洪小黑时刻不忘捉弄岳飞羽。我这可不是仇富,我是劫富,劫了他的气运,普济我们穷苦大众。洪小黑每次都会这样替自己开脱。

岳飞羽听着这声音就知道是谁,本着惹不过还躲不过么的原则,继续低头寻找,不理睬洪小黑。洪小黑见他不理会自己,便觉无趣,于是抛出杀手锏,说:“哈,刚才不知道是哪个王八在地上趴了一下,结果他的龟壳里竟然掉了一本天书。正好让洪爷爷今晚生火取暖。”

岳飞羽一听,便知书籍在洪小黑手里了。于是精气神一下上来了,提着偌大个蝶香斋的盒子一路小跑过去,满脸堆满笑容地说:“洪老弟,不,洪大哥,吃早饭没?没有?走,小弟请你吃你最喜欢吃的馒头。”

最喜欢吃的馒头?小爷最喜欢吃的是鸡腿,只不过平时连馒头也吃不起罢了。洪小黑白了他一眼,自言自语地说:“唉,昨天晚上梦到云来客栈在做凤入竹林,醒了都还是口角留香呀。”

岳飞羽一听头都大了,他的私房钱可不够去云来客栈点一个菜的,更何况刚刚在蝶香斋花了一大半。

洪小黑看他急得满头大汗,心里叫一个爽。

“洪大哥,平安巷新开一家馒头铺,他们家用的面引子特好,发出来的馒头香喷喷,特有嚼劲,回味无穷,口齿留香,那可是绕梁三日哪。小弟请您去尝尝?”

切,味道再好还不是馒头。洪小黑可不是大鱼大肉吃多了怀念吃素的人。洪小黑理都懒得理他,撇过头假装欣赏蓝天。

“洪大哥,南淮街那家老字号馒头铺,$\cdots\cdots$”

“洪大哥,天风道那家私房菜里的馒头,$\cdots\cdots$”

“洪大哥,山阳寺那里的和尚做的馒头,$\cdots\cdots$”

你烦不烦!来来去去除了馒头就是馒头,我就是那么像馒头的人吧?洪小黑恨不得破口大骂。

岳飞羽见洪小黑对他仍旧不理不睬,急得搔首挠头,毕竟他心里还存着残念,想要通过《天铸开物引》将他心中最完美的发簪打造出来送给无双。

“洪大哥,要不我明天亲自给您做馒头------”岳飞羽话未说完,只听啪的一声,他的嘴巴就被挡住了。原来洪小黑被他的馒头馒头馒头说得心烦头大,实在是忍不下去了就随手将最顺手的东西扔了过去。就这么巧,正好打到了岳飞羽的嘴巴,于是馒头声终于停了,世界清静了。

岳飞羽不停地告诫自己君子不与小人斗,何况现在还有求于人呢。同时接过从嘴巴上滑落的东西,定睛一看,得来全不费工夫,原来洪小黑扔过来的正是那本《天铸开物引》。岳飞羽拿到了书,心花怒放,也不去与洪小黑再作计较,毕竟今日在他手上吃的亏已经够多了,开局不顺,来日择个吉日再与他斗一场扳回场子罢。

“洪小黑,我要去吃平安巷的馒头,你来不?”岳飞羽失而复得,心情大好,真心实意地发出邀请。

“去,不吃白不吃!”洪小黑失手弄丢了谈判的筹码,恼怒地回答。

“岳小八,你提着个小姑娘的盒子干什么?难不成你有龙阳之癖?”一路上,洪小黑再次发扬了他优良风格,再次动用了他的言语武器。

岳飞羽决定不理他,否则只会自讨无趣。

见岳飞羽不回话不上当,洪小黑决定换个策略,先从他关注点入手。“岳小八,你要那本破破烂烂的天铸引干什么?”

“才不破烂呢,肯定是到了你手上被你弄破的。”岳飞羽反驳说。忽然他反应过来了,他一个小小的叫化子,如何认得了天铸开物引这几个字?“洪小黑,你怎么知道那本是什么书的?”

“你洪小爷天赋异秉,天生就会。世界上没有什么东西是你家洪小爷不认识的。”

岳飞羽也不在乎他吹牛,他关心的是另一件事。“那你肯定看得懂咯?”

“废话,都说了你洪小爷上知天文下知地理,区区一本天铸引,能难得到我?”

“洪大哥,你可真是我的大福星!今天的馒头,你要多少有多少!”岳飞羽想不到竟然在洪小黑这里找到了线索,激动得抱着洪小黑跳了起来。

“啊,你想干什么?勒死我了,快放我下来。”洪小黑黝黑的脸被他勒得都要变红了,一时竟然忘了自己力气比岳飞羽还大,没作挣扎。

岳飞羽嘿嘿一笑,拿出他的发簪杰作图给洪小黑看。“二叔说那本《天铸开物引》里可能有打造这个发簪的方法。快帮我看看有没有。”

洪小黑接过去,先给了他一个鄙视的眼神后再去看向那幅发簪设计图。“幼稚,无聊。”洪小黑手上作势就要将那纸张撕碎,眼睛却是偷偷地看向岳飞羽。

岳飞羽却是没有注意到他的小动作,急忙喊到:“不要撕!不要撕!”

洪小黑见好就收,笑吟吟地说:“岳小八,你的春天来了啊?看上谁家的姑娘了?让你洪大爷帮你把把关吧?”

“什么乱七八糟的,我看上谁家姑娘了?这个发簪我是要做了送给我姐的!”

“女大三,抱金砖。你还挺识货的。”

“住口!不要再胡闹了!那是我亲姐姐!”岳飞羽怒了,有些事他不会忍。

见他如此,洪小黑便知玩笑开大了,吐了吐舌头。为了补偿,便认真地看起那张发簪图。看完后便快速地在《天铸开物引》中翻查起来。“木之灵,在山之阳。$\cdots\cdots$,灌之以气,熏之以光。”洪小黑默默念叨着。过了良久,只见他展眉一笑:“真的是馒头管够?”啊呸,怎么我也说起馒头来了,被传染了被传染了。真是近朱者赤近墨者黑。可是说出去的话,泼出去的水,洪小黑也不是个无赖之人。

“管够管够!”岳飞羽听他如此说,知道他有办法了,激动得连忙点头,生怕他反悔不帮忙了。

“我要今天,明天,后天,不,一直到大大大大大大大后天都要管够!”既然是馒头,那就不跟他客气了。洪小黑一下子要了好多天的馒头。这下我真的成了洪馒头了,他暗暗自嘲。

“没问题,管够!”岳飞羽大方的回应。“那么,洪大哥,这个发簪应该如何打造呢?”

“我又不是铁匠,我怎么会?”

“可是你不是说你知道吗?”

“我是知道啊,可是我不会呀。有问题吗?”

你是没问题,可是我有问题啊。岳飞羽着急了,急则生乱,乱则丢了他本是遇事不惊的优势。

“你先找人把这发簪形状给打造出来呀,真是的,叫你岳小八真没叫错,榆木脑袋才可以活千年。”本着为将来好多天的馒头着想,洪小黑善意地提醒。

“你等着。”说完岳飞羽就跑了。

这小八子还真是跑不死的啊,不是说起床到现在都未进食的吗?不是刚晕醒的吗?感情是天马下凡吧,不吃不喝也能跑。想到这,洪小黑咯咯地自个笑了起来。

岳飞羽再次来到雁南飞的草庐,甫一进门,就二叔二叔地大声嚷到。雁南飞可没想到岳飞羽会杀个马枪,毕竟天赋这东西,有就是有,没有就是没有。天铸这行当,没有天赋的话任凭你折腾一万年也看不出来个门道。雁南飞正在书房里沉迷在他书中的黄金屋和颜如玉里呢,被岳飞羽的大嗓门吓了一跳,赶紧跑出来。

“只管用打铁的方式打出来?这个倒是简单的紧。”毕竟是小孩心性,雁南飞心想,也没去多想多问。岳飞羽则是一心想着快点弄好好让洪小黑开始下一步骤,也没多说。

发簪虽小巧,但设计者毕竟只是个小孩儿,再加上雁南飞本是个高明的铁匠,很快发簪就被打造出来了。岳飞羽定睛一看,开口夸到:“二叔果然巧手,这发簪就跟图中的一模一样。”岳飞羽虽是赞不绝口,脚下功夫却也没停,边赞边跑,话音未落便消失于草庐门外,进来时随手放在地上的蝶香斋小盒也不要了。

很快,岳飞羽便再次出现在洪小黑面前。洪小黑拿着雁南飞打造的发簪,他从未近距离观摩过如此精致的东西,竟有些爱不释手,细细把玩。岳飞羽可容不得洪小黑耽误时间,不停地催促他。

“看到这发簪上中间这一点没?画龙要点睛,做簪要点灵。用天铸引上面的话就是需要在上面灌灵。但你这个东西太幼稚了,太粗糙了,随便用点什么垃圾填上去就可以了。”

岳飞羽听他前面说的还挺正经,听到后面被他埋汰得脸都红了,但想到有求于人,只好忍着不作反抗,小心翼翼地问道:“那需要用些什么东西来填呢?”

“木之灵,在山之阳。要去山南采啊。得亏靖安在东山的南边,要是在北边的话,还得翻过东山到山对面才行。要是那样的话,你这几天的馒头可不够的。”洪小黑认真地算起了账。

岳飞羽听得此言,也不问他要采什么东西了,拉着洪小黑就往东山跑。

“我的馒头还没吃呢!”

“回来给你双份!”

\chapter{吾有断刀,可斩强豪}
\label{chap:xiao-ban-dao}

虽说靖安镇就在东山脚下,但两小孩儿也花了约摸个把时辰才到了东山的南山脚。东山是靖安的天然屏障,外面要到靖安,要不就翻过三面环绕着靖安的东山,要不就从其余一面的水路渡江而过。早在大汉王朝之前,东山上已经修了官道,时常人来人往,普通土著野兽早已被人吓得搬了家挪了窝,剩余的都是些不怕人的。

此刻乃清明前后,兼之刚下过暴雨,官道虽被多年的踩踏结实无比但此刻仍有些泥泞。不少措手不及被骤然而来的暴雨打湿了衣衫的旅客行人行色匆匆地在官道上走着。靖安虽是小镇,但历来地灵人杰,出过不少名人,导致过往靖安的行人也不少。

那时的人当家早,岳飞羽和洪小黑两人走在官道上也无人觉得诧异,更何况洪小黑的一身打扮让人一眼就辨别出他的职业,因此一路上也没有好心的大婶上来问两小孩是不是走丢啦,你们家长在哪里呀等等。两人倒也落得清闲。洪小黑老远看到酒旗招展,遂往地上一坐,大口喘气,说:“走不动啦!没力气啦!”

岳飞羽与他打交道以来,吃亏长智,听得他语气便知他心意。于是半拉半扯拽着他走到酒旗所在的小茶馆前。这是官道上专供行脚商歇息打尖的地方。骤雨初歇,来此喝杯热茶取暖的行人倒也不少。衣着朴素的老板和老板娘忙得乐开了花,笑得合不拢嘴。

茶馆人多桌少,岳飞羽二人根本找不到空位置,再者鉴于洪小黑的职业,先前座落的人大多对他没有好脸色,更加不会友善地挤挤让他们挤一起拼桌。还是洪小黑眼尖,瞧着一个雄壮男人一人一桌,拉着岳飞羽过去问也不问就坐下。岳飞羽向雄壮男人看去,只见他头带毡帽遮脸,一身玄色胡服打扮,背上用粗麻绳背着个长长的木匣。岳飞羽见他并无阻拦之意,便非常有家教非常老成地朝他拱了拱手,便也坐下了。岳飞羽高声唤到:“老板,来壶热茶,一笼大白馒头。”他稚嫩的童声显得格外明显,引得低头喝茶吃东西的客人纷纷抬头看来。老板娘循声看来,先见着洪小黑,本觉不喜,再见了岳飞羽,见他衣着虽不奢华却也明显异于常人,便知他不是普通人,便快速端了壶热茶与一笼馒头过来。

想来这小茶馆也没有什么高端货色,再说洪小黑本就不是挑食之人。见了热气腾腾的馒头,洪小黑一不挑剔二不见外,一手一个拿了就啃。岳飞羽似乎早已习以为常,也不去管他,拿了个馒头就着茶水也吃了些。正当他们茶足饭饱,洪小黑满足地捧着肚子闭眼享受之时,只听得官道上人马熙攘起来,山上的人马不断的快速涌了下来。忽然一匹驮满货物的马受到惊吓,甩开主人直直往茶馆冲过来,吓得路人慌忙躲闪,茶馆客人纷纷起身躲避。眼见那马就要冲进茶馆掀翻桌椅撞倒客人,老板和老板娘直吓得脸色苍白。岳飞羽抬头看得真切,急忙拉了犹自在闭眼不管窗外事的洪小黑一把,也要寻找安全方向。

说时迟,那时快。就在这千钧一发这际,只见那雄壮男人一个箭步冲向前,双手握住了那马的缰绳,顺势一个鹞子翻身稳稳当当地坐到了马背上,然后用力一拉,那马就被拉得前蹄腾空乱扑腾,后蹄停止不前,高声嘶吼起来。制止失控马匹之后,雄壮男子往先前的桌上抛下一枚碎银,腾空一跃,从茶馆上空的树上就飞走了。

想不到在靖安这个小镇也能看到活生生的武林高人,这趟行程太值当了,回去可有吹嘘的本钱了。路上各人纷纷窃语私议。

见过清风子的神奇,洪小黑倒也没那么惊叹于雄壮男人刚才的行为。倒是岳飞羽,虽然他知道岳东来武功不错,但他从未见过如此场景,直把他从头震撼到脚。男儿在世当如此,岳飞羽心想。他对此人立马生出了滔滔的崇拜之心。

失控马重新被主人安抚下来后,山上纷涌而下的人也渐渐少了起来。于是都交头接耳打探起来。原来有人似乎看到了久未曾出现过的吊睛白虎,于是一传十,十传百,吓得所有人都往山下跑了。幸亏此官道不是旅游胜地没有人山人海,否则发生踩踏事故必定会出现重多伤亡。

无知者无畏。岳飞羽也不管是否真的有白虎,结了茶款拉了洪小黑就走。

“大哥,岳爷爷,你没听说吗,有老虎啊,你还往里走啊?要去你自己去,我才不要跟你一起去送命。”洪小黑赖在地上不走,任凭岳飞羽如何拉扯就是不动。

“那我们就在山脚,不上去,可以吧?”岳飞羽拿他没招,又不死心,就想碰碰运气。洪小黑不为所动。

“再加十天馒头管饱。”

“二十天!”洪小黑还价了。

“成交!”

洪小黑可不敢往山上跑,就带着岳飞羽在山脚下人多的地方瞎转悠,看着蝴蝶就扑,见到蜻蜓就追。岳飞羽着急,不断的问他要找的是什么东西,洪小黑就一句告诉你你也不懂给他顶回去,慢慢地岳飞羽也就不问了,只管跟着他。洪小黑掐花扑蝶玩得兴起,不知不觉间二人已远离官道,远离人群。

洪小黑正瞪着一个五彩斑斓的罕见蝴蝶,蹑手蹑脚,酝酿致命一击要活抓此蝶,突然发觉右手被人轻轻握住。他蓦然一惊,回头一看,原来是岳飞羽。洪小黑正在开口大骂,只见岳飞羽另一手放嘴边示意他不要出声,然后轻轻指了指前面。洪小黑疑惑地朝他指着的方向看去,只见深深的草丛里十分隐蔽地隐藏着一只三角蛇头,不停地吞吐着长长的蛇信子。

毒蛇!

叫化子都是舞蛇高手,那都是假的。起码洪小黑不会,非但不会,还非常怕。他这一怕,立马觉得手脚发软,再无力气站立。就在他要倒下去时,那蛇出击了。洪小黑只觉脚踝一麻,眼睛一黑,便不醒人事了。

岳飞羽见状,也不管那蛇是否会再次攻击,弯腰抱起洪小黑,运用了全身的洪荒之力,转身就跑。跑了没多远,一脚踏空,二人齐齐扑倒在地,顺着缓缓的山脚往下滚,一直滚到一个半人深的山涧里。岳飞羽先着的地,紧接着又被落下的洪小黑砸了一次,于是就有了啊啊两声。岳飞羽倒也皮实,如此滚下来只是起了几个大包,倒也无其它大碍。他一骨碌坐了起来,只觉涧水之下屁股冰凉得紧。

他赶紧查看洪小黑的伤势,褪去他右脚破烂袜子,只见脚踝早已肿得像猪脚。那涧有半人高,他一时半会也无法将洪小黑驮上去,于是就坐在涧里想办法,依稀记得来福给他讲过的放血疗毒的民间故事。可是摸遍浑身也没有能放血的武器,这冰冷的涧水中也没有蚂蟥。罢了,洪小黑,碰到你总是我倒霉。再说若不是我拉你来,你也不会受这伤,救得了你就算扯平,救不了你就算赔你了。岳飞羽自幼忠厚,虽未习武却也满身正气。念及此,岳飞羽张嘴咬破洪小黑的伤口,给他吸起毒血来。岳飞羽吸了不多会,洪小黑有脚踝已然消肿,只剩乌黑一片。然而,这蛇毒遇冷还好,一遇到热,立马发作。那蛇毒到了岳飞羽口中,多少也进入了他身体。再吸了几口,岳飞羽便只觉昏昏沉沉,天旋地转,也倒了下去。临闭睛前只隐约听到了几声巨兽的嘶吼声。

却说茶馆那雄壮男人离开茶馆之后,很快便追踪到了众人口中的吊睛白虎。山上有白虎并不稀奇,稀奇的是绝迹之后再重现。雄壮男人毕竟艺高人胆大,追上白虎,抽出背上的刀,一把断了半截的刀,三刀两刀便伤了白虎。那白虎也是有灵性,见敌不过,便逃匿起来。断刀客一路追来,终于在东山山脚结果了它。断刀客正欲离去,忽然见到那条三角头的蛇在草丛中逃窜。断刀客见多识广,知道此蛇名烈焰峰,乃巨毒之物,但其毒液只能供其攻击一次,攻击后便要逃跑隐慝,等待一天后恢复。断刀客见那蛇逃跑,看得真切,手腕一抖,断刀飞出,将那蛇从七寸处斩断。蛇身虽断,那蛇尾犹在地上摆动。断刀客收回断刀,取了烈焰峰的胆用玉盒盛了。

烈焰逃跑,必有人伤。断刀客跳上一棵大树,举目望去,很快发现了不远处岳飞羽二人滚下山涧的痕迹。断刀客寻着二人,见其症状,已刻不容缓,连忙取出烈焰峰的胆,一分为二,给二人分别服下。原来,烈焰峰的毒用其胆可解。也幸亏二人早已昏迷,否则一分为二的蛇胆,怕是苦得没人能咽下去。断刀客任由二人在浅浅的涧水中泡着,也不搬他们上来。

过了约摸一柱香的功夫,料想蛇胆也差不多起了效用,断刀客将二人提起放到草丛上,一掌一人分别为二人灌输真气化除余毒。二人在断刀客的治疗下脸色渐渐由黑转白,再由白变红。然后,断刀客跳上树梢,再次消失。

良久之后,洪小黑率先醒来,想爬起来却发觉浑身乏力,扭头看去,见岳飞羽躺在边上一动不动,心想难不成那蛇见我是他叫化爷爷便放了我转而咬死了这厮?再过了一会,洪小黑恢复了点力气,坐了起来,便去察看一直麻麻的隐隐作痛的脚踝,只见上面隐约可见的一排小小的牙齿印,整整齐齐的甚是显眼。难道是这厮救了我?不对,蛇毒哪这么简单就能解完全的。洪小黑边迷惑边去探察岳飞羽的鼻息,见他呼吸平稳,便料想他也无事。便安坐一旁,边恢复体力边等岳飞羽醒来。

停不下来的洪小黑虽不能走动,眼睛却没闲,不断地四处打量。清风吹拂,洪小黑便隐约闻到了一阵香味。自幼对食物敏感的洪小黑很快便顺着气味寻到了源头,正是那烈焰峰攻击他们时的藏身之处。等恢复了足够的体力后,洪小黑慢慢挪了过去,拨开草丛,赫然发现里面竟然藏了一株龙涎草。龙涎草本是无味之物,想来是被那烈焰峰咬破了枝干才导致香气外泄。传说龙涎草乃是蕴含了龙息与光的产物,无味无毒,五行属火,可入药专治脾胃虚弱。鲜有人知道的是,按天铸引的说法,龙涎草可作灌灵的入门引子,打破屏障,导入天地灵气。

踏破铁鞋无觅处,得来全不废功夫。二十天馒头算是保住了。见岳飞羽仍未醒来,洪小黑也不好自己走人,好歹给自己吃过那么多馒头算是给自己吸过毒的人,便坐在龙涎草边上,拿出岳飞羽之前交给他的发簪,放在龙涎草边上。该死的,怎么灌灵呢,灌之以气,熏之以光,说的简单,哪来的气哪来的光,你大爷。

眼见二十天馒头仍有不保的风险,洪小黑孤注一掷,将那龙涎草连根拔起,揉成一团,胡乱涂抹在发簪上。奇迹不是随便有的。洪小黑心想看来我不是天选之子,故事里的都是骗人的。拿着抹满龙涎草碎末紫得发亮的发簪,洪小黑发起了呆。

岳飞羽也慢慢醒来,睁眼看到洪小黑在不远处发呆,便知他暂时无碍,心中欢喜,总算不欠他一条命。洪小黑听得他咳了一声,便转过来移到他身边,不成想刚坐下,岳飞羽毫无征兆地哇的一声喷了好大一口血,将洪小黑整个染成了血人。洪小黑啊的一声跳起来,手中的发簪也随手一丢丢到那半人深的山涧里去了。

“岳小八!老子跟你没完!”洪小黑正欲飞踢他一脚以泄心头怨气,却见他喷完血后又直挺挺地倒了下去,这次再无动弹。洪小黑慌了,期待的说着:“岳小八,别再装死了,这招已经过时了。”洪小黑再探他鼻息,还是平稳得很,想来是他起来过急血气一时供给不上所致,便安下心来,静待他再次醒来。

“傻呀!这里离官道那么近,直接喊人把他抬回去,岳家还不会大有赏吗?不对,我不能跟着去,否则岳家问起这厮是如何中毒,到头还是怪到我头上。”洪小黑跳入涧中摸回那发簪,放到岳飞羽怀里,便朝官道方向大喊:“救命啊,这里有人被毒蛇咬了。”

不久后便从官道方向走来了几名健壮的青年,手里都拿着一根长枝条。青年们顺着声音很快便找到了二人,其中一人指着岳飞羽问洪小黑:“小兄弟,怎么回事?是他中毒了么?”

“我们二人在此玩耍,谁知遇到了毒蛇,他被咬了一口,便成这样了。你们快帮忙送他回靖安岳家吧。”众人均是淳朴的穷苦人家,没那么多花花心思,见他一个小孩儿,也不疑他作假,便急忙扛起岳飞羽往靖安跑去。

“岳小八,你可要早点好,你还欠我二十天馒头呢。”

\chapter{山雨欲来风满楼}
\label{chap:shan-yu-yu-lai-feng-man-lou}

日薄西山但仍未落时,岳府已华灯初上,上上下下忙碌一团,准备着迎接小姐回来。唯有来福在边干活边发愣,他出去一上午都没寻着小少爷,心里担心得紧。

岳飞羽虽一天未回,但沐夫人似胸有成竹丝毫不担心,只是安排府里上下准备迎接双儿,准备晚宴。沐夫人站在府中最高的主楼上,迎着稍有凉意的晚风,突然心里无来由的一阵悸动。她抚着心口,微蹙着眉,情不自禁的想到岳东来,心中暗暗祈祷丈夫平安归来。

正想着,一阵急促的脚步声将她惊醒,只见金伯急匆匆地跑了上来。“金伯,可是老爷回来了?”

“夫人,不好了,少爷晕倒了,被人送了回来。”金伯跑得有点上气不接下气。

闻言沐夫人脑袋嗡的一声只觉天旋地转,世界都在变得模糊。她强忍着不倒下去,只是在大口大口喘气。“人呢?”

“已送回少爷房里,我已命人去请靖安城西华一针。还有二爷那里,我也遣了人去送信。只是老爷那里如何打算,还需夫人定夺。”金伯是见过风浪的人,临危不乱,早已安排的妥妥当当。

沐夫人不语,脚步不稳地要走去岳飞羽房间。旁边侍女赶紧上来搀扶着她。毕竟常居高位的人,很快沐夫人便恢复了常态,健步如飞步进了岳飞羽的房间。沐夫人虽不习武,但耳濡目染之下多少懂点武学医术常识,她探过岳飞羽的手后见他脉相平稳,便知他大概无事,醒来也是早晚之事,心便稍安下来。

此时,城西华一针也被接到。华一针乃靖安医术最高之人,亦是最古怪之人。传闻他治病从来只要一针,一针就痊愈。若一针不好,便不要再找他了,找了他也不会再施第二针。华一针替岳飞羽早已看过多次,对岳府及府中人相当熟悉,他对着沐夫人施了一礼,二话不说便坐在岳飞羽床边,给他把起脉来。华一针自不是沐夫人所能比的,岳飞羽平稳的脉相甚至能迷惑到一流医师,但在华一针眼里却显得有些奇怪。华一针脸色阴晴不定。过了很久,他轻吐一口浊气,对沐夫人说:“夫人但请放心,公子性命无碍,无需动作他自会醒来。”

沐夫人也是会察颜观色之人,观他脸色便知他有后话,便对转头说道:“金伯,公子既然无事,你忙去吧,接到双儿立马禀报于我。若兰、若兮,你二人去厢房给华大夫准备好茶。”金伯与两侍女领命而去。

沐夫人对华一针说:“华大夫,有话请讲。”

“老朽外号华一针,夫人应该知道我的规矩。但我多次进府替公子把脉,从未施过一针。夫人可知缘由?”

“大夫说过小儿脉相稳中带奇,如石上垒卵,平衡至极致。若受任意外力冲击,平衡即破。”

“没错。我施一针,是外力。我开药调和,亦是外力。故老朽从未替公子施针开方。然则今日,吾观公子脉相,平衡已破,恐心脉首当其冲,今后公子怕是要与噬心之痛作伴了。”

沐夫人如五雷轰顶。回过神后把华一针当救命稻草,急忙问到:“还请华大夫指点一二。”

华一针摇摇头,沉默不语。

“靖安西北八百里处,有一隐秘地方,名曰雪莲坞,鲜有人知。传闻雪莲坞中有一神医,能活死人,肉白骨。若是求得此人出手,公子之疾或可根除。”华一针是真没办法了,见沐夫人这般模样,于心不忍,便连传说都搬出来了。

果然,沐夫人如黑暗中见到光明,大海中遇到了陆地,一下充满了希望。华一针行医多年,虽生死早已看淡,却不忍戳破她的幻想,便告辞而去。

沐夫人轻抚岳飞羽双颊,双目垂泪,暗怨上苍不公,让岳飞羽命运多舛,自幼便承受常人所不能之痛。怨归怨,沐夫人还是双手合十,默诵心经祈祷神灵降福。一通经文下来,见岳飞羽仍是沉睡,沐夫人便思索起雪莲坞神医一事。待大哥与双儿回来,明日便派人去寻一遭吧。

沐夫人起身走出房外,掩上房门,便见金伯快步而来。

“华大夫送走了?”

“是,夫人。迎接小姐的人也回来了。”

“双儿回来了?哪儿呢,如何不先来见我?”沐夫人心中一喜。

“夫人,小姐没回来。来齐与来兴二人在东郊乌沙河对岸的迎客亭处守了一天也没见着影子,便着来兴先回来报信了。”

“这丫头与她师傅一同回来,想来也不会出什么意外。许是又遇难民,治病救人去了,导致耽搁了时辰。老爷那边有消息吗?”

“来勤到县衙门前打听过了,众人都瞧着老爷与吴克敌进了衙门。来勤守候多时,没听着衙门内传出异样,也不见老爷出来。来勤亦到县狱买通了狱卒,打听到今日未曾有新犯入监。”

沐夫人自是对枕边人充满信心,无论文武,料来这小小的靖安衙门还没有人是岳东来的对手。最让她放心不下的还是岳飞羽。

少时,雁南飞亦到了。雁南飞对着沐夫人行了一礼,见她除却脸上略有忧色之外,神色甚是淡定,便料想岳飞羽并无大碍。“嫂嫂,羽儿可好?”

“无碍,华大夫说他不日自会醒来,无需特别护理。”此处人多耳杂,沐夫人只说了一半。

“二哥,今早大哥被官府邀去,至今未回。你可有收到风声,能猜测官府此举何为?”沐夫人一五一十地将早间岳东来被袁克敌请走之事说与雁南飞听了。雁南飞听后,便细细思索起来。

“大哥素来不愿与官家牵扯,但也从未交恶。岳家在靖安以来也是安分守己,未见飞扬跋扈之事。至于钱银一事,以大哥的脾性,必定没有瞒税不报之举。如今为官当权上位者,无非钱权色。岳府生钱有道,与权无沾。再者岳府这等家底,根本入不了大人物眼里。至于色,府上女眷$\cdots\cdots$”

“双儿?”沐夫人与雁南飞对视一眼,不约而同的蹦出同一个名字。

“双儿这丫头自幼标致,人见人怜。多年不见,想来必已长成国色天香沉鱼落雁之姿。窈窕淑女,君子好逑。难不成双儿被上位者相中了?”雁南飞轻声自语,不断推测。

“若是相中,必要交好于我家。不来我岳府下拜贴,却将大哥带入衙门不放,这是哪门子的交好之理?”沐夫人不理解。

“大哥与嫂嫂均是善良有礼之人,自是不会做出此等行径。然则上位者久居高位,凡事有人在周边对其唯唯诺诺,多的是不可一世之徒。如若双儿遇到一家势权倾大汉天下的纨绔子弟,他如何会将小小靖安岳家放在眼里。没有强行将双儿抢去,只是将大哥扣压在衙门逼迫就范,对其而言,已是给足了双儿面子。”

沐夫人将信将疑,双儿打小乖巧,如何会招惹到那等人物?

“无需双儿去招惹。酒香不怕巷子深,富在深山有远亲。想来是双儿被人招惹了。”

沐夫人闻言,越想越有可能,再念及双儿至今未归,不觉又为小女担忧起来。便将来齐二人一天都未等到双儿之事也告诉了雁南飞。

“双儿的师尊是世外高人,凡世间还无能谋算她敢谋算她之人。嫂嫂但请放心,双儿必能平安归来。至于大哥,以大哥的本事,想来也能遇事化吉。”

他们结拜兄妹里,雁南飞是智谋核心,历来算无遗策。虽知他是安慰的话语,但听后亦自心安下来。晚风习习,吹得二人长袖翩翩欲舞。沐夫人无心感受,只觉阵阵寒意,遂对雁南飞说:“二哥,你去瞧瞧羽儿这孩子吧。”雁南飞点头称诺。二人不一会便到了岳飞羽房间,推门而入,便见到了安安静静躺在床上的岳飞羽。走近了只见岳飞羽脸色安详,气息平稳,乍一眼看去并无大碍。

“适才嫂嫂似乎有难言之隐,可否告知南飞参谋参谋?”雁南飞是聪颖之人,先前早已猜到沐夫人有未尽之言,心中疑惑,直到此刻才提起。沐夫人遂对雁南飞说了华一针所言之雪莲坞一事。雁南飞向来视岳飞羽如己出,闻得此言神色一变,但瞬间又恢复如常。

“华大夫乃靖安医术泰斗,比起吾等的野路子,实是高明太多。既然华大夫如此诊断,我们不可不妨。至于雪莲坞,我少年游历时却也听闻一二。至于雪莲坞的神医,”雁南飞本想说从未听闻雪莲坞神医,瞥了一眼沐夫人后便接着说到:“向来神龙见首不见尾,要寻此人,府中下人恐怕都难当此任。等大哥回来,我便带羽儿走一遭吧。”

“此行若有二哥同往,必能成功。”沐夫人大喜,她本有此意,欲待岳东来回来后便让其请求雁南飞往雪莲坞走一趟,不想雁南飞自行先提了出来。二人再商议了若干出行细节后,夜色终于吞噬完了最后一缕阳光,天渐渐暗了下来,府中灯火显得格外通红明亮。

“嫂嫂,那我先行回去了。若双儿或大哥回来,你再遣人报信于我,我便过来。”商议完毕,雁南飞便要告辞回去早做准备。沐夫人颌首回礼:“有劳二哥了。”雁南飞正欲转身回他的草庐,忽闻外面吵杂大作,金伯神色匆匆赶了进来,便停下脚步。

“夫人,二爷。门外袁克敌带了百多军士,明火执仗,将岳府团团围了起来。”

沐夫人闻言,脑袋又是嗡的一声,摇摇欲坠。今日之事,恐怕早已击溃了寻常女子。然则沐夫人不寻常,她很快便稳住了身形,眉头紧蹙。

“恐怕大哥已然从岳府脱身,否则县衙不会如此声势浩大地派出军士来此。夫人不宜抛头露面,待我先去会一会那袁克敌。”雁南飞在一旁开口说罢,拱手一礼,便出去了。

岳府位于靖安镇僻静的东南一隅,附近只有寻常百姓三五家,少见喧闹。雁南飞出得府门,只见门外整整齐齐每隔丈许距离便站了一个手中拿着火炬的盔甲武士。那火炬整齐地围了岳府一圈,照得靖安东南角通红一片,早已惊动了在家躲寒的靖安百姓,纷纷出门观望交头接耳。那袁克敌还是骑着那头高大的西北黄毛战马,如鹰般冰冷的目光直视雁南飞,手中握着的战刀拔出了半截。

“来者何人,速速报上名来。上峰有令,任何人不得出入岳府。擅闯者,格杀无论!”

“袁统领,在下东山脚下雁老二。敢问袁统领来此是奉了何人之令?”

“擅问者,杀无赦。”袁克敌手中的刀又拔出了少许,刀峰闪烁着寒冷的火光。

雁南飞知多说无益,便退了回去,嘱咐来德把门关好便去找沐夫人。

正堂里,沐夫人坐在主位上以手抬额,雁南飞坐在下手沉思不语,金伯站在一旁纹丝不动。一波未平一波又起。过了良久,雁南飞叹了口气,说:“大哥消失与双儿未归,此二者必有联系。如今岳府被围,想来是出现了对方不可控的因素。如今之势,岳府已成鱼肉。依我之计,唯有大嫂你带着羽儿从密道走了,否则久围必生变,恐危及大嫂与羽儿安危。”

沐夫人闻言默然不语。

“大嫂,此乃生死存亡之危,当断则断,切勿犹豫。留得清山在,不怕没柴烧。我与金伯留在此处照应,一来迷惑袁克敌,二来若大哥双儿归来,亦好联络。”

沐夫人亦是女中豪杰,稍做思考便下了决心,于是当机立断回屋收拾了细软。雁南飞抱了岳飞羽,四人来到岳府祭祀先人的祠堂。沐夫人对金伯点头示意,金伯便趴在灵位台下扣动了机关,一顿吱吱的机关声后,墙边便出现了一人宽的一条密道,黑膝膝的。金伯早有准备,点了个火把。

“二哥,你把羽儿给我,你下去接应。”雁南飞将岳飞羽递给沐夫人,跳下了密道,转头正欲让沐夫人将岳飞羽递给他,却见沐夫人紧紧抱着岳飞羽,火光中隐约可见两道泪痕在脸上流淌。雁南飞见状,便闭上了刚张开的嘴,没有出声。

过了良久,沐夫人似乎是下了狠心,将岳飞羽递与了密道中的雁南飞。雁南飞接过并放下岳飞羽,转身正欲返回时,只见沐夫人将她的包裹丢了下来,密道门亦开始徐徐合拢,雁南飞一惊,立马往外跳,刚露得半个头在外时,却见金伯双手推来。雁南飞在空中无处借力,被金伯一推便跌回了密道中。待得他爬起来时,密道门已然合拢了,再不见一丝亮光,伸手见不着五指。

“二哥,我一女流之辈,如何能带着羽儿逃脱得出去。就算出去了也寻不着那雪莲坞神医。此行还是二哥去把握大些,就有劳二哥多担待些。我与金伯留在此处照应。二者大哥双儿未回,我怎可一人离去。”

雁南飞在密道中只听得沐夫人的声音越来越轻,想是渐行渐远了。雁南飞苦笑一声,自认已是算无遗策,想不到还是屡次中了沐夫人的算计。\marginpar{岳府的劫难先告一段落,后续通过他人之口交待一下。}

\chapter{吾有仲子,忘忧忘情}
\label{chap:wang-you-wang-qing}

密道里黑得伸手不见五指,只闻得到一股久了不通风的霉味。雁南飞在密道中摸索到了岳飞羽及沐夫人的包裹,深一步浅一步地往前挪动。密道狭窄,呯的一声雁南飞背上的岳飞羽撞到了密道顶上凸出来的石块,带得雁南飞一个踉跄摔倒在地,身上的包裹也掉地上了。

“羽儿,你没事吧?”雁南飞急忙问道。见无人回应,才想起岳飞羽仍在昏迷当中。

雁南飞伸手要去摸那包裹,胡乱摸索间手臂竟用力将那包裹拔了一下。只见那摔下来后本已有些松散的包裹此刻发出些微弱的光芒,显然是包裹松开之后里面物品暴露了出来。就着这些光芒,雁南飞打开包裹查看,一张厚厚的黑布里赫然包裹着一颗拳头般大小的夜明珠!在夜明珠的照耀下,密道登时亮如黄昏,一切地形与布置看得清清楚楚。雁南飞在珠光下检查了包裹中的物品,除了那夜明珠外,还有一个散发着淡淡蓝光的玉佩,玉佩入手亦温亦凉,一面雕刻着喷火的五爪金龙,一面雕刻着一昂首飞翔的凤凰,一龙一凤栩栩如生。此外包裹里还有一把暗淡无光的短匕首,以及金叶、碎银与银票若干。

“大嫂倒想得周到。”雁南飞收拾了包裹,再次背起岳飞羽,依着夜明珠的亮光顺着密道往前走。

走出百米距离,便听得轰的一声闷响,刚走过的密道坍塌了!滚滚尘土向雁南飞二人直扑而来,良久才平息下来。雁南飞二人除了满身灰尘之外,倒是毫发无损。

雁南飞一阵默然,只觉喉间哽塞。以他之智,自是知道沐夫人此举之意,是下定了决心要为他二人争取时间。雁南飞虽非武林中人,但也豪迈非凡,很快便拾掇了心绪,再次快步而去。

雁南飞以步计时,算着在密道中走了大概有三个时辰,终于依稀见着了前方有微弱的亮光。走近了看,原来天色已晚,那亮光是从密道尽头的洞口处倾洒下来的月光与星光。雁南飞心中计算了一番,估算现在位置大约是在东山山脉深处,便谨慎起来。走到洞口一看,果然洞口开在了悬崖峭壁之上,向下看黑漆漆一片不知深浅。

雁南飞心想唯有等天亮再做打算,便寻了个靠里的位置将岳飞羽安置好,他自己坐在外侧以防岳飞羽翻身坠落山崖。他一边闭目养神,一边将日间之事在心中反复推演,也没理出个靠谱的头绪,便也缓缓地睡去了。

第二天清晨,雨后的晨光柔和地洒入洞内。雁南飞慢慢睁开眼,坐起来朝岳飞羽的方向瞥了一眼,便蓦地跳了起来。岳飞羽不见了!雁南飞虽然人称神相,素来谋而后动,然而此时情急之下,也只剩余大喊的本能。他羽儿羽儿的大喊着,可是只有洞内的回音在回应他。雁南飞便静了下来,借着晨光查看四周,思索对策。

过不多久,洞外传来隐隐约约的窸窸窣窣的声音,在寂静的洞穴中听得格外清楚。雁南飞心中一惊,难道是蛇?雁南飞正欲寻个有利的地形,找个称手的武器,却只见一个小小的黑影从洞外一跃而入,直接跳到了他的肩头!雁南飞大惊,哪里还管他是蛇不是蛇,有毒无毒,慌忙用手去拔。那小黑影却也灵活得紧,见雁南飞双手拔来,便灵巧地一跳,立于两米开外,躲过了雁南飞的双手。

雁南飞定了定神,朝那黑影看去,原来是一只漂亮的花栗鼠。只见那花栗鼠毛茸茸的,背上一道与众不同的显眼的紫色纵纹,尾巴宽大到能把它自己包裹起来。那花栗鼠也不怕人,正用那滴溜溜的眼珠子盯着雁南飞。雁南飞被它那宛如人般的眼神打量得上下不自在。雁南飞见多识广,知这花栗鼠必有灵性,想来非是寻常野兽,便欲想法子要将它捉了来。

“呯”雁南飞正想着呢,突然听得一声响,洞中又落下了一个人影,赫然便是岳飞羽。雁南飞见状,又惊又喜,早将捉拿花栗鼠一事忘了,赶紧过去扶起岳飞羽,仔细检查他身上是否完整。只见岳飞羽口青鼻肿,两根香肠般的嘴唇红的发亮,脸上其余地方倒是完好无损。雁南飞连忙问他缘由,可是岳飞羽只能哇哇比划,说不出话来。

比划了半天,连猜带蒙,雁南飞总算是大概知道了事情的经过。原来岳飞羽昏迷多时,天刚亮时便先于雁南飞醒来,见状不知发生什么变故,想问个明白却又不忍将雁南飞叫起。正无聊间,见着那个花栗鼠在洞口盯着他看,好奇心起,便跟了过去。那花栗鼠见他过来,便跳出洞外,灵活地顺着峭壁往上爬走了。见洞口连着一粗壮树枝,自幼胆大的岳飞羽便抓住树枝爬了上去,见那花栗鼠正在上方另一洞口处吃着不知名的果子呢。算来岳飞羽也是有几顿没吃了,见着花栗鼠吃得有滋有味,便觉饥肠辘辘,借着树枝落脚便也爬到了那洞中,抢了花栗鼠的果子便吃,直气得那花栗鼠哇哇乱叫。奈何岳飞羽体型比它大太多,那花栗鼠见抗议无效,便跳上到他头上,高高跃起,使劲在他头顶踩了一下泄愤。岳飞羽吃得急,被它一踢,被一果子呛着了,咳了半天。稍稍舒服后再去寻那花栗鼠报仇时,却早已不见了它踪影。而后听得雁南飞在叫自己名字,正欲回话时却突然发现只能发出呜哇之声。岳飞羽这一下被惊得非同小可,连忙爬回去找二叔帮忙。然而上山容易下山难,好不容易岳飞羽才回到了雁南飞所在的洞里。

雁南飞见他口青鼻肿,嘴不能言,猜想这多半是吃了那果子的结果。便与岳飞羽再次来到花栗鼠吃果子那洞里。那洞里还有若干剩余的果子,红彤彤的,鲜艳得要滴出水来。雁南飞虽不识此果,但他多了个心眼,撕破衣袖,包了几颗收好。

相生相克,雁南飞猜测附近可能会有能解此果毒的生物。便在这洞内外细细打量起来。只见那洞方丈见宽,在晨光中只见到丈许深,再住里便是黢黑一片,不知深浅。地面与墙壁均是光秃秃的岩石,不知那花栗鼠是从何处寻来的果子。正想着,忽觉一阵凉意袭来,原来是一股凉风自洞内吹了过来。雁南飞一喜,洞内必有乾坤!

雁南飞找了块石头用力往里一扔,却半天不见声响。正犹豫着,却见那花栗鼠不知从何处又冒了出来,嗖的一声跳到洞内漆黑处不见了。雁南飞不再犹豫,拿出夜明珠,与岳飞羽一同往洞内走去。岳飞羽虽是岳家少爷却也未见过夜明珠,见此甚是惊奇,跟着雁南飞深一步浅一步往里走。走不多时便到底了,变成了一个垂直向下的坑洞。

雁南飞交待了岳飞羽几句,便拿着夜明珠往下一跃,跳了下去。滑滑梯般滑了一会,雁南飞便落地了,眼前豁然开朗。可是还未等他感叹完毕,咚的一声,他便被从后面撞倒在地。原来岳飞羽见雁南飞跳了下去,没有片刻犹豫便跟着也跳了。二人爬起来,相视而笑。

原来此处竟是东山一未为人知的小山岰,其间花草丛生,果树密布,穿花蛱蝶深深见,点水蜻蜓款款飞。更为稀奇的是不远处竟有一人家,炊烟袅袅!

二人连忙走过去敲门。开门的是一老叟,鹤发童颜,精神抖擞,穿着一褐色长袍,整齐的道士髻甚是惹人注目。

“老丈,我叔侄二人落难至此,还请方便一二。”雁南飞连忙行礼。

“此处人烟罕至,你我能在此相见,亦是缘份。”老人眯眼微微一笑还礼。

“贫道河西散人久居于此未曾外出,不知居士何处人士?”

雁南飞只觉此名有些耳熟,只是一时回想不起。除却隐瞒了密道一事外,雁南飞将二人遭遇一五一十地告诉了河西散人,并取出那红色果子请河西散人鉴别。河西散人接过果子仔细端详,再看了岳飞羽脸上的症状,说:“居士无需过于担心。这是此处峭壁上的特产,名曰忘情果。寡情之人吃了自是无反应;然而多情之人吃了虽各有不同,轻则忘忧,重则忘情,却均无性命之忧。吾观小友面相端庄,天庭饱满,虽多有劫难却会逢凶化吉。行到水穷处,坐看云起时,雁居士只需静观其变即可。我这有颗清灵丸,倒是对他脸上的红肿略有效果。”

雁南飞接过药丸,却没立刻给岳飞羽服用,心里飞速回忆着河西散人这名字。他突然灵光一闪,难道是他?“道长可曾去过洛水白云观?”

“想不到居士还知道我白云观。没错,我就是那个师父要将白云观交到我手里时便逃了出来的那人。现在想来应该是师弟玉真人在管理罢。”

大汉王朝四大圣地,一观一寺二院,乃东方洛水城白云观,西方飞来峰灵隐寺,南方圣灵城麓山书院,以及北方武陵山尚武院。此四地,一道一僧,一文一武,乃大汉稳定之根基。

雁南飞不再有疑:“不想在此遇到白云观的河西散人,雁某当真三生有幸。”雁南飞本想问问他为何隐居于此,但此问未免涉及人家私隐,但忍而不问,转头便将药丸给岳飞羽喂了。岳飞羽不似洪小黑那般多疑,张嘴便咽了下去,只觉此药丸入口清凉,酸中带甜,苦中藏辛,似集百般滋味于一体。不一会,岳飞羽便觉喉咙搔痒,便哇的一声喊了出来:“哇------二叔,我能说话啦!”

河西散人笑而不语,雁南飞则是高兴地拍了拍岳飞羽肩膀。高兴之余,雁南飞又开始担忧起来,轻则忘忧,重则忘情。河西散人看穿了雁南飞所想,便对他说:“贫道对医药一道只是略懂皮毛,无能治疗小友此误吃忘情果之后症。但吾少年时曾遇一人,医术天下无双。你若寻得此人,必能解得了令侄之疾。但此人来去无常,无人知其踪迹。吾与之略有渊源,也只听闻其出身在雪莲坞。”

又是雪莲坞!

雁南飞双眼一亮,一事不烦二主,只要能找到雪坞神医,问题就都能迎刃而解了。

此时清风袭来,带来了淡淡的饭香。岳飞羽虽然能抗饿,但毕竟是半大小伙,正是能吃能喝的年纪,闻到这香味,肚子便不由自主地咕咕起来。河西散人看着满脸通红的岳飞羽,笑着说:“两位居士早上想来还未曾吃过吧,正好贫道今早的稀饭多煮了点,两位将就着吃点罢。”

“如此便叨扰了。”雁南飞一来亦是饿了,二来因曾听闻过河西散人的传言,也没推辞。

饭后雁南飞向河西散人问明了方向,便带着岳飞羽告辞而去。

两人走远后,只见那只花栗鼠不知从何处跳了出来,直接跳到了河西散人的肩头上。一人一鼠静立了一会,随后河西散人说:“你闻到了?”

“吱吱。”神奇了,那花栗鼠就像是听懂了似的竟然回应了河西散人。一人一鼠似乎是老伙伴!

“想去便去吧,莫要学我,一言不合就逃避。逃来逃去,追悔莫及!”河西散人说完便将那花栗鼠远远地抛了出去。

那花栗鼠轻盈地落地,转身盯着河西散人,半晌后转过身,快速飞奔而去。

却说雁南飞与岳飞羽二人沿着河西散人所指的方向,翻山越岭。岳飞羽走着走着,突然问题:“二叔,你是不是有什么事情要告诉我?”

雁南飞怔住了,羽儿猜到了?他看着岳飞羽,只见他满脸迷惑。雁南飞正想说出早已编好的故事,却见岳飞羽突然蹲地,双手猛地紧拽着头发。雁南飞心里一紧,要发作了,是华一针所说的痛心还是河西散人所说的忘忧忘情?雁南飞虽有智谋,医术却只是略懂些皮毛,只能紧盯着岳飞羽,防止他做出些出格之事伤害到他自己。

岳飞羽除了痛苦的模样,倒没做其他出格之事,他慢慢平静了下来,双手渐渐地松开了紧拽着的头发,抬起头四处看了看,然后眼光盯在了雁南飞脸上。

雁南飞见他双眼似要滴血般的通红,担心之极,连忙问道:“羽儿,你现在感觉如何,还有哪里不舒服不?”

岳飞羽怔怔地看着雁南飞那关切的眼睛,缓缓地说:“大叔,请问你是谁?我怎么会在这里?”

忘情果!

“呦呦鹿鸣,食野之苹;吾有仲子,忘忧忘情。”雁南飞喃喃地自言自语。



% \include{明月出天山}
% 
\chapter{Note}
\label{note}

时间:架空,古装。理由:历史不熟:(,喜欢长袍飘飘。

女诗经,男楚辞,文论语,武周易

\section{人物}
\label{sec:characters}

武学系统:八卦
天地人三才宫为基础,亦称之为三灵
天地人三灵纯净者为修仙所喜,三灵杂驳者难登仙途,只能学学凡间武艺。

武技:
金木水火土五行为基础
雷风暗时四变异属性


岳东来,本姓月,楼兰月氏,国破而东来。

沐素衣,本姓穆,昭武九姓之一。

岳飞羽岳大公子

岳无双

洪小黑,真名月芽儿

\section{地点}
\label{sec:place}

江宁府(南京) 靖安镇

\section{主线}
\label{sec:main-story}

官府带走父亲,姐姐回家出事变,大枪王庙,心月



望月,既是地名望月湖,亦是人名。

楚月儿(洪小黑),冀北大燕楚家

% \showthe\parskip

\appendix
\backmatter
\end{document}
