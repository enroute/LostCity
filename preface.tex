\chapter*{楔子}
\label{chap:preface}
\addcontentsline{toc}{chapter}{楔子}

\begin{pcontent}{|}
青海长云暗雪山,孤城遥望玉门关。
黄沙百战穿金甲,不破楼兰终不还!||
\end{pcontent}
\footnotetext{【唐·王昌龄·从军行】}

相传在楼兰国内有一望月湖,乃楼兰族人祭天祈福之圣地。每年的初春祭天日,望月湖边望月亭,便挤满了相恋之人。因为楼兰族人都坚信,望月湖内有仙灵,若两人得到望月湖的祝福,便是天选之缘,任谁也无法拆散了。

时过境迁,朝代更迭,在大汉的铁骑践踏下,楼兰古国早已灰飞烟灭。登高而望,极目之处皆荒凉,竟使人无语凝噎\footnote{\ptitle{宋}{柳永}{雨霖铃}寒蝉凄切,对长亭晚,骤雨初歇。都门帐饮无绪,留恋处,兰舟催发。执手相看泪眼,竟无语凝噎。念去去,千里烟波,暮霭沉沉楚天阔。\cispace 多情自古伤离别,更那堪冷落清秋节!今宵酒醒何处?杨柳岸,晓风残月。此去经年,应是良辰好景虚设。便纵有千种风情,更与何人说?},惟念天地之悠悠,独怆然而涕下\footnote{【唐·陈子昂·登幽州台歌】前不见古人,后不见来者。念天地之悠悠,独怆然而涕下。}。楼兰已毁灭,望月成追忆,这世间还有人能得到祝福么?



