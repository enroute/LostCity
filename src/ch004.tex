
\chapter{吾有断刀,可斩强豪}
\label{chap:xiao-ban-dao}

虽说靖安镇就在东山脚下,但两小孩儿也花了约摸个把时辰才到了东山的南山脚。东山是靖安的天然屏障,外面要到靖安,要不就翻过三面环绕着靖安的东山,要不就从其余一面的水路渡江而过。早在大汉王朝之前,东山上已经修了官道,时常人来人往,普通土著野兽早已被人吓得搬了家挪了窝,剩余的都是些不怕人的。

此刻乃清明前后,兼之刚下过暴雨,官道虽被多年的踩踏结实无比但此刻仍有些泥泞。不少措手不及被骤然而来的暴雨打湿了衣衫的旅客行人行色匆匆地在官道上走着。靖安虽是小镇,但历来地灵人杰,出过不少名人,导致过往靖安的行人也不少。

那时的人当家早,岳飞羽和洪小黑两人走在官道上也无人觉得诧异,更何况洪小黑的一身打扮让人一眼就辨别出他的职业,因此一路上也没有好心的大婶上来问两小孩是不是走丢啦,你们家长在哪里呀等等。两人倒也落得清闲。洪小黑老远看到酒旗招展,遂往地上一坐,大口喘气,说:“走不动啦!没力气啦!”

岳飞羽与他打交道以来,吃亏长智,听得他语气便知他心意。于是半拉半扯拽着他走到酒旗所在的小茶馆前。这是官道上专供行脚商歇息打尖的地方。骤雨初歇,来此喝杯热茶取暖的行人倒也不少。衣着朴素的老板和老板娘忙得乐开了花,笑得合不拢嘴。

茶馆人多桌少,岳飞羽二人根本找不到空位置,再者鉴于洪小黑的职业,先前座落的人大多对他没有好脸色,更加不会友善地挤挤让他们挤一起拼桌。还是洪小黑眼尖,瞧着一个雄壮男人一人一桌,拉着岳飞羽过去问也不问就坐下。岳飞羽向雄壮男人看去,只见他头带毡帽遮脸,一身玄色胡服打扮,背上用粗麻绳背着个长长的木匣。岳飞羽见他并无阻拦之意,便非常有家教非常老成地朝他拱了拱手,便也坐下了。岳飞羽高声唤到:“老板,来壶热茶,一笼大白馒头。”他稚嫩的童声显得格外明显,引得低头喝茶吃东西的客人纷纷抬头看来。老板娘循声看来,先见着洪小黑,本觉不喜,再见了岳飞羽,见他衣着虽不奢华却也明显异于常人,便知他不是普通人,便快速端了壶热茶与一笼馒头过来。

想来这小茶馆也没有什么高端货色,再说洪小黑本就不是挑食之人。见了热气腾腾的馒头,洪小黑一不挑剔二不见外,一手一个拿了就啃。岳飞羽似乎早已习以为常,也不去管他,拿了个馒头就着茶水也吃了些。正当他们茶足饭饱,洪小黑满足地捧着肚子闭眼享受之时,只听得官道上人马熙攘起来,山上的人马不断的快速涌了下来。忽然一匹驮满货物的马受到惊吓,甩开主人直直往茶馆冲过来,吓得路人慌忙躲闪,茶馆客人纷纷起身躲避。眼见那马就要冲进茶馆掀翻桌椅撞倒客人,老板和老板娘直吓得脸色苍白。岳飞羽抬头看得真切,急忙拉了犹自在闭眼不管窗外事的洪小黑一把,也要寻找安全方向。

说时迟,那时快。就在这千钧一发这际,只见那雄壮男人一个箭步冲向前,双手握住了那马的缰绳,顺势一个鹞子翻身稳稳当当地坐到了马背上,然后用力一拉,那马就被拉得前蹄腾空乱扑腾,后蹄停止不前,高声嘶吼起来。制止失控马匹之后,雄壮男子往先前的桌上抛下一枚碎银,腾空一跃,从茶馆上空的树上就飞走了。

想不到在靖安这个小镇也能看到活生生的武林高人,这趟行程太值当了,回去可有吹嘘的本钱了。路上各人纷纷窃语私议。

见过清风子的神奇,洪小黑倒也没那么惊叹于雄壮男人刚才的行为。倒是岳飞羽,虽然他知道岳东来武功不错,但他从未见过如此场景,直把他从头震撼到脚。男儿在世当如此,岳飞羽心想。他对此人立马生出了滔滔的崇拜之心。

失控马重新被主人安抚下来后,山上纷涌而下的人也渐渐少了起来。于是都交头接耳打探起来。原来有人似乎看到了久未曾出现过的吊睛白虎,于是一传十,十传百,吓得所有人都往山下跑了。幸亏此官道不是旅游胜地没有人山人海,否则发生踩踏事故必定会出现重多伤亡。

无知者无畏。岳飞羽也不管是否真的有白虎,结了茶款拉了洪小黑就走。

“大哥,岳爷爷,你没听说吗,有老虎啊,你还往里走啊?要去你自己去,我才不要跟你一起去送命。”洪小黑赖在地上不走,任凭岳飞羽如何拉扯就是不动。

“那我们就在山脚,不上去,可以吧?”岳飞羽拿他没招,又不死心,就想碰碰运气。洪小黑不为所动。

“再加十天馒头管饱。”

“二十天!”洪小黑还价了。

“成交!”

洪小黑可不敢往山上跑,就带着岳飞羽在山脚下人多的地方瞎转悠,看着蝴蝶就扑,见到蜻蜓就追。岳飞羽着急,不断的问他要找的是什么东西,洪小黑就一句告诉你你也不懂给他顶回去,慢慢地岳飞羽也就不问了,只管跟着他。洪小黑掐花扑蝶玩得兴起,不知不觉间二人已远离官道,远离人群。

洪小黑正瞪着一个五彩斑斓的罕见蝴蝶,蹑手蹑脚,酝酿致命一击要活抓此蝶,突然发觉右手被人轻轻握住。他蓦然一惊,回头一看,原来是岳飞羽。洪小黑正在开口大骂,只见岳飞羽另一手放嘴边示意他不要出声,然后轻轻指了指前面。洪小黑疑惑地朝他指着的方向看去,只见深深的草丛里十分隐蔽地隐藏着一只三角蛇头,不停地吞吐着长长的蛇信子。

毒蛇!

叫化子都是舞蛇高手,那都是假的。起码洪小黑不会,非但不会,还非常怕。他这一怕,立马觉得手脚发软,再无力气站立。就在他要倒下去时,那蛇出击了。洪小黑只觉脚踝一麻,眼睛一黑,便不醒人事了。

岳飞羽见状,也不管那蛇是否会再次攻击,弯腰抱起洪小黑,运用了全身的洪荒之力,转身就跑。跑了没多远,一脚踏空,二人齐齐扑倒在地,顺着缓缓的山脚往下滚,一直滚到一个半人深的山涧里。岳飞羽先着的地,紧接着又被落下的洪小黑砸了一次,于是就有了啊啊两声。岳飞羽倒也皮实,如此滚下来只是起了几个大包,倒也无其它大碍。他一骨碌坐了起来,只觉涧水之下屁股冰凉得紧。

他赶紧查看洪小黑的伤势,褪去他右脚破烂袜子,只见脚踝早已肿得像猪脚。那涧有半人高,他一时半会也无法将洪小黑驮上去,于是就坐在涧里想办法,依稀记得来福给他讲过的放血疗毒的民间故事。可是摸遍浑身也没有能放血的武器,这冰冷的涧水中也没有蚂蟥。罢了,洪小黑,碰到你总是我倒霉。再说若不是我拉你来,你也不会受这伤,救得了你就算扯平,救不了你就算赔你了。岳飞羽自幼忠厚,虽未习武却也满身正气。念及此,岳飞羽张嘴咬破洪小黑的伤口,给他吸起毒血来。岳飞羽吸了不多会,洪小黑有脚踝已然消肿,只剩乌黑一片。然而,这蛇毒遇冷还好,一遇到热,立马发作。那蛇毒到了岳飞羽口中,多少也进入了他身体。再吸了几口,岳飞羽便只觉昏昏沉沉,天旋地转,也倒了下去。临闭睛前只隐约听到了几声巨兽的嘶吼声。

却说茶馆那雄壮男人离开茶馆之后,很快便追踪到了众人口中的吊睛白虎。山上有白虎并不稀奇,稀奇的是绝迹之后再重现。雄壮男人毕竟艺高人胆大,追上白虎,抽出背上的刀,一把断了半截的刀,三刀两刀便伤了白虎。那白虎也是有灵性,见敌不过,便逃匿起来。断刀客一路追来,终于在东山山脚结果了它。断刀客正欲离去,忽然见到那条三角头的蛇在草丛中逃窜。断刀客见多识广,知道此蛇名烈焰峰,乃巨毒之物,但其毒液只能供其攻击一次,攻击后便要逃跑隐慝,等待一天后恢复。断刀客见那蛇逃跑,看得真切,手腕一抖,断刀飞出,将那蛇从七寸处斩断。蛇身虽断,那蛇尾犹在地上摆动。断刀客收回断刀,取了烈焰峰的胆用玉盒盛了。

烈焰逃跑,必有人伤。断刀客跳上一棵大树,举目望去,很快发现了不远处岳飞羽二人滚下山涧的痕迹。断刀客寻着二人,见其症状,已刻不容缓,连忙取出烈焰峰的胆,一分为二,给二人分别服下。原来,烈焰峰的毒用其胆可解。也幸亏二人早已昏迷,否则一分为二的蛇胆,怕是苦得没人能咽下去。断刀客任由二人在浅浅的涧水中泡着,也不搬他们上来。

过了约摸一柱香的功夫,料想蛇胆也差不多起了效用,断刀客将二人提起放到草丛上,一掌一人分别为二人灌输真气化除余毒。二人在断刀客的治疗下脸色渐渐由黑转白,再由白变红。然后,断刀客跳上树梢,再次消失。

良久之后,洪小黑率先醒来,想爬起来却发觉浑身乏力,扭头看去,见岳飞羽躺在边上一动不动,心想难不成那蛇见我是他叫化爷爷便放了我转而咬死了这厮?再过了一会,洪小黑恢复了点力气,坐了起来,便去察看一直麻麻的隐隐作痛的脚踝,只见上面隐约可见的一排小小的牙齿印,整整齐齐的甚是显眼。难道是这厮救了我?不对,蛇毒哪这么简单就能解完全的。洪小黑边迷惑边去探察岳飞羽的鼻息,见他呼吸平稳,便料想他也无事。便安坐一旁,边恢复体力边等岳飞羽醒来。

停不下来的洪小黑虽不能走动,眼睛却没闲,不断地四处打量。清风吹拂,洪小黑便隐约闻到了一阵香味。自幼对食物敏感的洪小黑很快便顺着气味寻到了源头,正是那烈焰峰攻击他们时的藏身之处。等恢复了足够的体力后,洪小黑慢慢挪了过去,拨开草丛,赫然发现里面竟然藏了一株龙涎草。龙涎草本是无味之物,想来是被那烈焰峰咬破了枝干才导致香气外泄。传说龙涎草乃是蕴含了龙息与光的产物,无味无毒,五行属火,可入药专治脾胃虚弱。鲜有人知道的是,按天铸引的说法,龙涎草可作灌灵的入门引子,打破屏障,导入天地灵气。

踏破铁鞋无觅处,得来全不废功夫。二十天馒头算是保住了。见岳飞羽仍未醒来,洪小黑也不好自己走人,好歹给自己吃过那么多馒头算是给自己吸过毒的人,便坐在龙涎草边上,拿出岳飞羽之前交给他的发簪,放在龙涎草边上。该死的,怎么灌灵呢,灌之以气,熏之以光,说的简单,哪来的气哪来的光,你大爷。

眼见二十天馒头仍有不保的风险,洪小黑孤注一掷,将那龙涎草连根拔起,揉成一团,胡乱涂抹在发簪上。奇迹不是随便有的。洪小黑心想看来我不是天选之子,故事里的都是骗人的。拿着抹满龙涎草碎末紫得发亮的发簪,洪小黑发起了呆。

岳飞羽也慢慢醒来,睁眼看到洪小黑在不远处发呆,便知他暂时无碍,心中欢喜,总算不欠他一条命。洪小黑听得他咳了一声,便转过来移到他身边,不成想刚坐下,岳飞羽毫无征兆地哇的一声喷了好大一口血,将洪小黑整个染成了血人。洪小黑啊的一声跳起来,手中的发簪也随手一丢丢到那半人深的山涧里去了。

“岳小八!老子跟你没完!”洪小黑正欲飞踢他一脚以泄心头怨气,却见他喷完血后又直挺挺地倒了下去,这次再无动弹。洪小黑慌了,期待的说着:“岳小八,别再装死了,这招已经过时了。”洪小黑再探他鼻息,还是平稳得很,想来是他起来过急血气一时供给不上所致,便安下心来,静待他再次醒来。

“傻呀!这里离官道那么近,直接喊人把他抬回去,岳家还不会大有赏吗?不对,我不能跟着去,否则岳家问起这厮是如何中毒,到头还是怪到我头上。”洪小黑跳入涧中摸回那发簪,放到岳飞羽怀里,便朝官道方向大喊:“救命啊,这里有人被毒蛇咬了。”

不久后便从官道方向走来了几名健壮的青年,手里都拿着一根长枝条。青年们顺着声音很快便找到了二人,其中一人指着岳飞羽问洪小黑:“小兄弟,怎么回事?是他中毒了么?”

“我们二人在此玩耍,谁知遇到了毒蛇,他被咬了一口,便成这样了。你们快帮忙送他回靖安岳家吧。”众人均是淳朴的穷苦人家,没那么多花花心思,见他一个小孩儿,也不疑他作假,便急忙扛起岳飞羽往靖安跑去。

“岳小八,你可要早点好,你还欠我二十天馒头呢。”