

\chapter{善藏于民}
\label{chap:cang-shan-yu-min}

大汉人最是迷信,平民尤甚。杨老头听得雁南飞此言,便再次坐下,将信将疑地问道:“燕先生何出此言?”

雁南飞捋了捋胡须,俯首靠过去,低声问道:“老丈,先不说那莫二公子。你观他身旁那些奴仆,有何感想?”

杨老头低头默然不语。

雁南飞见他脸色,心中了然,便接着说道:“想来老丈心中已有结论,正所谓上梁不正下梁歪。如若那莫二公子果真是个正直善良之辈,如何容得下这般贴身下人?那些无非是他打猎用的鹰犬耳。”

“你想,那莫二公子在莫家堡里是何等人物?如今因何会对杨姑娘如何礼遇?不过,莫二公子既然是名声在外,对付你这么个无权无势在人家地盘上讨生活的老头子,想来他是不会亲自出手的。但那些鹰犬呢?”

杨老头虽然老实,但能把兮如这么个标致姑娘完好无损地养这么大,显然也不是个蠢人,一点即透。如今被雁南飞点醒,只觉口干舌燥,便颤抖地拿起茶杯慌张地喝了一口,不想却被那粗糙的茶叶呛了一口,弄得衣袖都湿了一片。

“这可如何是好?还请先生想个法子。”杨老头急病乱投医,直把雁南飞当作了神仙。

“爹爹无需担心,若是那莫二公子来强的,女儿便从了吧。”旁边一直没有言语的兮如突然幽幽地说了一句。

“你混帐!你从来都不是贪图富贵的人,难不成今日被那莫二给迷住了不成!”杨老头登时心头火起,大声喝到。那正在打盹的店家被吓了一跳,睁开了眼瞥了他们一眼,随即又闭上去寻那庄周晓梦去了。

雁南飞亦是一惊,再次端详起兮如。爱屋及乌,他对这初见面的兮如有好感,不愿相信她有点点瑕疵,想要从她表情上找出给她开脱的理由。

“二叔,那个漂亮的公子哥是坏人吗?”岳飞羽人虽小,却也大概理解了他们的言语。

“羽儿,画龙画虎难画骨,知人知面不知心。人心与皮囊是不相干的。”

岳飞羽似懂非懂的点了点头,突然又想起了什么,再次问道:“可你不经常说过相由心生吗,那好人自是好看的吧?”

“好人自是好看的,但需要你用心眼去看,而不是用肉眼去看。至于相由心生,你搞反了,不是别人的心造就的别人的相,而是你的心观察到的别人的相。”岳飞羽隐约了解了雁南飞的意思,便不再言语,静坐一旁细细体味去了。

兮如初见莫二公子时那脸上的潮红早已褪去,此刻略显稚嫩的脸上透露着的只有坚定。雁南飞沉吟半晌,突然问道:“老丈家中可还有什么人?”