\chapter{结庐东山下,悠然采南花}
\label{chap:cai-ju-dong-nan-xia}

不知不觉间二人已回到了靖安镇,跑到了一偏僻巷子一小草庐前。此时清明的微风早已将发过威风的雨云吹散,天空恢复了午时的明亮。路边铺的石头水渍也在渐渐消隐。

岳飞羽推门而入,嘴里大喊二叔。一中年男子闻声而出,从里屋走了出来,见了双手撑在膝盖上大口喘气的岳飞羽,便眼睛一亮,招呼岳飞羽:“小羽来了,快到屋里来。”

岳飞羽正欲跨步往里走,洪小黑也到了。原来洪小黑见他进了草庐,心想正好来个瓮中捉鳖,便跟了进来。洪小黑正欲往岳飞羽扑去,见着边上的中年男子,便硬生生停了下来,定睛打量起中年男子。只见此人瘦削的身体上松垮垮地套着一套破旧却不失整洁的普通衣衫,头上梳着整整齐齐的发髻,单薄的身体立在雨后的寒风中却给人坚韧之感。洪小黑的目光遇到中年男子炯炯有神的目光后,便立马收回了视线,对着中年男子说:“大叔,千万别让这小贼进了你屋里。我来帮你把他押送官府。”说着便要去抓岳飞羽。

岳飞羽见他双手伸来,急忙往里躲,同时口中大叫:“二叔二叔,快帮拦住他!他力气太大了,我打不过他。”岳大公子本是天不怕地不怕的人,奈何洪小黑似乎就是他的克星,让他从心底里打不起武力反抗的念头,只好求助于他人。

二叔?洪小黑愣了一下。糟糕,这不是虎口夺食吗?洪小黑虽小,却也知道老牛舐犊之情,在他人长辈面前打人,焉有不被打之理。于他开始慌了,颤抖着收回伸出老远的双手,结结巴巴地说:“二------二------二叔好,我------我跟岳------岳小------不------不------岳------公子闹------闹着------玩------玩呢。既------既然您------您老人家------在,那------我------我改天------再------再来了,告------告辞------了。”话音未落,洪小黑转身一溜烟地跑了,瞬间消失了。

似乎洞察其中一切般,中年男子捋须一笑,对着岳飞羽说:“小羽,你这朋友很有意思。”

他不打我不挤兑我的时候是挺有意思的。岳飞羽心中老大不同意二叔的结论,但他本非喜欢说人是非之人,只在心中暗暗反驳二叔。

“二叔,我今天找你有很重要的事。”岳飞羽说出心中来意。原来他前几日无意中得知\marginpar{后面需要交待一下,岳东来夫妇不想他过早知道的原因,是因为怕他胡闹?}岳无双今日难得回府,自幼与姐姐亲密无间的岳飞羽自然想要给姐姐准备一个超级无敌大惊喜。然而自从上次出门被揍了之后,岳东来夫妇便不准他随意私自出门。今日好不容易逃跑重获自由,本想急急忙忙过来就找足智多谋的二叔来帮忙,无奈半路被洪小黑胡搅蛮缠浪费了大半天时间。

原来此中年男子姓雁名南飞,以面相与谋略闻名于草庐小巷一隅,附近百姓都管他叫神相雁南飞。然而鲜有人知道,雁南飞还有一本事,就是打铁,也就是铁匠。

雁南飞听得他来意,便知岳飞羽需要自己帮他打造某件物品。这岳飞羽年龄虽小,却心思聪敏,常常能想出些异想天开而大人却无法合理解释反驳的事情。果然,岳飞羽从怀里摸出一张桑皮纸摊开递给了雁南飞。纸上画着一个与发簪似又不似的图形,旁边写满了工工整整的蝇头小楷注解。

雁南飞看着这些工整的小字,心中安慰他没有落下习作书法的工夫。再瞧那图形时,雁南飞的脸慢慢地严肃起来。发簪的设计自然充满稚气,然而其中竟然暗含了五行灵气相生之意。雁南飞再去看岳飞羽,只见他一脸期待的看着自己,于是放下心思,暗想只是巧合罢。

“小羽你这个发簪,不能以普通铁匠的方式打造,需要辅以天铸,否则就只得其形而丢其魂。”

“二叔,什么天铸?”

原来普通铁匠打造的是凡人所使用的器具,若要使打造出来的器具拥有灵性则需要天铸。世上谁人都能当铁匠,唯有天铸师是天生的。大汉王朝八百万人口都没有发现过任何一个有当天铸师的资质。听着雁南飞如此解释,岳飞羽目光暗淡下来。他不去想为何他画出来的东西需要天铸师才能打造,只是一心在想如何才能打造出来给他亲爱的姐姐。他仍不死心,小心翼翼地问道:“二叔,既然你知道这发簪需要天铸师才能打造,那你应该有办法吧?”

“没有”。答案很干脆。岳飞羽很沮丧,好不容易逃出来,难道要功亏一篑吗?

“不过,有个方法倒是可以一试,反正死马当活马医了。”雁南飞在岳飞羽即将崩溃的时候给了他一根救命稻草,岳飞羽也立马抓住了:“什么方法?”

原来大汉王朝虽然没有出现过一位天铸师,然而在它开疆拓土的这些年来,也缴获了不少外域物品,其中就有一本关于天铸师的简介书籍流传到了民间,只是留意的人不多罢了。恰好雁南飞就收藏了一本。岳飞羽拿着这本《天铸开物引》,如获至宝,快速翻看起来。翻完后,岳飞羽仰天长叹一声:“二叔,你这本是天书啊。你能给我讲解讲解吗?”

果然只是巧合罢了。雁南飞摇头表示不能,含笑不语,伸手要接回书籍继续收藏。岳飞羽可不是那么容易死心的人,他把书往袖袋里一塞,往外跑了。“二叔,这书借我看几天,回头我就给您送过来。”雁南飞也不留他,任他去了。

怎么办,姐姐马上就要回来了,这玩意估计也做不出来了。启动备用方案吧,去蝶香斋买些胭脂水粉算了。想到这里,岳飞羽就往蝶香斋跑去,全然忘了自己仍未吃过东西。

咚!刚跑出巷子岳飞羽就觉得眼前一黑,与人撞了个满怀,又熟练地在地上摆了个大字。岳飞羽还未爬起来,就听得一个熟悉的声音在大叫:“岳~小~八!你洪爷爷今天真是倒了八辈子的大霉了!”

原来洪小黑反正闲来无事,索性就在巷口守着岳飞羽,没想成又被他撞倒一次。

怕是我才是倒了八辈子的大霉。岳飞羽可不敢再与洪小黑纠缠,正事要紧,再说他从来也纠缠不赢。于是趁着洪小黑还在揉脑门的绝好机会,岳飞羽迅速爬起来悄悄溜走了,一路顺利跑到蝶香斋。

蝶香斋的侍女倒也见多识广,知道岳飞羽这小娃儿是岳家公子,殷勤地招呼。好不容易在侍女的热情推荐下选了一堆胭脂水粉,正要掏袖袋里的私房钱结账,却蓦然发现二叔的那本书不翼而飞了!这一惊非同小可,岳飞羽快速结完账,原路返回,低头寻找。

洪小黑坐在与岳飞羽相撞的路边,正摇着一根茅草赶小咬,看见岳飞羽手提一个蝶香斋的大盒子,低头踩蚂蚁般地踱步走来。

“岳小八,地上有金子吗?能让我一起捡不?”洪小黑时刻不忘捉弄岳飞羽。我这可不是仇富,我是劫富,劫了他的气运,普济我们穷苦大众。洪小黑每次都会这样替自己开脱。

岳飞羽听着这声音就知道是谁,本着惹不过还躲不过么的原则,继续低头寻找,不理睬洪小黑。洪小黑见他不理会自己,便觉无趣,于是抛出杀手锏,说:“哈,刚才不知道是哪个王八在地上趴了一下,结果他的龟壳里竟然掉了一本天书。正好让洪爷爷今晚生火取暖。”

岳飞羽一听,便知书籍在洪小黑手里了。于是精气神一下上来了,提着偌大个蝶香斋的盒子一路小跑过去,满脸堆满笑容地说:“洪老弟,不,洪大哥,吃早饭没?没有?走,小弟请你吃你最喜欢吃的馒头。”

最喜欢吃的馒头?小爷最喜欢吃的是鸡腿,只不过平时连馒头也吃不起罢了。洪小黑白了他一眼,自言自语地说:“唉,昨天晚上梦到云来客栈在做凤入竹林,醒了都还是口角留香呀。”

岳飞羽一听头都大了,他的私房钱可不够去云来客栈点一个菜的,更何况刚刚在蝶香斋花了一大半。

洪小黑看他急得满头大汗,心里叫一个爽。

“洪大哥,平安巷新开一家馒头铺,他们家用的面引子特好,发出来的馒头香喷喷,特有嚼劲,回味无穷,口齿留香,那可是绕梁三日哪。小弟请您去尝尝?”

切,味道再好还不是馒头。洪小黑可不是大鱼大肉吃多了怀念吃素的人。洪小黑理都懒得理他,撇过头假装欣赏蓝天。

“洪大哥,南淮街那家老字号馒头铺,$\cdots\cdots$”

“洪大哥,天风道那家私房菜里的馒头,$\cdots\cdots$”

“洪大哥,山阳寺那里的和尚做的馒头,$\cdots\cdots$”

你烦不烦!来来去去除了馒头就是馒头,我就是那么像馒头的人吗?洪小黑恨不得破口大骂。

岳飞羽见洪小黑对他仍旧不理不睬,急得搔首挠头,毕竟他心里还存着残念,想要通过《天铸开物引》将他心中最完美的发簪打造出来送给无双。

“洪大哥,要不我明天亲自给您做馒头------”岳飞羽话未说完,只听啪的一声,他的嘴巴就被挡住了。原来洪小黑被他的馒头馒头馒头说得心烦头大,实在是忍不下去了就随手将最顺手的东西扔了过去。就这么巧,正好打到了岳飞羽的嘴巴,于是馒头声终于停了,世界清静了。

岳飞羽不停地告诫自己君子不与小人斗,何况现在还有求于人呢。同时接过从嘴巴上滑落的东西,定睛一看,得来全不费工夫,原来洪小黑扔过来的正是那本《天铸开物引》。岳飞羽拿到了书,心花怒放,也不去与洪小黑再作计较,毕竟今日在他手上吃的亏已经够多了,开局不顺,来日择个吉日再与他斗一场扳回场子罢。

“洪小黑,我要去吃平安巷的馒头,你来不?”岳飞羽失而复得,心情大好,真心实意地发出邀请。

“去,不吃白不吃!”洪小黑失手弄丢了谈判的筹码,恼怒地回答。

“岳小八,你提着个小姑娘的盒子干什么?难不成你有龙阳之癖?”一路上,洪小黑再次发扬了他优良风格,再次动用了他的言语武器。

岳飞羽决定不理他,否则只会自讨无趣。

见岳飞羽不回话不上当,洪小黑决定换个策略,先从他关注点入手。“岳小八,你要那本破破烂烂的天铸引干什么?”

“才不破烂呢,肯定是到了你手上被你弄破的。”岳飞羽反驳说。忽然他反应过来了,他一个小小的叫化子,如何认得了天铸开物引这几个字?“洪小黑,你怎么知道那本是什么书的?”

“你洪小爷天赋异秉,天生就会。世界上没有什么东西是你家洪小爷不认识的。”

岳飞羽也不在乎他吹牛,他关心的是另一件事。“那你肯定看得懂咯?”

“废话,都说了你洪小爷上知天文下知地理,区区一本天铸引,能难得到我?”

“洪大哥,你可真是我的大福星!今天的馒头,你要多少有多少!”岳飞羽想不到竟然在洪小黑这里找到了线索,激动得抱着洪小黑跳了起来。

“啊,你想干什么?勒死我了,快放我下来。”洪小黑黝黑的脸被他勒得都要变红了,一时竟然忘了自己力气比岳飞羽还大,没作挣扎。

岳飞羽嘿嘿一笑,拿出他的发簪杰作图给洪小黑看。“二叔说那本《天铸开物引》里可能有打造这个发簪的方法。快帮我看看有没有。”

洪小黑接过去,先给了他一个鄙视的眼神后再去看向那幅发簪设计图。“幼稚,无聊。”洪小黑手上作势就要将那纸张撕碎,眼睛却是偷偷地看向岳飞羽。

岳飞羽却是没有注意到他的小动作,急忙喊到:“不要撕!不要撕!”

洪小黑见好就收,笑吟吟地说:“岳小八,你的春天来了啊?看上谁家的姑娘了?让你洪大爷帮你把把关吧?”

“什么乱七八糟的,我看上谁家姑娘了?这个发簪我是要做了送给我姐的!”

“女大三,抱金砖。你还挺识货的。”

“住口!不要再胡闹了!那是我亲姐姐!”岳飞羽怒了,有些事他不会忍。

见他如此,洪小黑便知玩笑开大了,吐了吐舌头。为了补偿,便认真地看起那张发簪图。看完后便快速地在《天铸开物引》中翻查起来。“木之灵,在山之阳。$\cdots\cdots$,灌之以气,熏之以光。”洪小黑默默念叨着。过了良久,只见他展眉一笑:“真的是馒头管够?”啊呸,怎么我也说起馒头来了,被传染了被传染了。真是近朱者赤近墨者黑。可是说出去的话,泼出去的水,洪小黑也不是个无赖之人。

“管够管够!”岳飞羽听他如此说,知道他有办法了,激动得连忙点头,生怕他反悔不帮忙了。

“我要今天,明天,后天,不,一直到大大大大大大大后天都要管够!”既然是馒头,那就不跟他客气了。洪小黑一下子要了好多天的馒头。这下我真的成了洪馒头了,他暗暗自嘲。

“没问题,管够!”岳飞羽大方的回应。“那么,洪大哥,这个发簪应该如何打造呢?”

“我又不是铁匠,我怎么会?”

“可是你不是说你知道吗?”

“我是知道啊,可是我不会呀。有问题吗?”

你是没问题,可是我有问题啊。岳飞羽着急了,急则生乱,乱则丢了他本是遇事不惊的优势。

“你先找人把这发簪形状给打造出来呀,真是的,叫你岳小八真没叫错,榆木脑袋才可以活千年。”本着为将来好多天的馒头着想,洪小黑善意地提醒。

“你等着。”说完岳飞羽就跑了。

这小八子还真是跑不死的啊,不是说起床到现在都未进食的吗?不是刚晕醒的吗?感情是天马下凡吧,不吃不喝也能跑。想到这,洪小黑咯咯地自个笑了起来。

岳飞羽再次来到雁南飞的草庐,甫一进门,就二叔二叔地大声嚷到。雁南飞可没想到岳飞羽会杀个马枪,毕竟天赋这东西,有就是有,没有就是没有。天铸这行当,没有天赋的话任凭你折腾一万年也看不出来个门道。雁南飞正在书房里沉迷在他书中的黄金屋和颜如玉里呢,被岳飞羽的大嗓门吓了一跳,赶紧跑出来。

“只管用打铁的方式打出来?这个倒是简单的紧。”毕竟是小孩心性,雁南飞心想,也没去多想多问。岳飞羽则是一心想着快点弄好好让洪小黑开始下一步骤,也没多说。

发簪虽小巧,但设计者毕竟只是个小孩儿,再加上雁南飞本是个高明的铁匠,很快发簪就被打造出来了。岳飞羽定睛一看,开口夸到:“二叔果然巧手,这发簪就跟图中的一模一样。”岳飞羽虽是赞不绝口,脚下功夫却也没停,边赞边跑,话音未落便消失于草庐门外,进来时随手放在地上的蝶香斋小盒也不要了。

很快,岳飞羽便再次出现在洪小黑面前。洪小黑拿着雁南飞打造的发簪,他从未近距离观摩过如此精致的东西,竟有些爱不释手,细细把玩。岳飞羽可容不得洪小黑耽误时间,不停地催促他。

“看到这发簪上中间这一点没?画龙要点睛,做簪要点灵。用天铸引上面的话就是需要在上面灌灵。但你这个东西太幼稚了,太粗糙了,随便用点什么垃圾填上去就可以了。”

岳飞羽听他前面说的还挺正经,听到后面被他埋汰得脸都红了,但想到有求于人,只好忍着不作反抗,小心翼翼地问道:“那需要用些什么东西来填呢?”

“木之灵,在山之阳。要去山南采啊。得亏靖安在东山的南边,要是在北边的话,还得翻过东山到山对面才行。要是那样的话,你这几天的馒头可不够的。”洪小黑认真地算起了账。

岳飞羽听得此言,也不问他要采什么东西了,拉着洪小黑就往东山跑。

“我的馒头还没吃呢!”

“回来给你双份!”