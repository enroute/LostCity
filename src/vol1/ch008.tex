\chapter{作案未遂逍遥法外,行侠仗义难逃法网}

岳飞羽对着大尾招了招手,大尾闭目养神假装没看到他。果然是女大不中留,想来这大尾是个母的吧。岳飞羽歪头看了洪小黑一会,只觉对此人有莫名其妙的亲切感,再加上实是无法还他那二十天馒头,便说:“只要它愿意,大尾便归你了。”

听得此言,洪小黑大喜过望,大跳起来,吓得那大尾也在他肩上跟着跳了一下,差点没掉下地。他紧紧地握住大尾的一只后爪,生怕它跑了。

“我们要赶路了。”洪小黑向他的七叔叔跑去,生怕岳飞羽反悔将大尾要回去。他拉着那青年乞丐沿着官道快步离开了。

岳飞书再坐了一会,觉得无趣,便回了猎户家。雁南飞已经回来了,他铁青着脸,站在屋门前一言不发。岳府之事闹得沸沸扬扬,他没花什么功夫便打听清楚,料来那救了沐夫人的仙女便是双儿。只是奇怪双儿既入仙门,便不能再管凡间事,她师尊又怎会袖手旁观让她出手?原来此间世界有仙凡之分,二者素不相干,仙门自有仙门的约束,凡间亦有凡间的法则。

雁南飞只是空有点打铁力气的一介文人,无处用力,四处暗访过后未找到岳府其余下人的下落后便只得先行回到猎户家再作打算。

岳飞羽见他一脸不善,心想糟了,二叔叫我不要乱跑,该不会是生我气了吧。他打小点子多敢做敢为,为此没少被岳东来收拾,但也是攒下了不少的挨训抗揍经验。

嬉皮笑脸是不行的,只会火上浇油。苦情戏也是不行的,一眼便知是在作弊。岳飞羽的窍门是,坦白从宽,外加一点点半真半假的借口。于是他上前坦白:“二叔,我未听从你的嘱咐,走了出门,请二叔责罚。”

雁南飞见他平安无事,本无意要责罚他。见他如此说来,所谓执法必严,不罚也不行了,便罚他在附近砍柴。可怜岳飞羽聪明反被聪明误。

岳飞羽拿着猎户那被磨得锃亮的斧头,对着一颗枯死的小树用力地砍着,不一会便累得浑身是汗,口干舌燥想要找水喝。他举起手以袖擦汗。用力之下,只觉那被洪小黑赏过一爆栗的额头被袖中一硬物硌得隐隐作痛。




\section{}
\section{}
\section{}

后有说书人将此案编成故事,在茶馆中供人消遣并以儆效尤,其题曰:“欺男霸女莫无天,作案未遂逍遥法外;扶危救难雁南飞,行侠仗义难逃法网。”此是后话,不表。




\chapter{善藏于民}
雁南飞拼死维护岳飞羽不遭牵连,被关了县衙大狱。岳飞羽含泪

\section{}
\section{}
\section{}
\section{}



\chapter{国之为国,在于麓山}
\section{麓山书院}


\section{国将不国}
古镇岳背对着他,负手而立,缓缓说道:“大汉八百万人口,谁都可以变腐,唯麓山一脉不可。若我麓山一腐,往后百年,庙堂之上将难有匡扶正义之士,大汉根基必轰然倒塌,国将不国矣。”

一阵山风从门口吹进来,只吹得古镇岳长发与袍摆乱舞,在飒飒的秋风声中显得甚是萧瑟。

英武侯非他弟子,平时只见他玩世不恭,哪里见过他如此之态。回想起老师之言,方知院长果然非一般人。当下涌起一腔热血,朗声说道:“谨听院长教诲,弟子知道如何做了。”他对着古镇岳行了一个大礼,转身便走了。

\section{}
\section{}

