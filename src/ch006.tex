
\chapter{牧童遥指杏花村}
\label{chap:mu-tong-yao-zhi-xing-hua-cun}

密道里黑得伸手不见五指,只闻得到一股久了不通风的霉味。雁南飞在密道中摸索到了岳飞羽及沐夫人的包裹,深一步浅一步地往前挪动。密道狭窄,呯的一声雁南飞背上的岳飞羽撞到了密道顶上凸出来的石块,带得雁南飞一个踉跄摔倒在地,身上的包裹也掉地上了。

“羽儿,你没事吧?”雁南飞急忙问道。见无人回应,才想起岳飞羽仍在昏迷当中。

雁南飞伸手要去摸那包裹,胡乱摸索间手臂竟用力将那包裹拔了一下。只见那摔下来后本已有些松散的包裹此刻发出些微弱的光芒,显然是包裹松开之后里面物品暴露了出来。就着这些光芒,雁南飞打开包裹查看,一张厚厚的黑布里赫然包裹着一颗拳头般大小的夜明珠!在夜明珠的照耀下,密道登时亮如黄昏,一切地形与布置看得清清楚楚。雁南飞在珠光下检查了包裹中的物品,除了那夜明珠外,还有一个散发着淡淡蓝光的玉佩,玉佩入手亦温亦凉,一面雕刻着喷火的五爪金龙,一面雕刻着一昂首飞翔的凤凰,一龙一凤栩栩如生。此外包裹里还有一把暗淡无光的短匕首,以及金叶、碎银与银票若干。

“大嫂倒想得周到。”雁南飞收拾了包裹,再次背起岳飞羽,依着夜明珠有亮光顺着密道往前走。

走出百米距离,便听得轰的一声闷响,刚走过的密道坍塌了!滚滚尘土向雁南飞二人直扑而来,良久才平息下来。雁南飞二人除了满身灰尘之外,倒是毫发无损。

雁南飞一阵默然,只觉喉间哽塞。以他之智,自是知道沐夫人此举之意,是下定了决心要为他二人争取时间。雁南飞虽非武林中人,但也豪迈非凡,很快便拾掇了心绪,再次快步而去。

雁南飞以步计时,算着在密道中走了大概有三个时辰,终于依稀见着了前方有微弱的亮光。走近了看,原来天色已晚,那亮光是从密道尽头的洞口处倾洒下来的月光与星光。雁南飞心中计算了一番,估算现在位置大约是在东山山脉深处,便谨慎起来。走到洞口一看,果然洞口开在了悬崖峭壁之上,向下看黑漆漆一片不知深浅。

雁南飞心想唯有等天亮再做打算,便寻了个靠里的位置将岳飞羽安置好,他自己坐在外侧以防岳飞羽翻身坠落山崖。他一边闭目养神,一边将日间之事在心中反复推演,也没理出个靠谱的头绪,便也缓缓地睡去了。




\section{密道逃脱}


\section{翻越东山}
\section{再见断刀客}
\section{遥指杏花村}

雁南飞与岳飞羽二人改头换面,再历经长途跋涉,早已是面目全非,


\chapter{作案未遂逍遥法外,行侠仗义难逃法网}


\section{}
\section{}
\section{}

后有说书人将此案编成故事,在茶馆中供人消遣并以儆效尤,其题曰:“欺男霸女莫无天,作案未遂逍遥法外;扶危救难雁南飞,行侠仗义难逃法网。”此是后话,不表。




\chapter{善藏于民}
雁南飞拼死维护岳飞羽不遭牵连,被关了县衙大狱。岳飞羽含泪

\section{}
\section{}
\section{}
\section{}



\chapter{国之为国,在于麓山}
\section{麓山书院}


\section{国将不国}
古镇岳背对着他,负手而立,缓缓说道:“大汉八百万人口,谁都可以变腐,唯麓山一脉不可。若我麓山一腐,往后百年,庙堂之上将难有匡扶正义之士,大汉根基必轰然倒塌,国将不国矣。”

一阵山风从门口吹进来,只吹得古镇岳长发与袍摆乱舞,在飒飒的秋风声中显得甚是萧瑟。

英武侯非他弟子,平时只见他玩世不恭,哪里见过他如此之态。回想起老师之言,方知院长果然非一般人。当下涌起一腔热血,朗声说道:“谨听院长教诲,弟子知道如何做了。”他对着古镇岳行了一个大礼,转身便走了。

\section{}
\section{}

