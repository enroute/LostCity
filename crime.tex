(要摆脱毛利小五郞的阴影)私人侦探:主要业务是找猫找狗的琐事,偶尔也会有贵妇来查花心老公


人物结构:

雷震天(当朝开国皇帝,70,快要入土了)
15开始征战天下,历经10年的征伐建立了大汉帝国

周历元年
雷震天 25,称周武帝







跟踪的方法:
尾随,交换,分段,迂回

       
仵作,子孙四代不能参加科考,除却走投无路,一般人不会却干这个

解剖尸体是现代法医的主要业务之一,但在百多年前的广州城里,解剖还是大忌,仵作多是通过尸表检验,来推断伤害。但若是陈
年旧案,尸身已腐烂殆尽,只留下一副白骨,仵作就要施展检骨绝技了。简单来说,就是挖个地窖,或找一个大瓮,将骨头洗干净后,
与醋一起,放进去,或蒸或煮,烧滚后再拿出来,细查有无破裂、血痕,骨头色泽有无变化,以推断生前受伤或中毒情况。仔细一想,
这个场景还是满“重口味”的,不知刚做完骨检的仵作,端起排骨汤的时候会不会有点不适应?


汉朝官制
上公:荣誉职位,太师,太傅,太保
三公:大司徒(丞相,相国,左右丞相()若两人,行政长官),大司马(太尉,军政),大司空(御史大夫,监察)
九卿:总管庶政,
九卿之外:列卿宫官,将军尚书台